% !TeX root = raytracing/slides/main.tex
% raytracing_preamble.tex
% Common styling for Ray Tracing presentation

\documentclass[10pt]{beamer}
\usetheme{metropolis}
\usefonttheme{professionalfonts}

% --- Color Definitions ---
\definecolor{PrimaryColor}{RGB}{33,52,72}         % Deep navy
\definecolor{SecondaryColor}{RGB}{84,119,146}     % Desaturated blue-gray
\definecolor{AccentColor}{RGB}{100, 156, 165}       % Soft steel blue

\definecolor{BackgroundColor}{RGB}{245,247,250}   % Very light cool gray/blue-tinted white
\definecolor{TextColor}{RGB}{40,55,70}            % Darker, cool-toned charcoal (slightly less saturated)
\definecolor{LightGray}{RGB}{220,225,230}         % Soft, cool light gray
\definecolor{DarkGray}{RGB}{70,90,105}            % Cool-toned dark gray (coherent with Primary/Secondary)

\definecolor{RayColor}{RGB}{230,150,80}           % Muted warm orange (less saturated than pure orange)
\definecolor{ObjectColor}{RGB}{128,101,160}       % Muted violet (adjusted purple to match cool palette)
\definecolor{LightColor}{RGB}{235,200,100}        % Soft golden yellow (to complement cool blues)

% --- Theme Customization ---
\setbeamercolor{background canvas}{bg=BackgroundColor}
\setbeamercolor{normal text}{fg=TextColor}
\setbeamercolor{frametitle}{bg=PrimaryColor, fg=white}
\setbeamercolor{section in toc}{fg=PrimaryColor}
\setbeamercolor{block title}{bg=PrimaryColor!80, fg=white}
\setbeamercolor{block body}{bg=PrimaryColor!10}
\setbeamercolor{alerted text}{fg=AccentColor}
\setbeamercolor{itemize item}{fg=PrimaryColor}
\setbeamercolor{itemize subitem}{fg=SecondaryColor}
\setbeamerfont{frametitle}{size=\large,series=\bfseries}

% --- Packages ---
\usepackage[utf8]{inputenc}
\usepackage{url}
\usepackage{booktabs}
\usepackage{amsmath, amssymb}
\usepackage{fontawesome5}
\usepackage{pifont}
\usepackage[most]{tcolorbox}
\tcbuselibrary{skins}
\usepackage{colortbl}
\usepackage{array}
\usepackage{tikz}
\usepackage{graphicx}
\usetikzlibrary{shapes.callouts, positioning, arrows.meta, shapes.geometric, shadows, calc, patterns, 3d, backgrounds, shadings}
\usepackage{adjustbox}
\usepackage{ragged2e}
\usepackage{pgfplots}
\usepackage{graphicx}
\usepackage{caption}
\usepackage{minted}
\pgfplotsset{compat=1.18}

% --- Custom Commands ---
\newcommand{\cmark}{\textcolor{SecondaryColor}{\ding{51}}}
\newcommand{\xmark}{\textcolor{AccentColor}{\ding{55}}}
\newcommand{\highlight}[1]{\textcolor{PrimaryColor}{\textbf{#1}}}
\newcommand{\raycolor}[1]{\textcolor{RayColor}{\textbf{#1}}}
\newcommand{\objectcolor}[1]{\textcolor{ObjectColor}{\textbf{#1}}}

% --- TikZ styles for ray tracing visualizations ---
\tikzstyle{process} = [rectangle, rounded corners=3mm, minimum width=2cm, minimum height=0.8cm, text centered, draw=PrimaryColor, thick, fill=PrimaryColor!15, drop shadow]
\tikzstyle{arrow} = [thick, PrimaryColor, ->, >=stealth]
\tikzstyle{ray} = [thick, RayColor, ->, >=stealth]
\tikzstyle{lightray} = [thick, LightColor, ->, >=stealth]
\tikzstyle{reflectray} = [thick, SecondaryColor, ->, >=stealth]
\tikzstyle{refractray} = [thick, AccentColor, ->, >=stealth]
\tikzstyle{shadowray} = [thick, DarkGray, ->, >=stealth, dashed]

% --- 3D object styles ---
\tikzstyle{sphere} = [circle, minimum size=1.5cm, draw=ObjectColor, thick, fill=ObjectColor!20, drop shadow]
\tikzstyle{plane} = [rectangle, minimum width=3cm, minimum height=0.2cm, draw=ObjectColor, thick, fill=ObjectColor!20]
\tikzstyle{triangle} = [regular polygon, regular polygon sides=3, minimum size=1.5cm, draw=ObjectColor, thick, fill=ObjectColor!20]

% --- Eye/Camera styles ---
\tikzstyle{eye} = [circle, minimum size=0.8cm, draw=PrimaryColor, thick, fill=PrimaryColor!30]
\tikzstyle{pixel} = [rectangle, minimum size=0.2cm, draw=AccentColor, fill=AccentColor!30]

% --- Text box styles ---
\tikzstyle{conceptbox} = [rectangle, rounded corners, fill=PrimaryColor!10, draw=PrimaryColor, thick, text width=0.8\textwidth, inner sep=8pt]
\tikzstyle{formulabox} = [rectangle, rounded corners, fill=SecondaryColor!10, draw=SecondaryColor, thick, inner sep=6pt]

% --- Custom environments ---
\newtcolorbox{raybox}[1]{
  colback=RayColor!10,
  colframe=RayColor,
  title=#1,
  fonttitle=\bfseries,
  sharp corners
}

\newtcolorbox{conceptbox}[1]{
  colback=PrimaryColor!10,
  colframe=PrimaryColor,
  title=#1,
  fonttitle=\bfseries,
  rounded corners
}

\newtcolorbox{mathbox}[1]{
  colback=SecondaryColor!10,
  colframe=SecondaryColor,
  title=#1,
  fonttitle=\bfseries,
  rounded corners
}

\newenvironment{timeline}{%
  \begin{tikzpicture}[scale=0.8]
    \coordinate (start) at (0,0);
    \newcounter{timelineitem}
    \setcounter{timelineitem}{0}
  }{%
  \end{tikzpicture}
}

\newcommand{\timelineitem}[4]{%
  \stepcounter{timelineitem}%
  % Name of this item’s node
  \edef\thisID{time\thetimelineitem}%
  \ifnum\value{timelineitem}=1
  % First item: anchor at (start)
  \node[rectangle, draw, fill=PrimaryColor!20,
  text width=1.5cm, minimum height=0.8cm, align=center]
  (\thisID) at (start) {\textbf{#2}};
  \else
  % Later items: below=#1 of previous time node
  \pgfmathtruncatemacro\prev{\value{timelineitem}-1}%
  \node[rectangle, draw, fill=PrimaryColor!20,
    text width=1.5cm, minimum height=0.8cm, align=center,
  below=#1 of time\prev]
  (\thisID) {\textbf{#2}};
  \fi
  % Description always to the right of this time node
  \node[rectangle, draw, fill=SecondaryColor!10,
    text width=4cm, minimum height=0.8cm,
  align=left, right=0.3cm of \thisID]
  (desc\thetimelineitem) {\textbf{#3}\\#4};
  % Draw arrow back to previous if not the first
  \ifnum\value{timelineitem}>1
  \draw[->, thick, PrimaryColor]
  (time\prev.south) -- (\thisID.north);
  \fi
}

\tikzset{
  camera/.style={fill=PrimaryColor!60, draw=PrimaryColor!80, rectangle, minimum size=8pt},
  image plane/.style={fill=AccentColor!10, draw=AccentColor!50, opacity=0.8},
  pixel/.style={fill=AccentColor!60, thick},
  primary ray/.style={->, very thick, red!90},
  object/.style={fill=ObjectColor!60, draw=ObjectColor!80, circle, minimum size=12pt},
  fovangle/.style={<->, thick, PrimaryColor, dashed}
}

\tikzset{
  lens/.style={thick, PrimaryColor, line width=3pt},
  focal plane/.style={thick, AccentColor},
  object ray/.style={->, thick, ObjectColor},
  image ray/.style={->, thick, SecondaryColor},
  optical axis/.style={dashed, gray}
}

\newcommand{\lens}[3]{%
  \begingroup
  % half‐dimensions
  \pgfmathsetlengthmacro{\a}{#2}%
  \pgfmathsetlengthmacro{\b}{#3}%
  \pgfmathsetlengthmacro{\c}{#3/2}%
  \begin{scope}[shift={#1}]
    \draw[line join=round, fill=blue!15]
    (0,-{\c})
    arc(-30:30:{\a} and {\b})
    arc(150:210:{\a} and {\b})
    ;
  \end{scope}
  \endgroup
}
