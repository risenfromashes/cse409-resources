\section{Physically Based Rendering Overview}

\begin{frame}{Beyond Phong - Limitations}
    \begin{columns}
        \begin{column}{0.6\textwidth}
            \begin{conceptbox}{Phong Model Problems}
                \textbf{Non-physical behavior:}
                \begin{itemize}
                    \item Energy is not conserved
                    \item Materials can reflect more light than received
                    \item Arbitrary parameter relationships
                \end{itemize}
                
                \vspace{0.3cm}
                \pause
                \textbf{Limited material types:}
                \begin{itemize}
                    \item Only plastic-like materials look good
                    \item Metals don't look realistic
                    \item No clear mapping to real material properties
                \end{itemize>
                
                \vspace{0.3cm}
                \pause
                \textbf{Artistic inconsistency:}
                \begin{itemize}
                    \item Same material looks different under different lighting
                    \item Hard to achieve photorealistic results
                    \item Parameters need tweaking for each scene
                \end{itemize>
            \end{conceptbox}
        \end{column>
        \begin{column>{0.4\textwidth>
            % IMAGE: Phong limitations
            % Show examples of unrealistic Phong materials vs real materials
            % \includegraphics[width=\linewidth]{images/phong_limitations.jpg}
            \vspace{2cm>
            \textcolor{gray>{[Phong model limitations examples]}
            
            \vspace{0.5cm>
            \begin{raybox>{Energy Conservation Issue}
                \footnotesize
                In Phong: $k_a + k_d + k_s$ can be $> 1$
                
                This means surface reflects more energy than it receives!
                
                \textbf{Physically impossible}
            \end{raybox>
        \end{column>
    \end{columns>
\end{frame>

\begin{frame}{The Rendering Equation}
    \begin{mathbox>{Kajiya's Rendering Equation (1986)}
        \textbf{Complete light transport equation:}
        \begin{align>
            L_o(\mathbf{p}, \omega_o) = L_e(\mathbf{p}, \omega_o) + \int_{\Omega} f_r(\mathbf{p}, \omega_i, \omega_o) L_i(\mathbf{p}, \omega_i) (\mathbf{n} \cdot \omega_i) d\omega_i
        \end{align>
        
        where:
        \begin{itemize>
            \footnotesize
            \item $L_o(\mathbf{p}, \omega_o)$ = outgoing radiance at point $\mathbf{p}$ in direction $\omega_o$
            \item $L_e(\mathbf{p}, \omega_o)$ = emitted radiance (if surface is a light source)
            \item $f_r(\mathbf{p}, \omega_i, \omega_o)$ = bidirectional reflectance distribution function (BRDF)
            \item $L_i(\mathbf{p}, \omega_i)$ = incoming radiance from direction $\omega_i$
            \item $\mathbf{n}$ = surface normal
            \item $\Omega$ = hemisphere above the surface
        \end{itemize}
        
        \vspace{0.3cm>
        \pause
        \textbf{Physical interpretation:} 
        
        Light leaving a surface = light emitted + light reflected from all incoming directions
    \end{mathbox>
\end{frame>

\begin{frame}{Modern PBR Material Model}
    \begin{columns>
        \begin{column>{0.6\textwidth>
            \begin{mathbox>{PBR Parameters}
                \textbf{Albedo/Base Color:} Diffuse color of the material
                \begin{align>
                    \text{Albedo} \in [0, 1]^3
                \end{align>
                
                \pause
                \textbf{Metallic:} How metallic the material is
                \begin{align>
                    \text{Metallic} \in [0, 1]
                \end{align>
                
                \pause
                \textbf{Roughness:} How rough the surface is
                \begin{align>
                    \text{Roughness} \in [0, 1]
                \end{align>
                
                \pause
                \textbf{Energy conservation built-in:}
                \begin{itemize>
                    \item Diffuse and specular are automatically balanced
                    \item Metals have no diffuse component
                    \item Physically plausible by design
                \end{itemize>
            \end{mathbox>
        \end{column>
        \begin{column>{0.4\textwidth>
            % IMAGE: PBR material examples
            % Show material chart with different metallic/roughness combinations
            % \includegraphics[width=\linewidth]{images/pbr_materials.jpg}
            \vspace{2cm>
            \textcolor{gray>{[PBR material examples]}
            
            \vspace{0.3cm>
            \begin{conceptbox>{Key Insight}
                \footnotesize
                \textbf{Fewer parameters} with \textbf{physical meaning}
                
                Artists can work with real-world material properties
                
                Same material looks correct under all lighting conditions
            \end{conceptbox>
        \end{column>
    \end{columns>
\end{frame>

\begin{frame}{Modern Rendering - Real-time PBR}
    \begin{columns>
        \begin{column>{0.5\textwidth>
            \begin{raybox>{Real-time Applications}
                \textbf{Game engines:}
                \begin{itemize>
                    \item Unreal Engine
                    \item Unity
                    \item Godot
                    \item Custom engines
                \end{itemize>
                
                \vspace{0.3cm>
                \textbf{Approximations for speed:}
                \begin{itemize>
                    \item Environment maps for global illumination
                    \item Screen-space reflections
                    \item Precomputed lighting
                    \item Simplified BRDFs
                \end{itemize>
                
                \vspace{0.3cm>
                \textbf{Hardware support:}
                \begin{itemize>
                    \item GPU ray tracing (RTX)
                    \item Specialized lighting units
                    \item Hardware-accelerated BRDFs
                \end{itemize>
            \end{raybox>
        \end{column>
        \begin{column>{0.5\textwidth>
            % IMAGE: Modern game graphics
            % Show screenshot from modern game with realistic PBR materials
            % \includegraphics[width=\linewidth]{images/modern_game_graphics.jpg}
            \vspace{2cm>
            \textcolor{gray>{[Modern game graphics with PBR]}
            
            \vspace{0.3cm>
            \begin{conceptbox>{Path Tracing}
                \footnotesize
                \textbf{Ultimate goal:} Solve rendering equation completely
                
                \textbf{Method:} Monte Carlo ray tracing
                
                \textbf{Recent advances:} Real-time path tracing in games (2023+)
            \end{conceptbox>
        \end{column>
    \end{columns>
    
    \vspace{0.3cm>
    \pause
    \begin{center>
        % IMAGE: Evolution of rendering
        % Show timeline: Phong → Blinn-Phong → PBR → Path Tracing
        % \includegraphics[width=\linewidth]{images/rendering_evolution.jpg}
        \textcolor{gray>{[Evolution of real-time rendering]}
    \end{center>
\end{frame>
