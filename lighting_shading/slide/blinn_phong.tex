\section{Blinn-Phong Model}

\begin{frame}{Limitations of Phong Specular}
  \begin{columns}
    \begin{column}{0.6\textwidth}
      \begin{conceptbox}{Phong Model Issues}
        \textbf{Computational cost:}
        \begin{itemize}
          \item Reflection vector calculation expensive
          \item Requires 2 dot products + vector operations
          \item Per-light calculation needed
        \end{itemize}

        \vspace{0.3cm}
        \pause
        \textbf{Visual artifacts:}
        \begin{itemize}
          \item Highlights can "pop" on/off abruptly
          \item Non-physical behavior at grazing angles
          \item Issues with interpolation across polygons
          \end{itemize>

            \vspace{0.3cm}
            \pause
            \textbf{Hardware concerns:}
            \begin{itemize}
              \item Reflection calculation not GPU-friendly
              \item Hard to optimize for real-time rendering
              \end{itemize>
              \end{conceptbox}
            \end{column>
              \begin{column}{0.4\textwidth}
                \begin{tikzpicture}[scale=0.8]
                  % Surface
                  \draw[ObjectColor, very thick] (-2,0) -- (2,0);
                  \fill[ObjectColor] (0,0) circle (3pt);

                  % Normal
                  \draw[->, PrimaryColor, thick] (0,0) -- (0,1.5);
                  \node[right] at (0.1,0.75) {\footnotesize $\mathbf{N}$};

                  % Light vector
                  \draw[->, lightray, thick] (0,0) -- (-1.2,1.2);
                  \node[above left] at (-0.6,0.6) {\footnotesize $\mathbf{L}$};

                  % View vector
                  \draw[->, AccentColor, thick] (0,0) -- (1.2,1.2);
                  \node[above right] at (0.6,0.6) {\footnotesize $\mathbf{V}$};

                  % Reflection vector (expensive to calculate)
                  \draw[->, reflectray, thick, dashed] (0,0) -- (1.2,1.2);
                  \node[above] at (1.4,1.4) {\footnotesize $\mathbf{R}$ (expensive)};

                  \pause

                  % Problem illustration
                  \node[red, below] at (0,-0.8) {\footnotesize Expensive calculation:};
                  \node[red, below] at (0,-1.1) {\footnotesize $\mathbf{R} = 2(\mathbf{N} \cdot \mathbf{L})\mathbf{N} - \mathbf{L}$};
                \end{tikzpicture>
                \end{column>
                \end{columns>

                  \vspace{0.3cm}
                  \pause
                  % IMAGE: Phong artifacts
                  % Show examples of Phong artifacts: popping highlights, interpolation issues
                  % \includegraphics[width=\linewidth]{images/phong_artifacts.jpg}
                  \textcolor{gray}{[Examples of Phong model artifacts]}
                \end{frame>

                  \begin{frame}{Blinn-Phong Introduction}
                    \begin{columns}
                      \begin{column}{0.5\textwidth}
                        \begin{raybox}{Blinn's Innovation (1977)}
                          \textbf{Key insight:} Instead of using reflection vector $\mathbf{R}$, use halfway vector $\mathbf{H}$

                          \vspace{0.3cm}
                          \textbf{Halfway vector:} Bisects angle between light and view directions

                          \vspace{0.3cm}
                          \textbf{Advantages:}
                          \begin{itemize}
                            \item Cheaper to calculate
                            \item More stable numerically
                            \item Better hardware optimization
                            \item Smoother interpolation
                            \end{itemize>

                              \vspace{0.3cm}
                              \textbf{Trade-off:} Slightly different visual result, but often preferred
                            \end{raybox>
                            \end{column>
                              \begin{column>{0.5\textwidth>
                                    \begin{tikzpicture>[scale=0.8]
                                        % Surface
                                        \draw[ObjectColor, very thick] (-2,0) -- (2,0);
                                        \fill[ObjectColor] (0,0) circle (3pt);

                                        % Normal
                                        \draw[->, PrimaryColor, thick] (0,0) -- (0,1.5);
                                        \node[right] at (0.1,0.75) {\footnotesize $\mathbf{N}$};

                                        % Light vector
                                        \draw[->, lightray, thick] (0,0) -- (-1.2,1.2);
                                        \node[above left] at (-0.6,0.6) {\footnotesize $\mathbf{L}$};

                                        % View vector
                                        \draw[->, AccentColor, thick] (0,0) -- (1.2,1.2);
                                        \node[above right] at (0.6,0.6) {\footnotesize $\mathbf{V}$};

                                        \pause

                                        % Halfway vector
                                        \draw[->, red, very thick] (0,0) -- (0,1.7);
                                        \node[right] at (0.1,1.3) {\footnotesize $\mathbf{H}$};

                                        % Angle bisector indication
                                        \draw[dashed, thin] (0,0) -- (-0.6,1.2);
                                        \draw[dashed, thin] (0,0) -- (0.6,1.2);

                                        \node[below] at (0,-0.5) {\footnotesize Halfway vector bisects};
                                        \node[below] at (0,-0.8) {\footnotesize $\mathbf{L}$ and $\mathbf{V}$};
                                      \end{tikzpicture>
                                      \end{column>
                                      \end{columns>

                                        \vspace{0.3cm>
                                          \pause
                                          \begin{center>
                                              % IMAGE: Blinn-Phong vs Phong comparison
                                              % Show side-by-side spheres rendered with both models
                                              % \includegraphics[width=0.8\linewidth]{images/blinn_phong_comparison.jpg}
                                              \textcolor{gray>{[Blinn-Phong vs Phong visual comparison]}
                                              \end{center>
                                              \end{frame>

                                                \begin{frame}{Half-Vector Mathematics}
                                                  \begin{mathbox>{Halfway Vector Calculation}
                                                      \textbf{Definition:} Vector that bisects the angle between light and view directions

                                                      \vspace{0.3cm>
                                                        \textbf{Simple calculation:}
                                                        \begin{align>
                                                            \mathbf{H} = \frac{\mathbf{L} + \mathbf{V}}{|\mathbf{L} + \mathbf{V}|}
                                                          \end{align>

                                                            where $\mathbf{L}$ and $\mathbf{V}$ are unit vectors

                                                            \vspace{0.3cm>
                                                              \pause
                                                              \textbf{Blinn-Phong specular formula:}
                                                              \begin{align>
                                                                  I_{\text{specular}} = k_s \cdot I_l \cdot \max(0, \mathbf{N} \cdot \mathbf{H})^{n'}
                                                                \end{align>

                                                                  \pause
                                                                  \textbf{Note:} Shininess exponent $n'$ in Blinn-Phong is typically 2-4 times larger than Phong's $n$ for similar visual result

                                                                  \vspace{0.3cm>
                                                                    \textbf{Relationship:} $n' \approx 2n$ to $4n$ (depends on material)
                                                                  \end{mathbox>

                                                                    \vspace{0.3cm>
                                                                      \pause
                                                                      % IMAGE: Half-vector geometry
                                                                      % Show geometric construction of halfway vector with angles
                                                                      % \includegraphics[width=\linewidth]{images/halfway_vector_geometry.jpg}
                                                                      \textcolor{gray>{[Half-vector geometric construction]}
                                                                      \end{frame>

                                                                        \begin{frame}{Blinn-Phong vs Phong Comparison}
                                                                          \begin{columns>
                                                                              \begin{column>{0.5\textwidth>
                                                                                    \begin{mathbox>{Performance Comparison}
                                                                                        \textbf{Phong reflection calculation:}
                                                                                        \begin{align>
                                                                                            \mathbf{R} &= 2(\mathbf{N} \cdot \mathbf{L})\mathbf{N} - \mathbf{L} \\
                                                                                            I &= (\mathbf{R} \cdot \mathbf{V})^n
                                                                                          \end{align>

                                                                                            \textbf{Operations:} 1 dot product, 1 scalar multiply, 2 vector operations, 1 power

                                                                                            \vspace{0.3cm>
                                                                                              \pause
                                                                                              \textbf{Blinn-Phong calculation:}
                                                                                              \begin{align>
                                                                                                  \mathbf{H} &= \text{normalize}(\mathbf{L} + \mathbf{V}) \\
                                                                                                  I &= (\mathbf{N} \cdot \mathbf{H})^{n'}
                                                                                                \end{align>

                                                                                                  \textbf{Operations:} 1 vector add, 1 normalize, 1 dot product, 1 power

                                                                                                  \vspace{0.3cm>
                                                                                                    \pause
                                                                                                    \textbf{Winner:} Blinn-Phong (especially on GPU)
                                                                                                  \end{mathbox>
                                                                                                  \end{column>
                                                                                                    \begin{column>{0.4\textwidth>
                                                                                                          % IMAGE: Performance comparison chart
                                                                                                          % Show bar chart of computation time: Phong vs Blinn-Phong
                                                                                                          % \includegraphics[width=\linewidth]{images/performance_comparison.jpg}
                                                                                                          \vspace{1.5cm>
                                                                                                            \textcolor{gray>{[Performance comparison chart]}

                                                                                                              \vspace{0.5cm>
                                                                                                                \begin{conceptbox>{Visual Differences}
                                                                                                                    \footnotesize
                                                                                                                    \textbf{Blinn-Phong highlights:}
                                                                                                                    \begin{itemize>
                                                                                                                      \item Slightly larger
                                                                                                                      \item Softer falloff
                                                                                                                      \item More stable interpolation
                                                                                                                      \end{itemize>

                                                                                                                        \textbf{Often preferred} for real-time graphics
                                                                                                                      \end{conceptbox>
                                                                                                                      \end{column>
                                                                                                                      \end{columns>

                                                                                                                        \vspace{0.3cm>
                                                                                                                          \pause
                                                                                                                          % IMAGE: Side-by-side visual comparison
                                                                                                                          % Show same scene rendered with Phong vs Blinn-Phong, highlighting differences
                                                                                                                          % \includegraphics[width=\linewidth]{images/phong_blinn_side_by_side.jpg}
                                                                                                                          \textcolor{gray>{[Side-by-side visual comparison of Phong vs Blinn-Phong]}
                                                                                                                          \end{frame>
