
\section{Acceleration Structures}

\begin{frame}{Performance Considerations}
  \begin{columns}
    \begin{column}{0.6\textwidth}
      \begin{raybox}{Computational Complexity}
        \small
        \textbf{Basic ray tracing:}
        \begin{itemize}
          \item $O(n \times m)$ where:
          \item $n$ = number of pixels
          \item $m$ = number of objects
        \end{itemize}

        \vspace{0.3cm}
        \textbf{With secondary rays:}
        \begin{itemize}
          \item Exponential growth with depth
          \item Multiple rays per intersection
        \end{itemize}
      \end{raybox}
    \end{column}
    \begin{column}{0.4\textwidth}
      \begin{tikzpicture}[scale=0.7]
        % Performance graph
        \begin{axis}[
            width=6cm,
            height=4cm,
            xlabel={Scene Complexity},
            ylabel={Render Time},
            grid=major,
            legend pos=north west
          ]
          \addplot[thick, PrimaryColor] coordinates {(1,1) (2,2) (3,3) (4,4) (5,5)};
          \addplot[thick, AccentColor] coordinates {(1,1) (2,4) (3,9) (4,16) (5,25)};
          \legend{Ray Casting, Ray Tracing}
        \end{axis}
      \end{tikzpicture}

      \vspace{0.3cm}
      \textbf{Optimization strategies:}
      \begin{itemize}
        \item Spatial data structures
        \item Parallel processing
        \item Early ray termination
      \end{itemize}
    \end{column}
  \end{columns}
\end{frame}

\begin{frame}{The Naive Approach Problem}
  \begin{center}
    \begin{tikzpicture}[scale=0.8]
      % Ray
      \draw[ray, very thick] (0,0) -- (8,2);
      \node[below] at (0,0) {\raycolor{Ray}};

      % Many objects scattered
      \foreach \x/\y in {1/0.5, 2/1.8, 3/0.2, 4/1.5, 5/0.8, 6/1.2, 7/0.4} {
        \node[sphere, scale=0.5] at (\x,\y) {};
      }

      % Intersection tests
      \foreach \x/\y in {1/0.5, 2/1.8, 3/0.2, 4/1.5, 5/0.8, 6/1.2, 7/0.4} {
        \draw[dashed, red, thin] (0,0) -- (\x,\y);
      }

      \node[below] at (4,-1) {\textcolor{AccentColor}{\textbf{Problem:} Test every object for every ray!}};
      \node[below] at (4,-1.5) {Scene with 1M objects = 1M tests per ray};
    \end{tikzpicture}
  \end{center}

  \begin{conceptbox}{Computational Explosion}
    \small
    \textbf{Complexity:} For $N$ objects and $R$ rays → $O(N \times R)$ intersection tests\\
    \textbf{Real scenes:} Millions of triangles, millions of rays → Billions of tests!
  \end{conceptbox}
\end{frame}

\begin{frame}{Bounding Volume Hierarchy (BVH)}
  \begin{columns}
    \begin{column}{0.6\textwidth}
      \begin{raybox}{The Big Idea}
        \small
        \textbf{Divide and Conquer:}
        \begin{enumerate}
          \item Group objects into bounding boxes
          \item Build hierarchical tree structure
          \item Test ray against boxes first
          \item Only test objects in hit boxes
        \end{enumerate}
      \end{raybox}

      \vspace{0.3cm}
      \textbf{Key Benefits:}
      \begin{itemize}
        \item $O(\log N)$ instead of $O(N)$
        \item Massive speedup for complex scenes
        \item Works with any primitive type
      \end{itemize}
    \end{column}
    \begin{column}{0.4\textwidth}
      \begin{tikzpicture}[scale=0.7]
        \node[rectangle, draw, fill=PrimaryColor!20, minimum width=3cm, minimum height=1cm] (root) at (0,3.5) {Root BBox};

        \node[rectangle, draw, fill=SecondaryColor!20, minimum width=1.3cm, minimum height=0.8cm] (left) at (-1.5,1.5) {Left AABB};
        \node[rectangle, draw, fill=SecondaryColor!20, minimum width=1.3cm, minimum height=0.8cm] (right) at (1.5,1.5) {Right AABB};

        \node[rectangle, draw, fill=AccentColor!20, minimum width=0.8cm, minimum height=0.6cm] (ll) at (-2.2,0) {LL};
        \node[rectangle, draw, fill=AccentColor!20, minimum width=0.8cm, minimum height=0.6cm] (lr) at (-0.8,0) {LR};
        \node[rectangle, draw, fill=AccentColor!20, minimum width=0.8cm, minimum height=0.6cm] (rl) at (0.8,0) {RL};
        \node[rectangle, draw, fill=AccentColor!20, minimum width=0.8cm, minimum height=0.6cm] (rr) at (2.2,0) {RR};

        \draw[arrow] (root) -- (left);
        \draw[arrow] (root) -- (right);
        \draw[arrow] (left) -- (ll);
        \draw[arrow] (left) -- (lr);
        \draw[arrow] (right) -- (rl);
        \draw[arrow] (right) -- (rr);

        \node[below] at (0,-1) {\small \raycolor{Ray tests hierarchy top-down}};
      \end{tikzpicture}
    \end{column}
  \end{columns}
\end{frame}

\begin{frame}{BVH Construction and Traversal}
  \begin{columns}
    \begin{column}{0.7\textwidth}
      \begin{mathbox}{Construction Algorithm}
        \small
        \textbf{Recursive subdivision:}
        \begin{enumerate}
          \item Compute bounding box for all objects
          \item Choose split axis (longest dimension)
          \item Sort objects by centroid
          \item Split into two groups
          \item Recursively build subtrees
        \end{enumerate}

        \vspace{0.3cm}
        \textbf{Split strategies:}
        \begin{itemize}
          \item \textbf{Surface Area Heuristic (SAH)}
          \item Median split
          \item Spatial splits
        \end{itemize}
      \end{mathbox}
    \end{column}
    \begin{column}{0.3\textwidth}
      \begin{tikzpicture}[scale=0.7]
        % Scene with objects and bounding boxes
        \draw[thick, PrimaryColor] (-0.5,-0.5) rectangle (3.5,2.5);
        \draw[thick, SecondaryColor] (-0.3,-0.3) rectangle (1.3,1.3);
        \draw[thick, SecondaryColor] (1.7,0.7) rectangle (3.3,2.3);

        % Objects
        \node[sphere, scale=0.3] at (0,0) {};
        \node[sphere, scale=0.3] at (1,1) {};
        \node[triangle, scale=0.3] at (2,1) {};
        \node[sphere, scale=0.3] at (3,2) {};

        % Split line
        \draw[dashed, AccentColor, thick] (1.5,-0.5) -- (1.5,2.5);
        \node[right] at (1.6,0) {\textcolor{AccentColor}{Split}};

        \node[below] at (1.5,-1) {\small Spatial subdivision};
      \end{tikzpicture}
    \end{column}
  \end{columns}
\end{frame}

\subsection{Hardware Ray Tracing Revolution}

\begin{frame}{The Hardware Revolution}
  \begin{center}
    \scriptsize
    \begin{tikzpicture}[scale=0.9]
      % Timeline
      \draw[thick, PrimaryColor] (0,0) -- (10,0);

      % Milestones
      \node[above] at (1,0.5) {\textbf{2018}};
      \node[below] at (1,-0.5) {\textcolor{SecondaryColor}{NVIDIA RTX 20}};
      \fill[SecondaryColor] (1,0) circle (3pt);

      \node[above] at (3,0.5) {\textbf{2020}};
      \node[below] at (3,-0.5) {\textcolor{AccentColor}{RTX 30 Series}};
      \fill[AccentColor] (3,0) circle (3pt);

      \node[above] at (5,0.5) {\textbf{2021}};
      \node[below] at (5,-0.5) {\textcolor{ObjectColor}{Vulkan RT}};
      \fill[ObjectColor] (5,0) circle (3pt);

      \node[above] at (7,0.5) {\textbf{2022}};
      \node[below] at (7,-0.5) {\textcolor{RayColor}{RDNA3, RTX 40}};
      \fill[RayColor] (7,0) circle (3pt);

      \node[above] at (9,0.5) {\textbf{2025}};
      \node[below] at (9,-0.5) {\textcolor{red}{RTX 50}};
      \fill[red] (9,0) circle (3pt);

      \node[below] at (4,-2) {\large \textcolor{PrimaryColor}{\textbf{From Software to Silicon}}};
    \end{tikzpicture}
  \end{center}

  \begin{columns}
    \small
    \begin{column}{0.5\textwidth}
      \textbf{Hardware Features:}
      \begin{itemize}
        \item Dedicated RT cores
        \item Hardware BVH traversal
        \item Triangle intersection units
        \item Tensor cores for denoising
      \end{itemize}
    \end{column}
    \begin{column}{0.5\textwidth}
      \textbf{Performance Impact:}
      \begin{itemize}
        \item 10-100x speedup
        \item Real-time ray tracing in games
        \item Interactive path tracing
        \item AI-accelerated denoising
      \end{itemize}
    \end{column}
  \end{columns}
\end{frame}

\begin{frame}{RTX Architecture}
  \begin{center}
    \begin{tikzpicture}[scale=0.8]
      \draw[thick, PrimaryColor] (0,0) rectangle (8.5,5);
      \node[above] at (4,5.3) {\textbf{RTX GPU Architecture}};

      \foreach \x in {0.5,2.5,4.5,6.5} {
        \draw[SecondaryColor, thick] (\x,3.5) rectangle (\x+1.5,4.8);
        \node at (\x+0.75,4.15) {\textcolor{SecondaryColor}{\textbf{SM}}};
      }

      \foreach \x in {1,3,5,7} {
        \draw[AccentColor, thick, fill=AccentColor!20] (\x,2.5) rectangle (\x+0.8,3.2);
        \node at (\x+0.4,2.85) {\textcolor{AccentColor}{\textbf{RT}}};
      }

      \foreach \x in {0.5,2.5,4.5,6.5} {
        \draw[RayColor, thick, fill=RayColor!20] (\x,1.5) rectangle (\x+1.5,2.2);
        \node at (\x+0.75,1.85) {\textcolor{RayColor}{\textbf{Tensor}}};
      }

      \draw[ObjectColor, thick] (0.5,0.3) rectangle (8,1.2);
      \node at (4,0.75) {\textcolor{ObjectColor}{\textbf{GDDR6 Memory}}};

      \node[right] at (8.7,4.15) {\textcolor{SecondaryColor}{Shader Cores}};
      \node[right] at (8.7,2.85) {\textcolor{AccentColor}{Ray Tracing}};
      \node[right] at (8.7,1.85) {\textcolor{RayColor}{AI Denoising}};
      \node[right] at (8.7,0.75) {\textcolor{ObjectColor}{High Bandwidth}};
    \end{tikzpicture}
  \end{center}

  \begin{raybox}{RT Core Functions}
    \small
    \textbf{Hardware accelerated:} BVH traversal, ray-triangle intersection, ray-box intersection\\
    \textbf{Result:} Massive parallel ray processing with dedicated silicon
  \end{raybox}
\end{frame}