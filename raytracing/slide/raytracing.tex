\section{From Ray Casting to Ray Tracing}

\begin{frame}{Ray Casting vs Ray Tracing}
  \begin{center}
    \begin{tikzpicture}[scale=0.8]
      % Ray Casting
      \draw[PrimaryColor, fill=PrimaryColor!20] (0,0) rectangle (6,4);
      \node[below] at (3, 4) {\textbf{Ray Casting}};
      \node[eye] (eye1) at (1,1) {\faIcon{eye}};
      \node[sphere] (obj1) at (4,2) {};
      \node[sphere, minimum size=0.5cm] (obj2) at (1,3) {};
      \draw[ray] (eye1) -- (obj1.west);
      \node[below] at (3,0) {\small Primary rays only};

      \begin{scope}[shift={(0,-5)}]
        % Ray Casting
        \draw[PrimaryColor, fill=PrimaryColor!20] (0,0) rectangle (6,4);
        \node[below] at (3, 4) {\textbf{Ray Tracing}};
        \node[eye] (eye1) at (1,1) {\faIcon{eye}};
        \node[sphere] (obj1) at (4,2) {};
        \node[sphere, minimum size=0.5cm] (obj2) at (1,3) {};
        \draw[ray] (eye1) -- (obj1.west);
        \draw[reflectray] (obj1.west) -- (obj2);
        \node[below] at (3,0) {\small Primary rays + Secondary Rays};
      \end{scope}

    \end{tikzpicture}
  \end{center}

\end{frame}

\begin{frame}{Secondary Rays}
  \begin{columns}
    \begin{column}{0.5\textwidth}
      \begin{center}
        \begin{tikzpicture}[scale=0.9]
          \scriptsize
          \node[eye] (eye) at (0,0) {\faIcon{eye}};
          \node[sphere, minimum size=1.2cm] (sphere) at (3,0) {};
          \node[circle, fill=LightColor, minimum size=0.8cm] (light) at (2,3) {\faIcon{lightbulb}};
          \node[sphere, minimum size=0.6cm] (mirror) at (5,3) {};

          \begin{scope}[shift={(3,0)}]
            \coordinate (P) at (135:0.65);
            \draw<1->[ray, very thick] (eye) -- (P) node[midway, above, yshift=0.1cm] {\raycolor{1. Primary Ray}};
            \draw<2->[shadowray] (P) -- (light) node[midway, left] {\textcolor{DarkGray}{2. Shadow Ray}};
            \draw<3->[reflectray] (P) -- (mirror) node[midway, right] {\textcolor{SecondaryColor}{3. Reflection Ray}};
            \draw<4->[reflectray, AccentColor] (P) -- (-10:1.5) node[midway, right, xshift=0.2cm]
            {\textcolor{AccentColor}{4. Refraction Ray}};
          \end{scope}

          \draw[thick, ObjectColor] (-0.5,-1.5) -- (6,-1.5);

          \node[above] at (2,3.5) {\textcolor{LightColor}{Light Source}};
          \node[below] at (3,-0.8) {\textcolor{ObjectColor}{Main Object}};
        \end{tikzpicture}
      \end{center}
    \end{column}
    \begin{column}{0.5\textwidth}
      \begin{conceptbox}{Types of Rays}
        \scriptsize
        \begin{enumerate}
          \item<1-> \textbf{Primary Ray: } From camera to object. Determines closest intersection.
          \item<2-> \textbf{Shadow Ray: } From intersection point to light source. Checks if point is illuminated by light.
          \item<3-> \textbf{Reflection Ray: } From intersection point reflecting off surfaces. For reflections.
          \item<4-> \textbf{Refraction Ray: } From intersection point through transparent surfaces. For refractions.
        \end{enumerate}
      \end{conceptbox}

    \end{column}
  \end{columns}

\end{frame}

\begin{frame}{Recursive Ray Tracing}
  \begin{center}
    \begin{tikzpicture}[scale=0.7]
      % Tree structure showing recursion
      \node[circle, draw, fill=PrimaryColor!30] (root) at (0,4) {Eye};

      % Level 1
      \node[circle, draw, fill=RayColor!30] (obj1) at (-2,2) {Obj1};
      \node[circle, draw, fill=RayColor!30] (obj2) at (2,2) {Obj2};

      % Level 2
      \node[circle, draw, fill=SecondaryColor!30] (mirror1) at (-3,0) {Mirror};
      \node[circle, draw, fill=AccentColor!30] (glass1) at (-1,0) {Glass};
      \node[circle, draw, fill=SecondaryColor!30] (mirror2) at (1,0) {Mirror};
      \node[circle, draw, fill=AccentColor!30] (glass2) at (3,0) {Glass};

      % Level 3
      \node[circle, draw, fill=ObjectColor!30] (final1) at (-3.5,-2) {Obj};
      \node[circle, draw, fill=ObjectColor!30] (final2) at (-2,-2) {Obj};
      \node[circle, draw, fill=ObjectColor!30] (final3) at (2.5,-2) {Obj};

      % Connections
      \draw[arrow] (root) -- (obj1);
      \draw[arrow] (root) -- (obj2);
      \draw[arrow] (obj1) -- (mirror1);
      \draw[arrow] (obj1) -- (glass1);
      \draw[arrow] (obj2) -- (mirror2);
      \draw[arrow] (obj2) -- (glass2);
      \draw[arrow] (mirror1) -- (final1);
      \draw[arrow] (glass1) -- (final2);
      \draw[arrow] (mirror2) -- (final3);

      % Depth labels
      \node[right] at (4,4) {\textcolor{PrimaryColor}{Depth 0}};
      \node[right] at (4,2) {\textcolor{RayColor}{Depth 1}};
      \node[right] at (4,0) {\textcolor{SecondaryColor}{Depth 2}};
      \node[right] at (4,-2) {\textcolor{ObjectColor}{Depth 3}};
    \end{tikzpicture}
  \end{center}

  \begin{conceptbox}{Recursion Control}
    \footnotesize
    \textbf{Termination:} Maximum depth reached OR ray contribution becomes negligible
  \end{conceptbox}
\end{frame}

\subsection{Shadow Rays}
\begin{frame}{Shadows: The Absence of Light}
  \begin{center}
    \scriptsize
    \begin{tikzpicture}[scale=0.8]
      \begin{scope}[shift={(5,0)}]
        \coordinate (P) at (135:0.75);
      \end{scope}
      \node[circle, fill=LightColor, minimum size=0.8cm] (light) at (0,3) {\faIcon{sun}};
      \node[eye] (eye) at (6,2) {\faIcon{eye}};
      \node[above] at (0,3.5) {\textcolor{LightColor}{Light Source}};

      \coordinate (dir) at ($(P)-(2,2)$);

      \node[sphere, minimum size=1.2cm] (target) at (5,0) {};
      \node[below] at (5,-0.8) {\textcolor{ObjectColor}{Target}};

      \draw[ray] (eye) -- (P) node[midway,below,anchor=west] {\raycolor{Primary}};
      \draw<2->[shadowray, dashed] (P) -- (light) node[below, shift={(0.3,-0.5)}] {\textcolor{DarkGray}{Shadow Ray}};
      \fill<4->[pattern=north east lines, pattern color=gray]
      (2,2) -- ($(P)+0.25*(dir)-0.65*(1,1)$) -- ($(P)+0.25*(dir)+0.4*(1,1)$) -- cycle;

      \node[sphere, minimum size=1cm] (occluder) at (2,2) {};
      \node[above] at (2,2.6) {\textcolor{ObjectColor}{Occluder}};

      \node<4->[right] at (3.5,1.5) {\textcolor{AccentColor}{BLOCKED!}};

      \node<4->[right] at (2.5,0.2) {\textcolor{DarkGray}{Shadow}};
    \end{tikzpicture}
  \end{center}

  \begin{raybox}{Shadow Ray Algorithm}
    \footnotesize
    \textbf{For each intersection point:}
    \begin{enumerate}
      \item<2-> Cast ray toward each light source
      \item<3-> Check if ray intersects any object before reaching light
      \item<4-> If \textbf{blocked} → point is in shadow
      \item<5-> If clear → point is illuminated
    \end{enumerate}
  \end{raybox}
\end{frame}

\subsection{Reflections}
\begin{frame}{Reflections}
  \begin{columns}
    \begin{column}{0.7\textwidth}
      \begin{mathbox}{Reflection Law}
        \small
        Given incident ray $\mathbf{I}$ and surface normal $\mathbf{N}$:
        \begin{align*}
          \mathbf{R} & = \mathbf{I} - 2(\mathbf{I} \cdot \mathbf{N})\mathbf{N}
        \end{align*}

        \only<1>{
          \textbf{Physical principle:}\\
          Angle of incidence = Angle of reflection
        }
      \end{mathbox}
    \end{column}
    \begin{column}{0.3\textwidth}
      \begin{tikzpicture}[scale=0.8]
        \draw[thick, ObjectColor] (-0.5,0) -- (3.5,0);
        \node[below] at (1.5,-0.3) {\objectcolor{Surface}};
        \draw[->, thick, AccentColor] (1.5,0) -- (1.5,2) node[above] {$\mathbf{N}$};
        \draw[ray] (0,1.5) -- (1.5,0) node[midway, left] {$\mathbf{I}$};

        \draw[reflectray] (1.5,0) -- (3,1.5) node[midway,right] {$\mathbf{R}$};

        \node[above] at (1.2,0.3) {$\theta$};
        \node[above] at (1.8,0.3) {$\theta$};

        \fill[AccentColor] (1.5,0) circle (2pt);
      \end{tikzpicture}
    \end{column}
  \end{columns}
  \only<2->{
    \begin{conceptbox}{Raytraced Reflection}
      \begin{enumerate}
          \small
        \item From the intersection point, cast a secondary ray in the direction of the reflection vector $\mathbf{R}$.
        \item Calculate the colour at the intersection point of the secondary ray with the scene $\mathbf{I}_{\text{reflected}}$.
        \item Combine the local colour $\mathbf{I}_{\text{local}}$ with the reflected colour:
          \begin{align*}
            \mathbf{I} &= \mathbf{I}_{\text{local}} + \mathbf{k}_r \odot \mathbf{I}_{\text{reflected}}
          \end{align*}
          Where $\mathbf{k}_r$ is the reflection coefficient (material property).
      \end{enumerate}
    \end{conceptbox}
  }
\end{frame}

\subsection{Implementation Challenges}

\begin{frame}{The Floating Point Precision Problem}
  \begin{center}
    \begin{tikzpicture}[scale=0.9]
      % Surface
      \draw[thick, ObjectColor] (-1,0) -- (4,0);
      \node[below] at (2,-0.5) {\objectcolor{Surface}};

      \fill[SecondaryColor] (2,0) circle (3pt);

      \draw[reflectray] (2, 0) -- (1,1) node[above] {\scriptsize \textcolor{SecondaryColor}{True intersection}};
      \draw[shadowray] (2, 0.1) arc [start angle=180, end angle=-140, radius=0.3] node[midway, right]
      {\textcolor{DarkGray}{\scriptsize Self-intersection}};

      % Problems
      \node[right] at (4.5,0.5) {
        \begin{minipage}{4cm}
          \scriptsize
          \textcolor{AccentColor}{\textbf{Problems:}}
          \begin{itemize}
            \item Self-intersection
            \item Incorrect shadows
            \item Ray escaping
          \end{itemize}
        \end{minipage}
      };
    \end{tikzpicture}
  \end{center}

  \begin{conceptbox}{The Evil Epsilon}
    \textbf{Solution:} Add small offset $\varepsilon$ when starting secondary rays from surfaces or set $t_{\min} = \varepsilon$ \\
    \textbf{Challenge:} Too small → still problems; Too large → artifacts
  \end{conceptbox}
\end{frame}