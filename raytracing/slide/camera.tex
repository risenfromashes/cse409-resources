\section{Cameras}

\begin{frame}{The Pinhole Camera Model}
    \begin{columns}
        \begin{column}{0.5\textwidth}
            \begin{mathbox}{Pinhole Camera}
                \footnotesize
                The simplest camera. A box with a pinhole. Light entering the pinhole creates an image on the other side of the box.
                \only<2->{
                    \begin{itemize}
                        \item Point aperture
                        \item Perfect focus everywhere
                        \only<3->{
                            \item Inverted Image
                        }
                    \end{itemize}
                }

                \only<4->{
                    \textbf{Ray Generation:}
                    \begin{align*}
                        \mathbf{R_o} & = \mathbf{hole}                  \\
                        \mathbf{R_d} & = \frac{\mathbf{hole} - \mathbf{pixel}}{|\mathbf{hole} - \mathbf{pixel}|}
                    \end{align*}
                }
            \end{mathbox}
        \end{column}
        \begin{column}{0.5\textwidth}
            \begin{tikzpicture}[scale=0.35]
                \tikzset{
                    screen/.style={fill=blue!10, draw=blue!50, opacity=0.8},
                    pixel/.style={fill=AccentColor!60, thick},
                    primary ray/.style={->, very thick, red!90},
                    object/.style={fill=orange!60, draw=orange!80, circle, minimum size=12pt}
                }
                \coordinate (hole) at (3, 0, 0);

                \node[below] at (-3,-4,0) {\scriptsize Image Plane};

                \begin{scope}[canvas is zy plane at x=-3]
                    % screen
                    \fill[screen] (-3,-3) rectangle (3,3);
                    \foreach \x in {-3,-2,...,3} {
                            \draw[gray!60, thin] (\x,-3) -- (\x,3);
                        }
                    \foreach \z in {-3,-2,...,3} {
                            \draw[gray!60, thin] (-3,\z) -- (3,\z);
                        }

                    \foreach \x in {-2.5,-1.5,...,2.5} {
                        \foreach \z in {-2.5,-1.5,...,2.5} {
                                \coordinate (dir) at ($(hole)-(\x,\z)$);
                                \draw[ray, thin, opacity=0.3] (\x,\z) -- (hole);
                                \draw[red!40] (\x,\z) circle (1pt);
                            }
                        }
                    \only<3->{
                        \draw[AccentColor, fill=AccentColor!30, opacity=0.7] (2.5, 1.5) -- (-2.5, 2.5) -- (0.5, -2.5) -- cycle;
                    }
                    \draw[ray, thick, opacity=0.9] (-2.5,2.5) -- (hole);
                    \draw[ray, thick, opacity=0.9] (0.5,-2.5) -- (hole);
                \end{scope}
                \draw[-, PrimaryColor] (-3,-3,-3) -- (3, -3, -3);
                \draw[-, PrimaryColor] (-3,-3,3) -- (3, -3, 3);
                \draw[-, PrimaryColor] (-3,3,-3) -- (3, 3, -3);
                \draw[-, PrimaryColor] (-3,3,3) -- (3, 3, 3);
                \begin{scope}[canvas is zy plane at x=3]
                    \draw[black, fill=gray!60] (-3,-3) rectangle (3,3);
                    \draw[PrimaryColor,fill=white] (0,0) circle (3pt) node[above] {\tiny Pinhole};
                \end{scope}
                \begin{scope}[canvas is zy plane at x=-3]
                    \foreach \x in {-2.5,-1.5,...,2.5} {
                        \foreach \z in {-2.5,-1.5,...,2.5} {
                                \coordinate (dir) at ($(hole)-(\x,\z)$);
                                \draw[ray, thin, opacity=0.3] (hole) -- ($(hole)+(dir)$);
                            }
                        }
                    
                    \coordinate (dir1) at ($(hole)-(-2.5,2.5)$);
                    \coordinate (dir2) at ($(hole)-(0.5,-2.5)$);
                    \draw[ray, thick, opacity=0.9] (hole) -- ($(hole)+(dir1)$);
                    \draw[ray, thick, opacity=0.9] (hole) -- ($(hole)+(dir2)$);
                \end{scope}

                \draw[AccentColor, fill=AccentColor!30] (9, -2, 3) -- (8, 1.5, -3) -- (13, -3, 4) -- cycle;
                \node[below] at (10,-4.5,0) {\scriptsize \textcolor{AccentColor}{Object}};
            \end{tikzpicture}
        \end{column}
    \end{columns}
    \only<5->{
        \begin{conceptbox}{Physical Reality}
            \footnotesize
            Real pinhole cameras exist! They create sharp images but require very long exposure times due to tiny aperture.
        \end{conceptbox}
    }
\end{frame}

\begin{frame}{Simplified Pinhole Camera}
    \begin{columns}
        \begin{column}{0.5\textwidth}
            \begin{mathbox}{Simplification}
                \footnotesize
                Place image plane in front! \\ 
                Equivalent to pinhole camera.

                \only<2->{   
                    \begin{itemize}
                        \item Physically unrealizable
                        \only<3->{
                            \item Non-inverted image
                        }
                    \end{itemize}
                }
                \only<4->{
                    \textbf{Ray Generation:}
                        \begin{align*}
                            \mathbf{R_o} & = \mathbf{eye}                  \\
                            \mathbf{R_d} & = \frac{\mathbf{pixel} - \mathbf{eye}}{|\mathbf{pixel} - \mathbf{eye}|}
                        \end{align*}
                }
            \end{mathbox}
        \end{column}
        \begin{column}{0.5\textwidth}
            \centering
            \begin{tikzpicture}[scale=0.5]
                \tikzset{
                    screen/.style={fill=blue!10, draw=blue!50, opacity=0.8},
                    pixel/.style={fill=AccentColor!60, thick},
                    primary ray/.style={->, very thick, red!90},
                    object/.style={fill=orange!60, draw=orange!80, circle, minimum size=12pt}
                }
                \node[eye] (eye) at (0,0,0) {\faIcon{eye}};
                \draw[AccentColor, fill=AccentColor!30] (7.5, -1, 3) -- (6.25, 2.25, -3) -- (11.5, -2, 4) -- cycle;
                \node[below] at (8,-3,0) {\scriptsize \textcolor{AccentColor}{Object}};
                \node[below] at (0,-0.8,0) {\scriptsize Eye};
                \node[above] at (3.5,4,0) {\scriptsize Screen};
                \begin{scope}[canvas is zy plane at x=3.5]
                    \fill[screen] (-3,-3) rectangle (3,3);
                    \foreach \x in {-3,-2,...,3} {
                            \draw[gray!60, thin] (\x,-3) -- (\x,3);
                        }
                    \foreach \z in {-3,-2,...,3} {
                            \draw[gray!60, thin] (-3,\z) -- (3,\z);
                        }
                    \foreach \x in {-2.5,-1.5,...,2.5} {
                            \foreach \z in {-2.5,-1.5,...,2.5} {
                                    \fill[red!40] (\x, \z) circle (2pt);
                                    \coordinate (dir) at ($(\x,\z)-(eye.center)$);
                                    \draw[ray, thin, opacity=0.3] (eye.center) -- (\x,\z);
                                    \draw[ray, thin, opacity=0.3] (\x,\z) -- ($(\x,\z)+(dir)$);
                                }
                        }
                    \only<3->{
                        \draw[AccentColor, fill=AccentColor!30, opacity=0.7] (-0.5,1.5) -- (-1.5,-1.5) -- (1.5,-1.5) -- cycle;
                    }
                    \coordinate (p1) at (-0.5,1.5);
                    \coordinate (dir1) at ($(p1)-(eye.center)$);
                    \draw[ray, thick, opacity=0.9] (eye.center) -- ($(p1)+(dir1)$);
                    \coordinate (p2) at (-0.5,-1.5);
                    \coordinate (dir2) at ($(p2)-(eye.center)$);
                    \draw[ray, thick, opacity=0.9] (eye.center) -- ($(p2)+(dir2)$);
                \end{scope}
            \end{tikzpicture}
        \end{column}
    \end{columns}
    \vspace{-0.5cm}
    \only<5->{
        \begin{conceptbox}{Advantage}
            \footnotesize
            \textbf{Upright image}, simpler ray generation, equivalent to real pinhole!
        \end{conceptbox}
    }
\end{frame}

\endinput

\begin{frame}{Orthographic Camera}
    \begin{columns}
        \begin{column}{0.5\textwidth}
            \begin{mathbox}{Orthographic Projection}
                \textbf{Key Properties:}
                \begin{itemize}
                    \item No perspective distortion
                    \item Parallel projection rays
                    \item Objects same size regardless of distance
                    \item Infinite focal length
                \end{itemize}

                \textbf{Ray Generation:}
                \begin{align}
                    \mathbf{R_o} & = \mathbf{pixel}                \\
                    \mathbf{R_d} & = \mathbf{w} \text{ (constant)}
                \end{align}
            \end{mathbox}
        \end{column}
        \begin{column}{0.5\textwidth}
            \begin{tikzpicture}[scale=0.8]
                % Define 3D styles for orthographic
                \tikzset{
                    ortho plane/.style={fill=AccentColor!10, draw=AccentColor!50, opacity=0.8},
                    ortho pixel/.style={fill=AccentColor!60, thick},
                    ortho ray/.style={->, thick, blue!70},
                    ortho object/.style={fill=orange!60, draw=orange!80}
                }

                % Image plane in 3D (using canvas is plane)
                \begin{scope}[canvas is zy plane at x=0]
                    % Background image plane
                    \fill[ortho plane] (-2,-1.5) rectangle (2,1.5);
                    \node[left] at (-2.5,0) {\textcolor{AccentColor}{Image Plane}};

                    % Pixel grid
                    \foreach \y in {-1.5,-1,-0.5,0,0.5,1,1.5} {
                            \draw[thin, gray] (-2,\y) -- (2,\y);
                        }
                    \foreach \z in {-2,-1.5,-1,-0.5,0,0.5,1,1.5,2} {
                            \draw[thin, gray] (\z,-1.5) -- (\z,1.5);
                        }

                    % Sample pixels
                    \fill[ortho pixel] (0.5,0.5) rectangle (1,1);
                    \fill[ortho pixel] (-1,-1) rectangle (-0.5,-0.5);
                    \fill[ortho pixel] (0,-0.5) rectangle (0.5,0);
                \end{scope}

                % Objects in 3D scene
                \node[ortho object, circle, minimum size=1cm] (sphere1) at (5,1,1) {};
                \node[ortho object, circle, minimum size=0.8cm] (sphere2) at (5,0,-1) {};
                \node[ortho object, diamond, minimum size=0.7cm] (diamond) at (5,-1,0.5) {};

                % Parallel rays from pixels (orthographic property)
                \draw[ortho ray] (0.75,0.75,0) -- (5,1,1);
                \draw[ortho ray] (-0.75,-0.75,0) -- (5,0,-1);
                \draw[ortho ray] (0.25,-0.25,0) -- (5,-1,0.5);

                % All rays parallel to w direction
                \draw[ortho ray, dashed] (0.75,0.75,0) -- (1.5,0.75,0);
                \draw[ortho ray, dashed] (-0.75,-0.75,0) -- (1.5,-0.75,0);
                \draw[ortho ray, dashed] (0.25,-0.25,0) -- (1.5,-0.25,0);

                % 3D coordinate system
                \draw[<->, thick, PrimaryColor] (0,0,0) -- (1.5,0,0);
                \draw[<->, thick, PrimaryColor] (0,0,0) -- (0,1.5,0);
                \draw[<->, thick, PrimaryColor] (0,0,0) -- (0,0,1.5);
                \node[right] at (1.5,0,0) {\textcolor{PrimaryColor}{$\mathbf{u}$}};
                \node[above] at (0,1.5,0) {\textcolor{PrimaryColor}{$\mathbf{v}$}};
                \node[below] at (0,0,1.5) {\textcolor{PrimaryColor}{$\mathbf{w}$}};

                % Label showing parallel rays
                \node[above] at (3,2,0) {\textcolor{blue!70}{\small Parallel rays}};
            \end{tikzpicture}
        \end{column}
    \end{columns}

    \begin{conceptbox}{Applications}
        \textbf{Technical drawings}, \textbf{CAD software}, \textbf{2D games}, architectural visualization
    \end{conceptbox}
\end{frame}

\begin{frame}{Perspective vs Orthographic}
    \begin{center}
        \begin{tikzpicture}[scale=0.8]
            % Perspective
            \node[rectangle, draw, minimum width=3cm, minimum height=2cm, fill=PrimaryColor!20] at (-3,2) {Perspective};
            \node[eye] (eye1) at (-4,1) {};
            \node[triangle] (near1) at (-2,1.2) {};
            \node[triangle] (far1) at (-1,1) {};
            \draw[ray] (eye1) -- (-2,1.5) -- (-0.5,1.2);
            \draw[ray] (eye1) -- (-2,0.5) -- (-0.5,0.8);
            \node[below] at (-3,0.2) {\small Converging rays};
            \node[below] at (-3,-0.1) {\small Size $\propto$ 1/distance};

            % vs
            \node at (0,1) {\huge \textcolor{AccentColor}{vs}};

            % Orthographic  
            \node[rectangle, draw, minimum width=3cm, minimum height=2cm, fill=SecondaryColor!20] at (3,2) {Orthographic};
            \draw[ray] (2,1.5) -- (4.5,1.5);
            \draw[ray] (2,0.5) -- (4.5,0.5);
            \node[triangle] (near2) at (3,1.2) {};
            \node[triangle] (far2) at (4,1.2) {};
            \node[below] at (3,0.2) {\small Parallel rays};
            \node[below] at (3,-0.1) {\small Constant size};
        \end{tikzpicture}
    \end{center}

    \vspace{0.5cm}
    \begin{columns}
        \begin{column}{0.5\textwidth}
            \begin{raybox}{When to use Perspective}
                \begin{itemize}
                    \item Natural/realistic scenes
                    \item Human vision simulation
                    \item Games and films
                    \item Depth perception important
                \end{itemize}
            \end{raybox}
        \end{column}
        \begin{column}{0.5\textwidth}
            \begin{raybox}{When to use Orthographic}
                \begin{itemize}
                    \item Technical illustrations
                    \item CAD/Engineering
                    \item UI elements overlay
                    \item Precise measurements
                \end{itemize}
            \end{raybox}
        \end{column}
    \end{columns}
\end{frame}

\begin{frame}{Thin Lens Camera: Fundamentals}
    \begin{columns}
        \begin{column}{0.5\textwidth}
            \begin{mathbox}{Gaussian Lens Equation}
                \textbf{Fundamental relationship:}
                \begin{align}
                    \frac{1}{f} = \frac{1}{z} + \frac{1}{z'}
                \end{align}

                \textbf{Where:}
                \begin{itemize}
                    \item $f$ = focal length of lens
                    \item $z$ = object distance from lens
                    \item $z'$ = image distance from lens
                \end{itemize}

                \vspace{0.3cm}
                \textbf{Key Properties:}
                \begin{itemize}
                    \item Objects at focal plane are in perfect focus
                    \item Other distances create blur (circle of confusion)
                    \item Aperture size controls depth of field
                \end{itemize}
            \end{mathbox}
        \end{column}
        \begin{column}{0.5\textwidth}
            \begin{tikzpicture}[scale=0.7]
                % Define styles for thin lens
                % Optical axis
                \draw[optical axis] (-4,0) -- (6,0);
                
                % Thin lens at origin
                \lens{(0,0)}{2cm}{5cm};
                \node[below] at (0,-2.8) {\textcolor{PrimaryColor}{Thin Lens}};
                
                % Object at distance z
                \draw[thick, AccentColor] (-3,-1.5) -- (-3,1.5);
                \node[left] at (-3,0) {\objectcolor{Object}};
                \draw[<->, thin] (-3,-3) -- (0,-3);
                \node[below] at (-1.5,-3) {$z$};
                
                % Image at distance z'
                \draw[thick, SecondaryColor] (4,-1) -- (4,1);
                \node[right] at (4,0) {\textcolor{SecondaryColor}{Image}};
                \draw[<->, thin] (0,-3.5) -- (4,-3.5);
                \node[below] at (2,-3.5) {$z'$};
                
                % Focal points
                \fill[AccentColor] (-2,0) circle (2pt);
                \fill[AccentColor] (2,0) circle (2pt);
                \node[below] at (-2,-0.3) {\textcolor{AccentColor}{$F$}};
                \node[below] at (2,-0.3) {\textcolor{AccentColor}{$F'$}};
                
                % Focal length markers
                \draw[<->, thin] (-2,2.8) -- (0,2.8);
                \node[above] at (-1,2.8) {$f$};
                \draw[<->, thin] (0,3.2) -- (2,3.2);
                \node[above] at (1,3.2) {$f$};
                
                % Principal rays
                \draw[object ray] (-3,1.5) -- (0,1.5) -- (4,-1);
                \draw[object ray] (-3,1.5) -- (0,0.75) -- (4,-1);
                \draw[object ray] (-3,0) -- (0,0) -- (4,0);
                
                % Ray labels
                \node[above] at (-1.5,1.5) {\objectcolor{\tiny Parallel ray}};
                \node[above] at (2,0.7) {\textcolor{SecondaryColor}{\tiny Through focus}};
            \end{tikzpicture}
        \end{column}
    \end{columns}

    \begin{conceptbox}{Real Camera Behavior}
        Unlike pinhole cameras, thin lens cameras exhibit \textbf{depth of field} - only objects at the focal distance are perfectly sharp.
    \end{conceptbox}
\end{frame}

\begin{frame}{Depth of Field and Circle of Confusion}
    \begin{columns}
        \begin{column}{0.5\textwidth}
            \begin{mathbox}{Circle of Confusion}
                \textbf{For objects not at focal distance:}
                \begin{align}
                    c = \frac{A}{z'} \left| z'_{focus} - z' \right|
                \end{align}

                \textbf{Where:}
                \begin{itemize}
                    \item $c$ = circle of confusion diameter
                    \item $A$ = aperture diameter
                    \item $z'$ = image distance for object
                    \item $z'_{focus}$ = image distance for focus
                \end{itemize}

                \vspace{0.3cm}
                \textbf{Depth of Field:}
                \begin{itemize}
                    \item Region where $c < $ acceptable limit
                    \item Larger aperture $\rightarrow$ shallower DOF
                    \item Smaller aperture $\rightarrow$ deeper DOF
                \end{itemize}
            \end{mathbox}
        \end{column}
        \begin{column}{0.5\textwidth}
            \begin{tikzpicture}[scale=0.6]
                % Three object planes at different distances
                \draw[thick, AccentColor] (-4,2) -- (-4,-2);
                \draw[thick, AccentColor] (-2.5,1.5) -- (-2.5,-1.5);
                \draw[thick, AccentColor] (-1,1) -- (-1,-1);
                
                \node[left] at (-4,0) {\objectcolor{\tiny Far}};
                \node[left] at (-2.5,0) {\objectcolor{\tiny Focus}};
                \node[left] at (-1,0) {\objectcolor{\tiny Near}};
                
                % Lens
                \lens{(0,0)}{2cm}{6cm};
                \node[below] at (0,-3.3) {\textcolor{PrimaryColor}{Lens}};
                
                % Image plane
                \draw[thick, gray] (3,-3) -- (3,3);
                \node[right] at (3,3.3) {\textcolor{gray}{Image Plane}};
                
                % Rays from far object (out of focus)
                \draw[object ray, opacity=0.7] (-4,1) -- (0,0.3) -- (3,-0.8);
                \draw[object ray, opacity=0.7] (-4,-1) -- (0,-0.3) -- (3,0.8);
                \fill[red, opacity=0.6] (3,0) circle (8pt);
                \node[right] at (3.3,0.5) {\textcolor{red}{\tiny Large blur}};
                
                % Rays from focus object (sharp)
                \draw[object ray] (-2.5,0.75) -- (0,0.3) -- (3,-0.6);
                \draw[object ray] (-2.5,-0.75) -- (0,-0.3) -- (3,0.6);
                \fill[green!80!black] (3,0) circle (2pt);
                \node[right] at (3.3,-0.5) {\textcolor{green!80!black}{\tiny Sharp}};
                
                % Rays from near object (out of focus)
                \draw[object ray, opacity=0.7] (-1,0.5) -- (0,0.25) -- (3,-0.4);
                \draw[object ray, opacity=0.7] (-1,-0.5) -- (0,-0.25) -- (3,0.4);
                \fill[orange, opacity=0.6] (3,0) circle (5pt);
                \node[right] at (3.3,-1) {\textcolor{orange}{\tiny Medium blur}};
                
                % Aperture indication
                \draw[thick, blue] (-0.2,-1.5) -- (0.2,-1.5);
                \draw[thick, blue] (-0.2,1.5) -- (0.2,1.5);
                \node[right] at (0.3,0) {\textcolor{blue}{\tiny Aperture}};
            \end{tikzpicture}
        \end{column}
    \end{columns}

    \begin{conceptbox}{Artistic Control}
        \textbf{Shallow DOF}: Subject isolation, portrait photography\\
        \textbf{Deep DOF}: Landscape photography, technical documentation
    \end{conceptbox}
\end{frame}

\begin{frame}{Aperture Effects on Depth of Field}
    \begin{center}
        \begin{tikzpicture}[scale=0.8]
            % Large aperture scene
            \node[above] at (-3,3) {\textbf{Large Aperture (f/1.4)}};
            
            % Lens with large aperture
            \lens{(-3,0)}{2cm}{4cm};
            \draw[thick, blue] (-3.1,-1.5) -- (-2.9,-1.5);
            \draw[thick, blue] (-3.1,1.5) -- (-2.9,1.5);
            \node[left] at (-3.3,0) {\textcolor{blue}{\tiny Wide}};
            
            % Objects at different distances
            \draw[thick, AccentColor] (-5,0.5) -- (-5,-0.5);
            \draw[thick, AccentColor] (-4,1) -- (-4,-1);
            \draw[thick, AccentColor] (-3.5,1.5) -- (-3.5,-1.5);
            
            % Image plane
            \draw[thick, gray] (-1,-2) -- (-1,2);
            
            % Blur circles (large aperture = shallow DOF)
            \fill[red, opacity=0.4] (-1,0) circle (10pt);
            \fill[green!80!black] (-1,0) circle (3pt);
            \fill[orange, opacity=0.4] (-1,0) circle (7pt);
            
            \node[below] at (-3,-2.5) {\textcolor{red}{\tiny Shallow DOF}};
            
            % vs divider
            \node at (0,0) {\huge \textcolor{AccentColor}{vs}};
            
            % Small aperture scene
            \node[above] at (3,3) {\textbf{Small Aperture (f/11)}};
            
            % Lens with small aperture
            \lens{(3,0)}{2cm}{4cm};
            \draw[thick, blue] (2.95,-0.5) -- (3.05,-0.5);
            \draw[thick, blue] (2.95,0.5) -- (3.05,0.5);
            \node[right] at (3.3,0) {\textcolor{blue}{\tiny Narrow}};
            
            % Objects at different distances
            \draw[thick, AccentColor] (1,0.5) -- (1,-0.5);
            \draw[thick, AccentColor] (2,1) -- (2,-1);
            \draw[thick, AccentColor] (2.5,1.5) -- (2.5,-1.5);
            
            % Image plane
            \draw[thick, gray] (5,-2) -- (5,2);
            
            % Blur circles (small aperture = deep DOF)
            \fill[green!80!black] (5,0) circle (3pt);
            \fill[green!80!black] (5,0) circle (3pt);
            \fill[green!80!black] (5,0) circle (3pt);
            
            \node[below] at (3,-2.5) {\textcolor{green!80!black}{\tiny Deep DOF}};
        \end{tikzpicture}
    \end{center}

    \vspace{0.5cm}
    \begin{columns}
        \begin{column}{0.5\textwidth}
            \begin{raybox}{Large Aperture}
                \begin{itemize}
                    \item More light gathering
                    \item Faster shutter speeds
                    \item Shallow depth of field
                    \item Background blur (bokeh)
                \end{itemize}
            \end{raybox}
        \end{column}
        \begin{column}{0.5\textwidth}
            \begin{raybox}{Small Aperture}
                \begin{itemize}
                    \item Less light gathering
                    \item Slower shutter speeds
                    \item Deep depth of field
                    \item Everything in focus
                \end{itemize}
            \end{raybox}
        \end{column}
    \end{columns}
\end{frame}

\begin{frame}{Thin Lens Ray Generation}
    \begin{columns}
        \begin{column}{0.5\textwidth}
            \begin{mathbox}{Ray Sampling Process}
                \textbf{1. Sample pixel position} $(x, y)$
                
                \textbf{2. Sample lens position:}
                \begin{align}
                    (u, v) &\sim \text{Uniform disk}\\
                    \mathbf{p}_{lens} &= (u \cdot r, v \cdot r, 0)
                \end{align}
                
                \textbf{3. Compute focal point:}
                \begin{align}
                    \mathbf{p}_{focus} &= \mathbf{p}_{pixel} \cdot \frac{f_{dist}}{n}
                \end{align}
                
                \textbf{4. Ray from lens to focal point:}
                \begin{align}
                    \mathbf{R_o} &= \mathbf{p}_{lens}\\
                    \mathbf{R_d} &= \mathbf{p}_{focus} - \mathbf{p}_{lens}
                \end{align}
                
                where $r$ = aperture radius, $f_{dist}$ = focus distance
            \end{mathbox}
        \end{column}
        \begin{column}{0.5\textwidth}
            \begin{tikzpicture}[scale=0.7]
                % Define styles for ray generation
                \tikzset{
                    lens disk/.style={fill=PrimaryColor!30, draw=PrimaryColor!80},
                    sample point/.style={fill=red!80, circle, inner sep=1pt},
                    focal point/.style={fill=AccentColor!80, circle, inner sep=1.5pt},
                    sample ray/.style={->, thick, red!70},
                    focal plane/.style={dashed, AccentColor}
                }

                % Image plane
                \draw[thick, gray] (-3,-2) -- (-3,2);
                \node[left] at (-3,2.3) {\textcolor{gray}{Image Plane}};
                
                % Pixel grid
                \foreach \y in {-1.5,-1,-0.5,0,0.5,1,1.5} {
                        \draw[thin, gray!50] (-3.2,\y) -- (-2.8,\y);
                    }
                
                % Sample pixel
                \fill[AccentColor!60] (-3,1) rectangle (-2.8,1.2);
                \fill[red!70] (-2.9,1.1) circle (1.5pt);
                \node[left] at (-3.3,1.1) {\textcolor{red!70}{\tiny Sample}};
                
                % Lens disk
                \fill[lens disk] (0,0) circle (1);
                \draw[thick, PrimaryColor] (0,-1) -- (0,1);
                \node[below] at (0,-1.3) {\textcolor{PrimaryColor}{Lens}};
                
                % Sample points on lens
                \node[sample point] at (0.3,0.4) {};
                \node[sample point] at (-0.5,-0.2) {};
                \node[sample point] at (0.1,-0.7) {};
                \node[sample point] at (-0.3,0.6) {};
                
                % Focal plane
                \draw[focal plane] (3,-2) -- (3,2);
                \node[right] at (3,2.3) {\textcolor{AccentColor}{Focal Plane}};
                
                % Focal point (where pixel ray hits focal plane)
                \fill[focal point] (3,0.55) circle (2pt);
                \node[right] at (3.2,0.55) {\textcolor{AccentColor}{\tiny Focus}};
                
                % Sample rays from lens points to focal point
                \draw[sample ray] (0.3,0.4) -- (3,0.55);
                \draw[sample ray] (-0.5,-0.2) -- (3,0.55);
                \draw[sample ray] (0.1,-0.7) -- (3,0.55);
                \draw[sample ray] (-0.3,0.6) -- (3,0.55);
                
                % Central ray (through lens center)
                \draw[dashed, gray] (-2.9,1.1) -- (0,0) -- (3,0.55);
                
                % Distance markers
                \draw[<->, thin] (-3,-2.7) -- (0,-2.7);
                \node[below] at (-1.5,-2.7) {$n$};
                \draw[<->, thin] (0,-2.7) -- (3,-2.7);
                \node[below] at (1.5,-2.7) {$f_{dist}$};
            \end{tikzpicture}
        \end{column}
    \end{columns}

    \begin{conceptbox}{Monte Carlo Integration}
        Multiple samples per pixel with different lens positions create realistic \textbf{depth of field blur}
    \end{conceptbox}
\end{frame}



\begin{frame}{Other Camera Types}
    \begin{columns}
        \begin{column}{0.5\textwidth}
            \begin{conceptbox}{Fish-eye Camera}
                \begin{itemize}
                    \item Very wide field of view (>180°)
                    \item Non-linear distortion
                    \item Curved ray paths
                    \item Surveillance, VR applications
                \end{itemize}
            \end{conceptbox}

            \vspace{0.3cm}
            \begin{conceptbox}{Environment Camera}
                \begin{itemize}
                    \item 360° panoramic view
                    \item Spherical or cylindrical
                    \item HDRI environment maps
                    \item VR content creation
                \end{itemize}
            \end{conceptbox}
        \end{column}
        \begin{column}{0.5\textwidth}
            \begin{tikzpicture}[scale=0.8]
                % Fish-eye illustration
                \node[above] at (0,3) {\textbf{Fish-eye FOV}};
                \node[eye] (fisheye) at (0,2) {\faIcon{eye}};
                \draw[ray, bend left=30] (fisheye) to (2,3);
                \draw[ray] (fisheye) -- (2,2);
                \draw[ray, bend right=30] (fisheye) to (2,1);
                \draw[ray, bend left=60] (fisheye) to (1,3.5);
                \draw[ray, bend right=60] (fisheye) to (1,0.5);
                \node[right] at (2.2,2) {\tiny Wide FOV};

                % Motion blur illustration  
                \node[above] at (0,0.5) {\textbf{Motion Blur}};
                \draw[thick, gray] (-1,-1) -- (1,-1);
                \node[left] at (-1,-1) {\textcolor{gray}{\tiny Image plane}};
                \draw[ray, opacity=0.3] (-0.5,-0.5) -- (0.5,-1);
                \draw[ray, opacity=0.5] (-0.3,-0.5) -- (0.3,-1);
                \draw[ray, opacity=0.7] (-0.1,-0.5) -- (0.1,-1);
                \draw[ray] (0.1,-0.5) -- (-0.1,-1);
                \node[right] at (1.2,-1) {\tiny Multiple samples};
            \end{tikzpicture}
        \end{column}
    \end{columns}

    \begin{columns}
        \begin{column}{0.5\textwidth}
            \begin{conceptbox}{Environment Camera}
                \begin{itemize}
                    \item 360° panoramic view
                    \item Spherical or cylindrical
                    \item HDRI environment maps
                    \item VR content creation
                \end{itemize}
            \end{conceptbox}
        \end{column}
        \begin{column}{0.5\textwidth}
            \begin{conceptbox}{Motion Blur Camera}
                \begin{itemize}
                    \item Simulates camera/object motion
                    \item Multiple time samples
                    \item Temporal anti-aliasing
                    \item Dynamic scene rendering
                \end{itemize}
            \end{conceptbox}
        \end{column}
    \end{columns}
\end{frame}




\begin{frame}{Camera Representation}
    \begin{center}
        \begin{tikzpicture}[scale=0.7]
            % Define 3D styles
            \tikzset{
                camera/.style={fill=PrimaryColor!60, draw=PrimaryColor!80, rectangle, minimum size=8pt},
                image plane/.style={fill=AccentColor!10, draw=AccentColor!50, opacity=0.8},
                pixel/.style={fill=AccentColor!60, thick},
                primary ray/.style={->, very thick, red!90},
                object/.style={fill=orange!60, draw=orange!80, circle, minimum size=12pt}
            }

            % 3D coordinate system
            \draw[<->, thick, PrimaryColor] (0,0,0) -- (1.5,0,0);
            \draw[<->, thick, PrimaryColor] (0,0,0) -- (0,1.5,0);
            \draw[<->, thick, PrimaryColor] (0,0,0) -- (0,0,2);
            \node[right] at (1.5,0,0) {\textcolor{PrimaryColor}{$\mathbf{u}$}};
            \node[above] at (0,1.5,0) {\textcolor{PrimaryColor}{$\mathbf{v}$}};
            \node[below] at (0,0,2) {\textcolor{PrimaryColor}{$\mathbf{w}$}};

            % Camera/Eye position in 3D space
            \node[eye] (camera) at (0,0,0) {\faIcon{camera}};
            % \node[below] at (0.2,-0.5,0) {\textcolor{PrimaryColor}{Camera}};


            \draw[thick, gray] (camera) -- (4,2,2);
            \draw[thick, gray] (camera) -- (4,-2,2);
            \draw[thick, gray] (camera) -- (4,2,-2);
            \draw[thick, gray] (camera) -- (4,-2,-2);

            % Objects in 3D scene
            \node[sphere, minimum size=1cm] (sphere1) at (8,1,2) {};
            \node[sphere, minimum size=0.8cm] (sphere2) at (7,-1,-1) {};
            \node[triangle, minimum size=1cm] (tri) at (9,0.5,0) {};

            \draw[<->, thick, gray] (0, 2.5, 0) -- (4, 2.5, 0) node[midway, above] {\textcolor{gray}{$n$}};

            \node[below] at (0, -0.5, 0) {\textcolor{PrimaryColor!80}{$\mathbf{e}$}};

            \begin{scope}[
                    ->,
                    plane x={(0.8944,-0.4472, 0)},
                    plane y={(0,0,-1)},
                    canvas is plane,
                ]
                \fill[PrimaryColor!30, opacity=0.3] 
                    (0,0) 
                    -- (2,0)
                    arc[start angle=0, end angle=25, radius=2]
                    -- (0,0)
                    -- (2,0)
                    arc[start angle=0, end angle=-25, radius=2]
                -- cycle;
                \draw[->, thick, PrimaryColor] (2,0) arc[start angle=0, end angle=25, radius=2];
                \draw[->, thick, PrimaryColor] (2,0) arc[start angle=0, end angle=-25, radius=2];
                \node[below] at (2.2,0.3) {\textcolor{PrimaryColor}{$fov$}};
            \end{scope}


            % Image plane in 3D (using canvas is plane)
            \begin{scope}[canvas is zy plane at x=4]
                % Background image plane
                \fill[image plane] (-2,-2) rectangle (2,2);
                \node[right] at (2.2,-2.8) {\textcolor{AccentColor}{Image Plane}};

                % Pixel grid
                \foreach \y in {-2, -1.5,-1,-0.5,0,0.5,1,1.5, 2} {
                        \draw[thin, gray] (-2,\y) -- (2,\y);
                    }
                \foreach \z in {-2, -1.5,-1,-0.5,0,0.5,1,1.5, 2} {
                        \draw[thin, gray] (\z,-2) -- (\z,2);
                    }

                % Sample pixels
                \fill[pixel] (0.5,1) rectangle (1,1.5);
                \fill[pixel] (0,-1.5) rectangle (0.5,-1);

                % Pixel centers
                \fill[red!70] (0.75, 1.25) circle (0.08);
                \fill[red!70] (0.25, -1.25) circle (0.08);

                % Primary rays from camera through pixels
                \draw[ray, very thick, opacity=0.5] (camera.center) -- (0.75, 1.25);
                \draw[ray, very thick, opacity=0.5] (camera.center) -- (0.25, -1.25);
                % Extended rays to objects
                \draw[ray, dashed, thick] (0.75, 1.25) -- (sphere1);
                \draw[ray, dashed, thick] (0.25, -1.25) -- (sphere2);

                % height and width of image plane
                \draw[<->, thin] (-2,-2.5) -- (2,-2.5) node[midway, below] {\textcolor{gray}{$w$}};
                \draw[<->, thin] (-2.5,-2) -- (-2.5,2) node[midway, right] {\textcolor{gray}{$h$}};
            \end{scope}
        \end{tikzpicture}
    \end{center}

    \begin{conceptbox}{Camera Description}
        Camera position $\mathbf{e}$, orthobasis $\{\mathbf{u}, \mathbf{v}, \mathbf{w}\}$, \\
        field of view $fov$, distance to near plane $n$, \\
        image dimensions $(w \times h)$.
    \end{conceptbox}
\end{frame}


\begin{frame}{Ray Generation Mathematics}
    \begin{columns}
        \begin{column}{0.5\textwidth}
            \begin{tikzpicture}[scale=0.7]
                % Define 3D styles for math diagram
                \tikzset{
                    camera/.style={fill=PrimaryColor!60, draw=PrimaryColor!80, rectangle, minimum size=8pt},
                    image plane/.style={fill=AccentColor!10, draw=AccentColor!50, opacity=0.8}
                }

                % Camera position
                \node[camera] (camera) at (0,0,0) {\faIcon{camera}};
                \node[below] at (0,-0.6,0) {\small $\mathbf{e}$};

                % Image plane with mathematical annotations
                \begin{scope}[canvas is zy plane at x=2.5]
                    \fill[image plane] (-1.5,-1.5) rectangle (1.5,1.5);

                    % Sample pixel
                    \fill[AccentColor!80] (0.5,0.8) rectangle (0.8,1.1);
                    \fill[red!70] (0.65, 0.95) circle (0.05);

                    % Dimensions
                    \draw[<->, thin] (-1.5,-1.8) -- (1.5,-1.8);
                    \node[below] at (0,-1.8) {\small $n_x$};
                    \draw[<->, thin] (-1.8,-1.5) -- (-1.8,1.5);
                    \node[left] at (-1.8,0) {\small $n_y$};

                    % Pixel coordinates
                    \draw[dashed, thin] (0.5,0.8) -- (0.5,-1.8);
                    \draw[dashed, thin] (-1.8,0.8) -- (0.5,0.8);
                    \node[below] at (0.5,-1.6) {\tiny $i$};
                    \node[left] at (-1.6,0.8) {\tiny $j$};
                \end{scope}

                % Ray from camera through pixel
                \draw[ray, very thick] (camera) -- (0.65, 0.95);
                \draw[ray, dashed, thick] (0.65, 0.95) -- (4, 1.5, 1.5);

                % Distance annotation
                \draw[<->, thin] (0,0.3,0) -- (2.5,0.3,0);
                \node[above] at (1.25,0.3,0) {\small $d$};

                % Coordinate system
                \draw[<->, thick, PrimaryColor] (0,0,0) -- (1,0,0);
                \draw[<->, thick, PrimaryColor] (0,0,0) -- (0,1,0);
                \draw[<->, thick, PrimaryColor] (0,0,0) -- (0,0,1);
                \node[below right] at (1,0,0) {\small $\mathbf{u}$};
                \node[above] at (0,1,0) {\small $\mathbf{v}$};
                \node[below] at (0,0,1) {\small $\mathbf{w}$};
            \end{tikzpicture}
        \end{column}
        \begin{column}{0.5\textwidth}
            \begin{mathbox}{Ray Equation}
                For pixel $(i,j)$:
                \begin{align}
                    s & = \frac{i + 0.5}{n_x} \\
                    t & = \frac{j + 0.5}{n_y}
                \end{align}

                \vspace{0.2cm}
                \textbf{Ray direction:}
                \begin{align}
                    \mathbf{d} & = (s - 0.5) \cdot \text{FOV} \cdot \mathbf{u}       \\
                               & \quad + (t - 0.5) \cdot \text{FOV} \cdot \mathbf{v} \\
                               & \quad + d \cdot \mathbf{w}
                \end{align}

                \vspace{0.2cm}
                \textbf{Parametric ray:}
                \begin{equation}
                    \mathbf{r}(\lambda) = \mathbf{e} + \lambda \mathbf{d}
                \end{equation}
            \end{mathbox}

            \vspace{0.3cm}
            \begin{itemize}
                \item $\mathbf{e}$: camera position
                \item $\{\mathbf{u}, \mathbf{v}, \mathbf{w}\}$: orthonormal basis
                \item $d$: distance to image plane
                \item FOV: field of view factor
                \item $\lambda \geq 0$: ray parameter
            \end{itemize}
        \end{column}
    \end{columns}
\end{frame}
