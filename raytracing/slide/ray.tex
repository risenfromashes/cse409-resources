\section{Ray Generation}
\begin{frame}{What is a Ray?}
    \begin{columns}
        \begin{column}{0.5\textwidth}
            \begin{mathbox}{Ray Representation}
                A ray is defined by:
                \begin{align}
                    \mathbf{P}(t) & = \mathbf{R_o} + t \cdot \mathbf{R_d}
                \end{align}
                where:
                \begin{itemize}
                    \item $\mathbf{R_o}$ = Origin point
                    \item $\mathbf{R_d}$ = Direction vector
                    \item $t$ = Parameter ($t \geq 0$)
                \end{itemize}
            \end{mathbox}
            \vspace{0.3cm}
            \href{https://www.desmos.com/calculator/0tloiudmdt}{Check out here on desmos.}
        \end{column}
        \begin{column}{0.5\textwidth}
            \begin{tikzpicture}[scale=0.8]
                % Origin

                % Points on ray
                \fill[SecondaryColor] (2,1.5) circle (3pt);
                \node[below] at (2,1) {$\mathbf{P}(t_1)$};
                \fill[SecondaryColor] (4,3) circle (3pt);
                \node[below] at (4,2.75) {$\mathbf{P}(t_2)$};


                % Direction vector
                \draw[->, very thick, red] (0,0) -- (2,1.5);
                \node[above] at (0.8,0.9) {$\mathbf{R_d}$};

                % Ray continuation
                \draw[ray, dashed] (2,1.5) -- (5,3.75);

                \fill[PrimaryColor] (0,0) circle (3pt);
                \node[below left] at (0,0) {$\mathbf{R_o}$};

                % Parameter illustration
                \draw[<->, thin] (0,-1) -- (2,-1);
                \node[below] at (1,-1) {$t_1=1$};
                \draw[<->, thin] (0,-2) -- (4,-2);
                \node[below] at (2,-2) {$t_2$};
            \end{tikzpicture}
        \end{column}
    \end{columns}
\end{frame}


\begin{frame}{The Heart of Ray Tracing}
    \begin{center}
        \huge \textcolor{PrimaryColor}{Finding Intersections}
    \end{center}

    \vspace{0.5cm}
    \begin{columns}
        \begin{column}{0.5\textwidth}
            \begin{tikzpicture}[scale=0.8]
                % Ray
                \draw[ray, very thick] (1.5,0.75) -- (4, 2);

                % Various objects
                \node[sphere] (sphere) at (2,1) {};
                \node[plane, rotate=30] (plane) at (3,0.5) {};
                \node[triangle] (tri) at (1,2) {};

                % Intersection points
                \fill[AccentColor] (1.5,0.75) circle (2pt);
                \fill[AccentColor] (2.5,1.25) circle (2pt);

                \draw[RayColor, very thick] (0,0) -- (1.5,0.75);
                \draw[RayColor, very thick, opacity=0.3] (1.5,0.75) -- (2.85,1.425);
                \node[below] at (0,0) {\raycolor{Ray}};

            \end{tikzpicture}
        \end{column}
        \begin{column}{0.5\textwidth}
            \textbf{Key Objects:}
            \begin{itemize}
                \item \objectcolor{Planes}
                \item \objectcolor{Spheres}
                \item \objectcolor{Triangles}
                \item \objectcolor{General Quadrics}
            \end{itemize}
            \vspace{0.5cm}
            \alert{Challenge:} Find the \textbf{closest} intersection efficiently!
        \end{column}
    \end{columns}
\end{frame}


\begin{frame}{3D Plane Representation}
    \begin{columns}
        \begin{column}{0.65\textwidth}
            \begin{mathbox}{Plane Definition}
                A plane is defined by:
                \begin{itemize}
                    \item Point $\mathbf{P_0} = (x_0, y_0, z_0)$ on plane
                    \item Normal vector $\mathbf{n} = (A, B, C)$
                \end{itemize}
                \textbf{Implicit equation:}
                \vspace*{-0.3cm}
                \begin{align*}
                    & \mathbf{n} \cdot (\mathbf{P} - \mathbf{P_0}) = 0 \\
                    & \mathbf{n} \cdot \mathbf{P} + D = 0 \text{ where }  D = -\mathbf{n} \cdot \mathbf{P_0} \\
                    & A x + B y + C z + D = 0\\
                \end{align*}
            \end{mathbox}
        \end{column}
        \begin{column}{0.35\textwidth}
            \begin{tikzpicture}[scale=0.6]
                % 3D coordinate system
                \draw[->] (0,0) -- (3,0) node[right] {$x$};
                \draw[->] (0,0) -- (0,3) node[above] {$y$};
                \draw[->] (0,0) -- (-1,-1) node[below left] {$z$};
                
                % 3D Plane using canvas transformation
                \begin{scope}[
                    plane origin={(2,1,0)},
                    plane x={(0.707,0, 0.707)},
                    plane y={(0,0.707,-0.707)},
                    canvas is plane]
                    \draw[thick, ObjectColor, fill=ObjectColor!20, opacity=0.6] 
                        (-1, -1) -- (-1, 1) -- (1, 1) -- (1, -1) -- cycle;

                    \only<1>{
                    \fill[AccentColor] (0.5,0.5) circle (2pt);
                    \node[above] at (0.5,0.5) {$\mathbf{P}$};
                    \draw[dashed, thin] (0, 0) -- (0.5,0.5);
                    }
                \end{scope}
                
                % Point on plane
                \fill[ObjectColor] (2,1,0) circle (2pt);
                \node[above right] at (2,1, 0) {$\mathbf{P_0}$};
                
                % Normal vector
                \draw[->, thick, SecondaryColor] (2, 1, 0) -- (1,2,-1);
                \node[above] at (1,2.5) {$\mathbf{n}$};
                
                \pause
                % Test point
                
                % Distance illustration
                \fill[AccentColor] (-0.5,2) circle (2pt);
                \node[above] at (-0.5,2) {$\mathbf{P}$};
                \draw[dashed, thin] (-0.5,2,0) -- (1,0.5,1.5) node[midway, left] {$d$};
            \end{tikzpicture}
        \end{column}
    \end{columns}
    \only<2->{
        \begin{conceptbox}{Point-Plane Distance}
            If $\mathbf{n}$ is normalized: $d = \mathbf{n} \cdot \mathbf{P} + D = \mathbf{n} \cdot (\mathbf{P} - \mathbf{P_0})$ \\
            \textbf{Signed distance:} $d > 0$ (front), $d < 0$ (back), $d = 0$ (on plane)
        \end{conceptbox}
    }
\end{frame}

\begin{frame}{Ray-Plane Intersection}
    \begin{columns}
        \begin{column}{0.6\textwidth}
            \begin{mathbox}{Intersection Method}
                \textbf{Step 1:} Substitute ray into equation
                \begin{align*}
                    \mathbf{n} \cdot (\mathbf{R_o} + t\mathbf{R_d}) + D &= 0\\
                    \mathbf{n} \cdot \mathbf{R_o} + t(\mathbf{n} \cdot \mathbf{R_d}) + D &= 0
                \end{align*}
                
                \textbf{Step 2:} Solve for parameter $t$
                \begin{align*}
                    t = -\frac{D + \mathbf{n} \cdot \mathbf{R_o}}{\mathbf{n} \cdot \mathbf{R_d}}
                \end{align*}
            \end{mathbox}
        \end{column}
        \begin{column}{0.4\textwidth}
            \begin{tikzpicture}[scale=0.8]
                % 3D Plane
                \begin{scope}[canvas is xy plane at z=0.3, transform shape]
                    \draw[thick, ObjectColor, fill=ObjectColor!20] 
                        (-0.5,0.5) -- (2.5,0.5) -- (3,2) -- (0,2) -- cycle;
                \end{scope}
                
                % Normal vector
                \draw[->, thick, SecondaryColor] (1.25,1.25) -- (0.75,2.25);
                \node[left] at (0.75,2.25) {$\mathbf{n}$};
                
                % Ray
                \draw[ray, very thick] (-1,-1) -- (2.5,2.5);
                \node[below] at (-1,-1) {$\mathbf{R_o}$};
                \draw[->, red, thick] (-1,-1) -- (0,0) node[midway, right] {$\mathbf{R_d}$};
                
                % Intersection point
                \fill[AccentColor] (1.25,1.25) circle (3pt);
                \node[left] at (1.25,1.25) {\textcolor{AccentColor}{$\mathbf{P}(t)$}};
                
                % Parameter illustration
                \draw[<->, thin] (-1,-2) -- (1.25,-2) node[midway, below] {$t$};
            \end{tikzpicture}
        \end{column}
    \end{columns}
    \pause
    \begin{conceptbox}{Cases}
        \begin{itemize}
            \item If $\mathbf{n} \cdot \mathbf{R_d} = 0$: Ray parallel to plane (0 or infinite)
            \item If $\mathbf{n} \cdot \mathbf{R_d} < 0$: Ray hits front face
            \item If $\mathbf{n} \cdot \mathbf{R_d} > 0$: Ray hits back face
        \end{itemize}
    \end{conceptbox}
\end{frame}


\begin{frame}{Additional Checks}
    \begin{columns}
        \begin{column}{0.6\textwidth}
            \begin{mathbox}{Validation Rules}
                After computing $t$, verify:
                \begin{enumerate}
                    \item \textbf{Behind check:} $t > t_{\text{min}}$
                    \item \textbf{Closest check:} $t < t_{\text{current}}$
                    \item \textbf{Valid range:} $t \geq 0$
                \end{enumerate}
                
                Where:
                \begin{itemize}
                    \item $t_{\text{min}}$: Minimum ray distance (not behind eye)
                    \item $t_{\text{current}}$: Distance to closest intersection so far
                \end{itemize}
            \end{mathbox}
        \end{column}
        \begin{column}{0.4\textwidth}
            \begin{tikzpicture}[scale=0.7]
                % Ray origin
                
                % Ray direction
                \draw[ray, very thick] (0,0) -- (5,2.5);
                
                % t_min region
                \draw[red, very thick] (0,0) -- (1.5,0.75) node[midway, above, xshift=-0.5cm] {\textcolor{red}{$t < t_{\text{min}}$}};

                \node[eye] (eye) at (1.5,0.75) {\faIcon{eye}};
                
                % Valid intersection
                \draw[green, very thick] (eye) -- (3.5,1.75) node[midway, below, yshift=-0.1cm] {\textcolor{green}{Valid}};
                
                % Too far intersection
                \fill[orange] (3.5,1.75) circle (3pt);
                \node[above, yshift=0.5cm, xshift=0.25cm] at (3.5,1.75) {\textcolor{orange}{$t > t_{\text{current}}$}};
                
                % Behind ray
                \fill[red] (-1,-0.5) circle (3pt);
                \node[below] at (-1,-0.5) {\textcolor{red}{$t < 0$}};
                \draw[red, dashed] (0,0) -- (-1,-0.5);

                \fill[PrimaryColor] (0,0) circle (3pt);
                \node[below] at (0.1,-0.1) {$\mathbf{R_o}$};
            \end{tikzpicture}
        \end{column}
    \end{columns}    

\end{frame}

\begin{frame}{Ray-Triangle Intersection Overview}
    \begin{columns}
        \begin{column}{0.65\textwidth}
            \begin{conceptbox}{Two Main Approaches}
                \textbf{Method 1: Two-Step Process}
                \begin{enumerate}
                    \item Ray-plane intersection
                    \item Inside/outside triangle test
                \end{enumerate}
                
                \vspace{0.5cm}
                \textbf{Method 2: Direct Barycentric}
                \begin{enumerate}
                    \item Set up 3×3 linear system
                    \item Solve for t, $\beta$, $\gamma$ simultaneously
                \end{enumerate}
            \end{conceptbox}
        \end{column}
        \begin{column}{0.35\textwidth}
            \begin{tikzpicture}[scale=0.6]
                \begin{scope}[plane x={(0.707, 0, 0.707)}, plane y={(0, 0.9, -0.436)}, canvas is plane]
                    % Triangle vertices in 3D space
                    \coordinate (A) at (0,0);
                    \coordinate (B) at (3,0);
                    \coordinate (C) at (1.5,2.5);
                    
                    % Triangle
                    \draw[thick, ObjectColor, fill=ObjectColor!20, opacity=0.7] (A) -- (B) -- (C) -- cycle;
                    
                    % Vertices
                    \fill[ObjectColor] (A) circle (2pt);
                    \fill[ObjectColor] (B) circle (2pt);
                    \fill[ObjectColor] (C) circle (2pt);
                    \node[below] at (A) {$\mathbf{a}$};
                    \node[below] at (B) {$\mathbf{b}$};
                    \node[above] at (C) {$\mathbf{c}$};
                \end{scope}
                
                % Ray coming from above
                \begin{scope}[canvas is xy plane at z=2]
                    \coordinate (RayStart) at (-1,1);
                \end{scope}
                \begin{scope}[canvas is xy plane at z=0]
                    \coordinate (RayEnd) at (2,1);
                \end{scope}
                
                \draw[ray, very thick] (RayStart) -- (RayEnd);
                \node[above] at (RayStart) {\raycolor{Ray}};
                
                % Intersection point
                \fill[AccentColor] (0.9,0.75) circle (3pt);
                \node[below right, xshift=0.25cm] at (0.9,0.75) {\textcolor{AccentColor}{Intersection}};

                               
                \draw[->, thick, SecondaryColor] (0.9,0.75) -- (-0.3,1.8) node[midway, left] {$\mathbf{n}$};
            \end{tikzpicture}
        \end{column}
    \end{columns}
\end{frame}


\begin{frame}{What Are Barycentric Coordinates?}
    \begin{columns}
        \begin{column}{0.5\textwidth}
            \begin{mathbox}{Barycentric Definition}
                Any point P in the triangle's plane:
                \begin{align*}
                    \mathbf{P}(\alpha,\beta,\gamma) &= \alpha\mathbf{a} + \beta\mathbf{b} + \gamma\mathbf{c}
                \end{align*}
                where: $\alpha + \beta + \gamma = 1$

                \vspace{0.3cm}

                \pause
                \textbf{Physical Interpretation:}
                \begin{itemize}
                    \item $\alpha$, $\beta$, $\gamma$ are \textit{weights}
                    \item P is the \textit{center of mass}
                    \item Also called \textit{barycenter}
                \end{itemize}
            \end{mathbox}
        \end{column}
        \begin{column}{0.5\textwidth}
            \begin{tikzpicture}[scale=0.8]
                \begin{scope}[canvas is xy plane at z=0]
                    % Triangle
                    \coordinate (A) at (0,0);
                    \coordinate (B) at (4,0);
                    \coordinate (C) at (2,3);
                    
                    \draw[thick, ObjectColor, fill=ObjectColor!15] (A) -- (B) -- (C) -- cycle;
                    
                    % Vertices with weights
                    \fill[ObjectColor] (A) circle (3pt);
                    \fill[ObjectColor] (B) circle (3pt);
                    \fill[ObjectColor] (C) circle (3pt);
                    \node[below] at (A) {$\mathbf{a}$ ($\alpha=0.3$)};
                    \node[below] at (B) {$\mathbf{b}$ ($\beta=0.5$)};
                    \node[above] at (C) {$\mathbf{c}$ ($\gamma=0.2$)};
                    
                    % Point P
                    \coordinate (P) at (2.2,1.2);
                    \fill[AccentColor] (P) circle (3pt);
                    \node[left] at (P) {\textcolor{AccentColor}{$\mathbf{P}$}};
                    
                    % Sub-triangles for area interpretation
                    \draw[thin, SecondaryColor, dashed] (P) -- (A);
                    \draw[thin, SecondaryColor, dashed] (P) -- (B);
                    \draw[thin, SecondaryColor, dashed] (P) -- (C);
                    
                    % Area labels
                    \fill[red!15, opacity=0.5] (A) -- (B) -- (P) -- cycle;
                    \fill[green!15, opacity=0.5] (C) -- (B) -- (P) -- cycle;
                    \fill[blue!15, opacity=0.5] (A) -- (C) -- (P) -- cycle;

                    \node[red] at (2.2,0.4) {$A_{\gamma}$};
                    \node[green] at (2.6,1.7) {$A_{\alpha}$};
                    \node[blue] at (1.6,1.7) {$A_{\beta}$};


                \end{scope}
            \end{tikzpicture}
            \vspace{0.3cm}
            \only<3->{
                \begin{conceptbox}{Area Relationship}
                    $\alpha = \frac{A_{\alpha}}{A_{total}}$, $\beta = \frac{A_{\beta}}{A_{total}}$, \\
                    $\gamma = \frac{A_{\gamma}}{A_{total}}$
                \end{conceptbox}
            }
        \end{column}
    \end{columns}
    \href{https://www.desmos.com/calculator/765wcscgs5}{Check out the Desmos demo.}
\end{frame}



\begin{frame}{Barycentric Coordinates: Inside vs Outside}
    \begin{columns}
        \begin{column}{0.5\textwidth}
            \begin{mathbox}{Triangle Interior Test}
                Point P is \textbf{inside} triangle if:
                \begin{align*}
                    \alpha, \beta, \gamma &\geq 0
                \end{align*}
                
                Since $\alpha + \beta + \gamma = 1$, we can rewrite as:
                \begin{align*}
                    \beta &\geq 0 \\
                    \gamma &\geq 0 \\
                    \beta + \gamma &\leq 1
                \end{align*}
            \end{mathbox}
        \end{column}
        \begin{column}{0.5\textwidth}
            \begin{tikzpicture}[scale=0.6]
                \begin{scope}[canvas is xy plane at z=0]
                    % Triangle
                    \coordinate (A) at (0,0);
                    \coordinate (B) at (4,0);
                    \coordinate (C) at (2,3);
                    
                    \draw[thick, ObjectColor, fill=ObjectColor!15] (A) -- (B) -- (C) -- cycle;
                    
                    % Vertices
                    \fill[ObjectColor] (A) circle (2pt);
                    \fill[ObjectColor] (B) circle (2pt);
                    \fill[ObjectColor] (C) circle (2pt);
                    \node[below] at (A) {$\mathbf{a}$};
                    \node[below] at (B) {$\mathbf{b}$};
                    \node[above] at (C) {$\mathbf{c}$};
                    
                    % Inside point
                    \fill[green] (1.5,1) circle (2pt);
                    \node[right, green] at (1.5,1) {Inside};
                    \node[right, green] at (0.3,0.3) {\scriptsize $\beta=0.4, \gamma=0.3$};
                    
                    % Outside points
                    \fill[red] (-0.5,1) circle (2pt);
                    \node[left, red] at (-0.5,1) {Outside};
                    \node[left, red] at (-0.5,0.4) {$\beta<0$};
                    
                    \fill[red] (3,3.5) circle (2pt);
                    \node[above, red] at (3,4.1) {Outside};
                    \node[above, red] at (3,3.5) {$\beta+\gamma>1$};
                    
                    \fill[red] (4.5,0.5) circle (2pt);
                    \node[right, red] at (4.5,0.5) {Outside};
                    \node[right, red] at (4.5,0.0) {$\gamma<0$};
                \end{scope}
            \end{tikzpicture}
        \end{column}
    \end{columns}
    \pause
    \begin{conceptbox}{Insight}
        Barycentric coordinates doesn't just tell us if a point is inside a triangle, but also it's position with respect to other vertices.
    \end{conceptbox}
\end{frame}

\endinput

\begin{frame}{Method 1: Two-Step Ray-Triangle Intersection}
    \begin{columns}
        \begin{column}{0.5\textwidth}
            \begin{mathbox}{Algorithm Steps}
                \textbf{Step 1:} Ray-Plane Intersection
                \begin{align}
                    \mathbf{n} \cdot (\mathbf{R_o} + t\mathbf{R_d}) + D &= 0 \\
                    t &= -\frac{D + \mathbf{n} \cdot \mathbf{R_o}}{\mathbf{n} \cdot \mathbf{R_d}}
                \end{align}
                
                \textbf{Step 2:} Inside/Outside Test
                \begin{align}
                    \mathbf{P} &= \mathbf{R_o} + t\mathbf{R_d} \\
                    \mathbf{P} &= \mathbf{a} + \beta(\mathbf{b}-\mathbf{a}) + \gamma(\mathbf{c}-\mathbf{a})
                \end{align}
                
                Solve for $\beta$, $\gamma$ and check bounds
            \end{mathbox}
        \end{column}
        \begin{column}{0.5\textwidth}
            \begin{tikzpicture}[scale=0.8]
                % Step 1: Ray-plane intersection
                \begin{scope}[plane x={(0.707, 0, 0.707)}, plane y={(0, 0.9, -0.436)}, canvas is plane]
                    % Extended plane
                    \fill[ObjectColor!10, opacity=0.3] (-1,-1) rectangle (5,4);
                    
                    % Triangle
                    \coordinate (A) at (0,0);
                    \coordinate (B) at (4,0);
                    \coordinate (C) at (2,3);
                    
                    \draw[thick, ObjectColor, fill=ObjectColor!20] (A) -- (B) -- (C) -- cycle;
                    
                    % Vertices
                    \fill[ObjectColor] (A) circle (2pt);
                    \fill[ObjectColor] (B) circle (2pt);
                    \fill[ObjectColor] (C) circle (2pt);
                    
                    % Plane intersection point (outside triangle)
                    \coordinate (PlaneHit) at (3.5,1);
                    \fill[orange] (PlaneHit) circle (2pt);
                    \node[right, orange] at (PlaneHit) {Step 1: Plane hit};
                    
                    % Triangle intersection point
                    \coordinate (TriHit) at (2,1);
                    \fill[AccentColor] (TriHit) circle (2pt);
                    \node[below, AccentColor] at (TriHit) {Step 2: Triangle hit};
                \end{scope}
                \draw[->, thick, SecondaryColor] (0.9,0.75) -- (-0.3,1.8) node[midway, left] {$\mathbf{n}$};
                
                % Rays
                \begin{scope}[canvas is xy plane at z=2]
                    \coordinate (RayStart1) at (2.5,0.5);
                    \coordinate (RayStart2) at (1,0.5);
                \end{scope}
                
                \draw[ray, orange] (RayStart1) -- (3.5,1);
                \draw[ray, AccentColor] (RayStart2) -- (2,1);
            
            \end{tikzpicture}
        \end{column}
    \end{columns}
    
    \begin{conceptbox}{Pros \& Cons}
        \textbf{Advantages:} Conceptually simple, reuses plane intersection code \\
        \textbf{Disadvantages:} Two separate computations, potential numerical issues
    \end{conceptbox}
\end{frame}


\endinput

\begin{frame}{Method 2: Direct Barycentric Intersection}
    \begin{columns}
        \begin{column}{0.5\textwidth}
            \begin{mathbox}{Direct Approach}
                Set ray equation equal to barycentric form:
                \begin{align}
                    \mathbf{R_o} + t\mathbf{R_d} &= \mathbf{a} + \beta(\mathbf{b}-\mathbf{a}) + \gamma(\mathbf{c}-\mathbf{a})
                \end{align}
                
                Rearrange to linear system:
                \begin{align}
                    \begin{bmatrix}
                        -\mathbf{R_d} & (\mathbf{b}-\mathbf{a}) & (\mathbf{c}-\mathbf{a})
                    \end{bmatrix}
                    \begin{bmatrix}
                        t \\ \beta \\ \gamma
                    \end{bmatrix}
                    = \mathbf{R_o} - \mathbf{a}
                \end{align}
                
                Solve using Cramer's rule or LU decomposition
            \end{mathbox}
        \end{column}
        \begin{column}{0.5\textwidth}
            \begin{tikzpicture}[scale=0.8]
                \begin{scope}[canvas is xy plane at z=0]
                    % Triangle
                    \coordinate (A) at (0,0);
                    \coordinate (B) at (4,0);
                    \coordinate (C) at (2,3);
                    
                    \draw[thick, ObjectColor, fill=ObjectColor!20] (A) -- (B) -- (C) -- cycle;
                    
                    % Vertices
                    \fill[ObjectColor] (A) circle (2pt);
                    \fill[ObjectColor] (B) circle (2pt);
                    \fill[ObjectColor] (C) circle (2pt);
                    \node[below] at (A) {$\mathbf{a}$};
                    \node[below] at (B) {$\mathbf{b}$};
                    \node[above] at (C) {$\mathbf{c}$};
                    
                    % Edge vectors
                    \draw[->, SecondaryColor, thick] (A) -- (B);
                    \node[below, SecondaryColor] at (2,0) {$\mathbf{b}-\mathbf{a}$};
                    \draw[->, SecondaryColor, thick] (A) -- (C);
                    \node[left, SecondaryColor] at (1,1.5) {$\mathbf{c}-\mathbf{a}$};
                    
                    % Intersection point
                    \coordinate (P) at (2.2,1.2);
                    \fill[AccentColor] (P) circle (3pt);
                    \node[right, AccentColor] at (P) {$\mathbf{P}(t,\beta,\gamma)$};
                    
                    % Parameter vectors
                    \draw[->, thin, AccentColor] (A) -- (P);
                    \node[below, AccentColor, small] at (1.1,0.6) {$\beta(\mathbf{b}-\mathbf{a}) + \gamma(\mathbf{c}-\mathbf{a})$};
                \end{scope}
                
                % Ray
                \begin{scope}[canvas is xy plane at z=2]
                    \coordinate (RayStart) at (1.2,0.2);
                \end{scope}
                
                \draw[ray, very thick] (RayStart) -- (2.2,1.2);
                \draw[->, RayColor, thick] (2.2,1.2) -- (3.2,2.2);
                \node[above, RayColor] at (2.7,1.7) {$\mathbf{R_d}$};
            \end{tikzpicture}
        \end{column}
    \end{columns}
\end{frame}

\begin{frame}{Cramer's Rule Solution}
    \begin{mathbox}{Matrix Form}
        \begin{align}
            \begin{bmatrix}
                -R_{dx} & (b_x-a_x) & (c_x-a_x) \\
                -R_{dy} & (b_y-a_y) & (c_y-a_y) \\
                -R_{dz} & (b_z-a_z) & (c_z-a_z)
            \end{bmatrix}
            \begin{bmatrix}
                t \\ \beta \\ \gamma
            \end{bmatrix}
            =
            \begin{bmatrix}
                R_{ox} - a_x \\
                R_{oy} - a_y \\
                R_{oz} - a_z
            \end{bmatrix}
        \end{align}
    \end{mathbox}
    
    \begin{columns}
        \begin{column}{0.5\textwidth}
            \begin{mathbox}{Cramer's Rule}
                Let $A$ be the coefficient matrix determinant
                \begin{align}
                    t &= \frac{1}{A} \begin{vmatrix}
                        (R_o-a)_x & (b-a)_x & (c-a)_x \\
                        (R_o-a)_y & (b-a)_y & (c-a)_y \\
                        (R_o-a)_z & (b-a)_z & (c-a)_z
                    \end{vmatrix} \\
                    \beta &= \frac{1}{A} \begin{vmatrix}
                        -R_{dx} & (R_o-a)_x & (c-a)_x \\
                        -R_{dy} & (R_o-a)_y & (c-a)_y \\
                        -R_{dz} & (R_o-a)_z & (c-a)_z
                    \end{vmatrix} \\
                    \gamma &= \frac{1}{A} \begin{vmatrix}
                        -R_{dx} & (b-a)_x & (R_o-a)_x \\
                        -R_{dy} & (b-a)_y & (R_o-a)_y \\
                        -R_{dz} & (b-a)_z & (R_o-a)_z
                    \end{vmatrix}
                \end{align}
            \end{mathbox}
        \end{column}
        \begin{column}{0.5\textwidth}
            \begin{conceptbox}{Implementation Notes}
                \textbf{Validity Check:}
                \begin{itemize}
                    \item $t > 0$ (intersection in front)
                    \item $\beta \geq 0$ (inside triangle)
                    \item $\gamma \geq 0$ (inside triangle)
                    \item $\beta + \gamma \leq 1$ (inside triangle)
                \end{itemize}
                
                \vspace{0.3cm}
                \textbf{Efficiency Tips:}
                \begin{itemize}
                    \item Early exit if $A \approx 0$ (parallel)
                    \item Test bounds before computing all determinants
                \end{itemize}
            \end{conceptbox}
        \end{column}
    \end{columns}
\end{frame}

\begin{frame}{Comparison and Applications}
    \begin{columns}
        \begin{column}{0.5\textwidth}
            \begin{conceptbox}{Method Comparison}
                \textbf{Two-Step Method:}
                \begin{itemize}
                    \item[+] Conceptually simple
                    \item[+] Reuses existing plane code
                    \item[-] Two separate computations
                    \item[-] Potential numerical issues
                \end{itemize}
                
                \vspace{0.5cm}
                \textbf{Barycentric Method:}
                \begin{itemize}
                    \item[+] Single computation
                    \item[+] Gets barycentric coords for free
                    \item[+] More numerically stable
                    \item[-] Slightly more complex setup
                \end{itemize}
            \end{conceptbox}
        \end{column}
        \begin{column}{0.5\textwidth}
            \begin{tikzpicture}[scale=0.8]
                \begin{scope}[canvas is xy plane at z=0]
                    % Triangle with texture coordinates
                    \coordinate (A) at (0,0);
                    \coordinate (B) at (4,0);
                    \coordinate (C) at (2,3);
                    
                    \draw[thick, ObjectColor, fill=ObjectColor!20] (A) -- (B) -- (C) -- cycle;
                    
                    % Vertices with texture coords
                    \fill[ObjectColor] (A) circle (2pt);
                    \fill[ObjectColor] (B) circle (2pt);
                    \fill[ObjectColor] (C) circle (2pt);
                    \node[below] at (A) {$\mathbf{a}$ (0,0)};
                    \node[below] at (B) {$\mathbf{b}$ (1,0)};
                    \node[above] at (C) {$\mathbf{c}$ (0.5,1)};
                    
                    % Intersection point
                    \coordinate (P) at (2.2,1.2);
                    \fill[AccentColor] (P) circle (3pt);
                    \node[right, AccentColor] at (P) {$\mathbf{P}$};
                    \node[right, AccentColor, small] at (2.2,0.9) {$tex = \beta t_b + \gamma t_c$};
                \end{scope}
            \end{tikzpicture}
            
            \vspace{0.3cm}
            \begin{alertbox}{Bonus Applications}
                Barycentric coordinates enable:
                \begin{itemize}
                    \item Texture coordinate interpolation
                    \item Normal vector interpolation
                    \item Color interpolation
                    \item Any per-vertex attribute!
                \end{itemize}
            \end{alertbox}
        \end{column}
    \end{columns}
\end{frame}


\begin{frame}{Ray-Sphere Intersection}
    \begin{columns}
        \begin{column}{0.5\textwidth}
            \begin{mathbox}{Sphere Equation}
                Implicit form (centered at origin):
                \begin{align}
                    \mathbf{P} \cdot \mathbf{P} - r^2 & = 0
                \end{align}

                Substituting ray equation:
                \begin{align}
                     & (\mathbf{R_o} + t\mathbf{R_d}) \cdot (\mathbf{R_o} + t\mathbf{R_d}) - r^2 = 0           \\
                     & t^2 + 2(\mathbf{R_d} \cdot \mathbf{R_o})t + (\mathbf{R_o} \cdot \mathbf{R_o} - r^2) = 0
                \end{align}

                Quadratic formula: $t = \frac{-b \pm \sqrt{b^2-4ac}}{2a}$
            \end{mathbox}
        \end{column}
        \begin{column}{0.5\textwidth}
            \begin{tikzpicture}[scale=0.8]
                % Sphere
                \draw[thick, ObjectColor] (0,0) circle (1.5);
                \node[below] at (0,-1.8) {\objectcolor{Sphere}};

                % Center
                \fill[ObjectColor] (0,0) circle (2pt);
                \node[below left] at (0,0) {$\mathbf{O}$};

                % Radius
                \draw[thin, ObjectColor] (0,0) -- (1.5,0);
                \node[below] at (0.75,0) {$r$};

                % Ray with two intersections
                \draw[ray] (-2,-0.5) -- (2,0.5);

                % Intersection points
                \fill[AccentColor] (-1.3,-0.35) circle (2pt);
                \fill[AccentColor] (1.3,0.35) circle (2pt);
                \node[above] at (-1.3,-0.35) {$t_1$};
                \node[above] at (1.3,0.35) {$t_2$};

                % Discriminant cases
                \node[right] at (2.5,1) {
                    \begin{minipage}{2cm}
                        \tiny
                        $\Delta > 0$: 2 roots\\
                        $\Delta = 0$: 1 root\\
                        $\Delta < 0$: no roots
                    \end{minipage}
                };
            \end{tikzpicture}
        \end{column}
    \end{columns}
\end{frame}

\begin{frame}{Ray-Triangle Intersection}
    \begin{columns}
        \begin{column}{0.5\textwidth}
            \begin{mathbox}{Barycentric Approach}
                Triangle defined by vertices $\mathbf{a}$, $\mathbf{b}$, $\mathbf{c}$:
                \begin{align}
                    \mathbf{P}(\beta,\gamma) & = \mathbf{a} + \beta(\mathbf{b}-\mathbf{a}) + \gamma(\mathbf{c}-\mathbf{a})
                \end{align}

                Set equal to ray equation:
                \begin{align}
                    \mathbf{R_o} + t\mathbf{R_d} & = \mathbf{a} + \beta(\mathbf{b}-\mathbf{a}) + \gamma(\mathbf{c}-\mathbf{a})
                \end{align}

                Solve 3×3 system for $t$, $\beta$, $\gamma$
            \end{mathbox}
        \end{column}
        \begin{column}{0.5\textwidth}
            \begin{tikzpicture}[scale=0.8]
                % Triangle
                \coordinate (A) at (0,0);
                \coordinate (B) at (3,0);
                \coordinate (C) at (1.5,2.5);

                \draw[thick, ObjectColor, fill=ObjectColor!20] (A) -- (B) -- (C) -- cycle;

                % Vertices
                \fill[ObjectColor] (A) circle (2pt);
                \fill[ObjectColor] (B) circle (2pt);
                \fill[ObjectColor] (C) circle (2pt);
                \node[below left] at (A) {$\mathbf{a}$};
                \node[below right] at (B) {$\mathbf{b}$};
                \node[above] at (C) {$\mathbf{c}$};

                % Ray
                \draw[ray] (-1,1) -- (2.5,1.5);

                % Intersection point
                \fill[AccentColor] (1.2,1.2) circle (2pt);

                % Barycentric coordinates illustration
                \draw[thin, SecondaryColor] (A) -- (1.2,1.2);
                \draw[thin, SecondaryColor] (B) -- (1.2,1.2);
                \draw[thin, SecondaryColor] (C) -- (1.2,1.2);

                \node[right] at (3.5,1.5) {
                    \begin{minipage}{2cm}
                        \tiny
                        Inside if:\\
                        $\beta \geq 0$\\
                        $\gamma \geq 0$\\
                        $\beta + \gamma \leq 1$
                    \end{minipage}
                };
            \end{tikzpicture}
        \end{column}
    \end{columns}
\end{frame}