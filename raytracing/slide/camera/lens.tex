\subsection{Thin Lens Camera}

\begin{frame}{Thin Lens Camera: Fundamentals}
  \begin{columns}
    \begin{column}{0.5\textwidth}
      \begin{mathbox}{Gaussian Lens Equation}
        \textbf{Fundamental relationship:}
        \begin{align}
          \frac{1}{f} = \frac{1}{z} + \frac{1}{z'}
        \end{align}

        \textbf{Where:}
        \begin{itemize}
          \item $f$ = focal length of lens
          \item $z$ = object distance from lens
          \item $z'$ = image distance from lens
        \end{itemize}

        \vspace{0.3cm}
        \textbf{Key Properties:}
        \begin{itemize}
          \item Objects at focal plane are in perfect focus
          \item Other distances create blur (circle of confusion)
          \item Aperture size controls depth of field
        \end{itemize}
      \end{mathbox}
    \end{column}
    \begin{column}{0.5\textwidth}
      \begin{tikzpicture}[scale=0.7]
        % Define styles for thin lens
        % Optical axis
        \draw[optical axis] (-4,0) -- (6,0);

        % Thin lens at origin
        \lens{(0,0)}{2cm}{5cm};
        \node[below] at (0,-2.8) {\textcolor{PrimaryColor}{Thin Lens}};

        % Object at distance z
        \draw[thick, AccentColor] (-3,-1.5) -- (-3,1.5);
        \node[left] at (-3,0) {\objectcolor{Object}};
        \draw[<->, thin] (-3,-3) -- (0,-3);
        \node[below] at (-1.5,-3) {$z$};

        % Image at distance z'
        \draw[thick, SecondaryColor] (4,-1) -- (4,1);
        \node[right] at (4,0) {\textcolor{SecondaryColor}{Image}};
        \draw[<->, thin] (0,-3.5) -- (4,-3.5);
        \node[below] at (2,-3.5) {$z'$};

        % Focal points
        \fill[AccentColor] (-2,0) circle (2pt);
        \fill[AccentColor] (2,0) circle (2pt);
        \node[below] at (-2,-0.3) {\textcolor{AccentColor}{$F$}};
        \node[below] at (2,-0.3) {\textcolor{AccentColor}{$F'$}};

        % Focal length markers
        \draw[<->, thin] (-2,2.8) -- (0,2.8);
        \node[above] at (-1,2.8) {$f$};
        \draw[<->, thin] (0,3.2) -- (2,3.2);
        \node[above] at (1,3.2) {$f$};

        % Principal rays
        \draw[object ray] (-3,1.5) -- (0,1.5) -- (4,-1);
        \draw[object ray] (-3,1.5) -- (0,0.75) -- (4,-1);
        \draw[object ray] (-3,0) -- (0,0) -- (4,0);

        % Ray labels
        \node[above] at (-1.5,1.5) {\objectcolor{\tiny Parallel ray}};
        \node[above] at (2,0.7) {\textcolor{SecondaryColor}{\tiny Through focus}};
      \end{tikzpicture}
    \end{column}
  \end{columns}

  \begin{conceptbox}{Real Camera Behavior}
    Unlike pinhole cameras, thin lens cameras exhibit \textbf{depth of field} - only objects at the focal distance are perfectly sharp.
  \end{conceptbox}
\end{frame}

\begin{frame}{Depth of Field and Circle of Confusion}
  \begin{columns}
    \begin{column}{0.5\textwidth}
      \begin{mathbox}{Circle of Confusion}
        \textbf{For objects not at focal distance:}
        \begin{align}
          c = \frac{A}{z'} \left| z'_{focus} - z' \right|
        \end{align}

        \textbf{Where:}
        \begin{itemize}
          \item $c$ = circle of confusion diameter
          \item $A$ = aperture diameter
          \item $z'$ = image distance for object
          \item $z'_{focus}$ = image distance for focus
        \end{itemize}

        \vspace{0.3cm}
        \textbf{Depth of Field:}
        \begin{itemize}
          \item Region where $c < $ acceptable limit
          \item Larger aperture $\rightarrow$ shallower DOF
          \item Smaller aperture $\rightarrow$ deeper DOF
        \end{itemize}
      \end{mathbox}
    \end{column}
    \begin{column}{0.5\textwidth}
      \begin{tikzpicture}[scale=0.6]
        % Three object planes at different distances
        \draw[thick, AccentColor] (-4,2) -- (-4,-2);
        \draw[thick, AccentColor] (-2.5,1.5) -- (-2.5,-1.5);
        \draw[thick, AccentColor] (-1,1) -- (-1,-1);

        \node[left] at (-4,0) {\objectcolor{\tiny Far}};
        \node[left] at (-2.5,0) {\objectcolor{\tiny Focus}};
        \node[left] at (-1,0) {\objectcolor{\tiny Near}};

        % Lens
        \lens{(0,0)}{2cm}{6cm};
        \node[below] at (0,-3.3) {\textcolor{PrimaryColor}{Lens}};

        % Image plane
        \draw[thick, gray] (3,-3) -- (3,3);
        \node[right] at (3,3.3) {\textcolor{gray}{Image Plane}};

        % Rays from far object (out of focus)
        \draw[object ray, opacity=0.7] (-4,1) -- (0,0.3) -- (3,-0.8);
        \draw[object ray, opacity=0.7] (-4,-1) -- (0,-0.3) -- (3,0.8);
        \fill[red, opacity=0.6] (3,0) circle (8pt);
        \node[right] at (3.3,0.5) {\textcolor{red}{\tiny Large blur}};

        % Rays from focus object (sharp)
        \draw[object ray] (-2.5,0.75) -- (0,0.3) -- (3,-0.6);
        \draw[object ray] (-2.5,-0.75) -- (0,-0.3) -- (3,0.6);
        \fill[green!80!black] (3,0) circle (2pt);
        \node[right] at (3.3,-0.5) {\textcolor{green!80!black}{\tiny Sharp}};

        % Rays from near object (out of focus)
        \draw[object ray, opacity=0.7] (-1,0.5) -- (0,0.25) -- (3,-0.4);
        \draw[object ray, opacity=0.7] (-1,-0.5) -- (0,-0.25) -- (3,0.4);
        \fill[orange, opacity=0.6] (3,0) circle (5pt);
        \node[right] at (3.3,-1) {\textcolor{orange}{\tiny Medium blur}};

        % Aperture indication
        \draw[thick, blue] (-0.2,-1.5) -- (0.2,-1.5);
        \draw[thick, blue] (-0.2,1.5) -- (0.2,1.5);
        \node[right] at (0.3,0) {\textcolor{blue}{\tiny Aperture}};
      \end{tikzpicture}
    \end{column}
  \end{columns}

  \begin{conceptbox}{Artistic Control}
    \textbf{Shallow DOF}: Subject isolation, portrait photography\\
    \textbf{Deep DOF}: Landscape photography, technical documentation
  \end{conceptbox}
\end{frame}

\begin{frame}{Aperture Effects on Depth of Field}
  \begin{center}
    \begin{tikzpicture}[scale=0.8]
      % Large aperture scene
      \node[above] at (-3,3) {\textbf{Large Aperture (f/1.4)}};

      % Lens with large aperture
      \lens{(-3,0)}{2cm}{4cm};
      \draw[thick, blue] (-3.1,-1.5) -- (-2.9,-1.5);
      \draw[thick, blue] (-3.1,1.5) -- (-2.9,1.5);
      \node[left] at (-3.3,0) {\textcolor{blue}{\tiny Wide}};

      % Objects at different distances
      \draw[thick, AccentColor] (-5,0.5) -- (-5,-0.5);
      \draw[thick, AccentColor] (-4,1) -- (-4,-1);
      \draw[thick, AccentColor] (-3.5,1.5) -- (-3.5,-1.5);

      % Image plane
      \draw[thick, gray] (-1,-2) -- (-1,2);

      % Blur circles (large aperture = shallow DOF)
      \fill[red, opacity=0.4] (-1,0) circle (10pt);
      \fill[green!80!black] (-1,0) circle (3pt);
      \fill[orange, opacity=0.4] (-1,0) circle (7pt);

      \node[below] at (-3,-2.5) {\textcolor{red}{\tiny Shallow DOF}};

      % vs divider
      \node at (0,0) {\huge \textcolor{AccentColor}{vs}};

      % Small aperture scene
      \node[above] at (3,3) {\textbf{Small Aperture (f/11)}};

      % Lens with small aperture
      \lens{(3,0)}{2cm}{4cm};
      \draw[thick, blue] (2.95,-0.5) -- (3.05,-0.5);
      \draw[thick, blue] (2.95,0.5) -- (3.05,0.5);
      \node[right] at (3.3,0) {\textcolor{blue}{\tiny Narrow}};

      % Objects at different distances
      \draw[thick, AccentColor] (1,0.5) -- (1,-0.5);
      \draw[thick, AccentColor] (2,1) -- (2,-1);
      \draw[thick, AccentColor] (2.5,1.5) -- (2.5,-1.5);

      % Image plane
      \draw[thick, gray] (5,-2) -- (5,2);

      % Blur circles (small aperture = deep DOF)
      \fill[green!80!black] (5,0) circle (3pt);
      \fill[green!80!black] (5,0) circle (3pt);
      \fill[green!80!black] (5,0) circle (3pt);

      \node[below] at (3,-2.5) {\textcolor{green!80!black}{\tiny Deep DOF}};
    \end{tikzpicture}
  \end{center}

  \vspace{0.5cm}
  \begin{columns}
    \begin{column}{0.5\textwidth}
      \begin{raybox}{Large Aperture}
        \begin{itemize}
          \item More light gathering
          \item Faster shutter speeds
          \item Shallow depth of field
          \item Background blur (bokeh)
        \end{itemize}
      \end{raybox}
    \end{column}
    \begin{column}{0.5\textwidth}
      \begin{raybox}{Small Aperture}
        \begin{itemize}
          \item Less light gathering
          \item Slower shutter speeds
          \item Deep depth of field
          \item Everything in focus
        \end{itemize}
      \end{raybox}
    \end{column}
  \end{columns}
\end{frame}

\begin{frame}{Thin Lens Ray Generation}
  \begin{columns}
    \begin{column}{0.5\textwidth}
      \begin{mathbox}{Ray Sampling Process}
        \textbf{1. Sample pixel position} $(x, y)$

        \textbf{2. Sample lens position:}
        \begin{align}
          (u, v) &\sim \text{Uniform disk}\\
          \mathbf{p}_{lens} &= (u \cdot r, v \cdot r, 0)
        \end{align}

        \textbf{3. Compute focal point:}
        \begin{align}
          \mathbf{p}_{focus} &= \mathbf{p}_{pixel} \cdot \frac{f_{dist}}{n}
        \end{align}

        \textbf{4. Ray from lens to focal point:}
        \begin{align}
          \mathbf{R_o} &= \mathbf{p}_{lens}\\
          \mathbf{R_d} &= \mathbf{p}_{focus} - \mathbf{p}_{lens}
        \end{align}

        where $r$ = aperture radius, $f_{dist}$ = focus distance
      \end{mathbox}
    \end{column}
    \begin{column}{0.5\textwidth}
      \begin{tikzpicture}[scale=0.7]
        % Define styles for ray generation
        \tikzset{
          lens disk/.style={fill=PrimaryColor!30, draw=PrimaryColor!80},
          sample point/.style={fill=red!80, circle, inner sep=1pt},
          focal point/.style={fill=AccentColor!80, circle, inner sep=1.5pt},
          sample ray/.style={->, thick, red!70},
          focal plane/.style={dashed, AccentColor}
        }

        % Image plane
        \draw[thick, gray] (-3,-2) -- (-3,2);
        \node[left] at (-3,2.3) {\textcolor{gray}{Image Plane}};

        % Pixel grid
        \foreach \y in {-1.5,-1,-0.5,0,0.5,1,1.5} {
          \draw[thin, gray!50] (-3.2,\y) -- (-2.8,\y);
        }

        % Sample pixel
        \fill[AccentColor!60] (-3,1) rectangle (-2.8,1.2);
        \fill[red!70] (-2.9,1.1) circle (1.5pt);
        \node[left] at (-3.3,1.1) {\textcolor{red!70}{\tiny Sample}};

        % Lens disk
        \fill[lens disk] (0,0) circle (1);
        \draw[thick, PrimaryColor] (0,-1) -- (0,1);
        \node[below] at (0,-1.3) {\textcolor{PrimaryColor}{Lens}};

        % Sample points on lens
        \node[sample point] at (0.3,0.4) {};
        \node[sample point] at (-0.5,-0.2) {};
        \node[sample point] at (0.1,-0.7) {};
        \node[sample point] at (-0.3,0.6) {};

        % Focal plane
        \draw[focal plane] (3,-2) -- (3,2);
        \node[right] at (3,2.3) {\textcolor{AccentColor}{Focal Plane}};

        % Focal point (where pixel ray hits focal plane)
        \fill[focal point] (3,0.55) circle (2pt);
        \node[right] at (3.2,0.55) {\textcolor{AccentColor}{\tiny Focus}};

        % Sample rays from lens points to focal point
        \draw[sample ray] (0.3,0.4) -- (3,0.55);
        \draw[sample ray] (-0.5,-0.2) -- (3,0.55);
        \draw[sample ray] (0.1,-0.7) -- (3,0.55);
        \draw[sample ray] (-0.3,0.6) -- (3,0.55);

        % Central ray (through lens center)
        \draw[dashed, gray] (-2.9,1.1) -- (0,0) -- (3,0.55);

        % Distance markers
        \draw[<->, thin] (-3,-2.7) -- (0,-2.7);
        \node[below] at (-1.5,-2.7) {$n$};
        \draw[<->, thin] (0,-2.7) -- (3,-2.7);
        \node[below] at (1.5,-2.7) {$f_{dist}$};
      \end{tikzpicture}
    \end{column}
  \end{columns}

  \begin{conceptbox}{Monte Carlo Integration}
    Multiple samples per pixel with different lens positions create realistic \textbf{depth of field blur}
  \end{conceptbox}
\end{frame}

\begin{frame}{Other Camera Types}
  \begin{columns}
    \begin{column}{0.5\textwidth}
      \begin{conceptbox}{Fish-eye Camera}
        \begin{itemize}
          \item Very wide field of view (>180°)
          \item Non-linear distortion
          \item Curved ray paths
          \item Surveillance, VR applications
        \end{itemize}
      \end{conceptbox}

      \vspace{0.3cm}
      \begin{conceptbox}{Environment Camera}
        \begin{itemize}
          \item 360° panoramic view
          \item Spherical or cylindrical
          \item HDRI environment maps
          \item VR content creation
        \end{itemize}
      \end{conceptbox}
    \end{column}
    \begin{column}{0.5\textwidth}
      \begin{tikzpicture}[scale=0.8]
        % Fish-eye illustration
        \node[above] at (0,3) {\textbf{Fish-eye FOV}};
        \node[eye] (fisheye) at (0,2) {\faIcon{eye}};
        \draw[ray, bend left=30] (fisheye) to (2,3);
        \draw[ray] (fisheye) -- (2,2);
        \draw[ray, bend right=30] (fisheye) to (2,1);
        \draw[ray, bend left=60] (fisheye) to (1,3.5);
        \draw[ray, bend right=60] (fisheye) to (1,0.5);
        \node[right] at (2.2,2) {\tiny Wide FOV};

        % Motion blur illustration
        \node[above] at (0,0.5) {\textbf{Motion Blur}};
        \draw[thick, gray] (-1,-1) -- (1,-1);
        \node[left] at (-1,-1) {\textcolor{gray}{\tiny Image plane}};
        \draw[ray, opacity=0.3] (-0.5,-0.5) -- (0.5,-1);
        \draw[ray, opacity=0.5] (-0.3,-0.5) -- (0.3,-1);
        \draw[ray, opacity=0.7] (-0.1,-0.5) -- (0.1,-1);
        \draw[ray] (0.1,-0.5) -- (-0.1,-1);
        \node[right] at (1.2,-1) {\tiny Multiple samples};
      \end{tikzpicture}
    \end{column}
  \end{columns}

  \begin{columns}
    \begin{column}{0.5\textwidth}
      \begin{conceptbox}{Environment Camera}
        \begin{itemize}
          \item 360° panoramic view
          \item Spherical or cylindrical
          \item HDRI environment maps
          \item VR content creation
        \end{itemize}
      \end{conceptbox}
    \end{column}
    \begin{column}{0.5\textwidth}
      \begin{conceptbox}{Motion Blur Camera}
        \begin{itemize}
          \item Simulates camera/object motion
          \item Multiple time samples
          \item Temporal anti-aliasing
          \item Dynamic scene rendering
        \end{itemize}
      \end{conceptbox}
    \end{column}
  \end{columns}
\end{frame}
