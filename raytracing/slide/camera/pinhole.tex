\subsection{Pinhole Camera}
\begin{frame}{The Pinhole Camera Model}
  \begin{columns}
    \begin{column}{0.5\textwidth}
      \only<1-5>{
        \begin{mathbox}{Pinhole Camera}
          \footnotesize
          The simplest camera. A box with a pinhole. Light entering the pinhole creates an image on the other side of the box.
          \only<2->{
            \begin{itemize}
              \item \textbf{Point aperture} \\
                \only<2>{Just a single point for rays to pass through.}
              \item \textbf{Perfect focus everywhere} \\
                \only<2>{All rays go through the pinhole. Therefore, all points in the image plane correspond to exactly one point in the scene.}
                \only<3->{
                \item \textbf{Inverted Image} \\
                  \only<3> {Rays from the top of the scene hit the bottom of the image plane, and vice versa.}
                }
            \end{itemize}
          }

          \only<4->{
            \textbf{Ray Generation:}
            \begin{align*}
              \mathbf{R_o} & = \mathbf{hole}                  \\
              \mathbf{R_d} & = \frac{\mathbf{hole} - \mathbf{pixel}}{|\mathbf{hole} - \mathbf{pixel}|}
            \end{align*}
          }
        \end{mathbox}
      }
      \only<6->{
        \begin{raybox}{Imaginary Plane}
          Let's place an imaginary plane in front of the hole.
        \end{raybox}
      }
    \end{column}
    \begin{column}{0.5\textwidth}
      \begin{tikzpicture}[scale=0.35]
        \tikzset{
          screen/.style={fill=blue!10, draw=blue!50, opacity=0.8},
          pixel/.style={fill=AccentColor!60, thick},
          primary ray/.style={->, very thick, red!90},
          object/.style={fill=orange!60, draw=orange!80, circle, minimum size=12pt}
        }
        \coordinate (hole) at (3, 0, 0);

        \node[below] at (-3,-4,0) {\scriptsize Image Plane};

        \begin{scope}[canvas is zy plane at x=-3]
          % screen
          \fill[screen] (-3,-3) rectangle (3,3);
          \foreach \x in {-3,-2,...,3} {
            \draw[gray!60, thin] (\x,-3) -- (\x,3);
          }
          \foreach \z in {-3,-2,...,3} {
            \draw[gray!60, thin] (-3,\z) -- (3,\z);
          }

          \foreach \x in {-2.5,-1.5,...,2.5} {
            \foreach \z in {-2.5,-1.5,...,2.5} {
              \coordinate (dir) at ($(hole)-(\x,\z)$);
              \draw[ray, thin, opacity=0.3] (\x,\z) -- (hole);
              \draw[red!40] (\x,\z) circle (1pt);
            }
          }
          \only<3->{
            \draw[AccentColor, fill=AccentColor!30, opacity=0.7] (2.5, 1.5) -- (-2.5, 2.5) -- (0.5, -2.5) -- cycle;
          }
          \draw[ray, thick, opacity=0.9] (-2.5,2.5) -- (hole);
          \draw[ray, thick, opacity=0.9] (0.5,-2.5) -- (hole);
        \end{scope}

        \draw[-, PrimaryColor] (-3,-3,-3) -- (3, -3, -3);
        \draw[-, PrimaryColor] (-3,-3,3) -- (3, -3, 3);
        \draw[-, PrimaryColor] (-3,3,-3) -- (3, 3, -3);
        \draw[-, PrimaryColor] (-3,3,3) -- (3, 3, 3);
        \begin{scope}[canvas is zy plane at x=3]
          \draw[black, fill=gray!60] (-3,-3) rectangle (3,3);
          \draw[PrimaryColor,fill=white] (0,0) circle (3pt) node[above] {\tiny Pinhole};
        \end{scope}
        \begin{scope}[canvas is zy plane at x=-3]
          \foreach \x in {-2.5,-1.5,...,2.5} {
            \foreach \z in {-2.5,-1.5,...,2.5} {
              \coordinate (dir) at ($(hole)-(\x,\z)$);
              \draw[ray, thin, opacity=0.3] (hole) -- ($(hole)+(dir)$);
            }
          }

          \coordinate (dir1) at ($(hole)-(-2.5,2.5)$);
          \coordinate (dir2) at ($(hole)-(0.5,-2.5)$);
          \draw[ray, thick, opacity=0.9] (hole) -- ($(hole)+(dir1)$);
          \draw[ray, thick, opacity=0.9] (hole) -- ($(hole)+(dir2)$);
        \end{scope}
        \only<6->{
          \begin{scope}[canvas is zy plane at x=9]
            % screen
            \fill[screen] (-3,-3) rectangle (3,3);
            \foreach \x in {-3,-2,...,3} {
              \draw[gray!60, thin] (\x,-3) -- (\x,3);
            }
            \foreach \z in {-3,-2,...,3} {
              \draw[gray!60, thin] (-3,\z) -- (3,\z);
            }

            \foreach \x in {-2.5,-1.5,...,2.5} {
              \foreach \z in {-2.5,-1.5,...,2.5} {
                \coordinate (dir) at ($(hole)-(\x,\z)$);
                \draw[ray, thin, opacity=0.3] (hole) -- (\x,\z);
                \draw[red!40] (\x,\z) circle (1pt);
              }
            }
            \draw[AccentColor, fill=AccentColor!30, opacity=0.7] (-2.5, -1.5) -- (2.5, -2.5) -- (-0.5, 2.5) -- cycle;
          \end{scope}
          \draw[<->] (-3,-4.5,3) -- (3,-4.5,3) node[midway, below] {\scriptsize $d$};
          \draw[<->] (3,-4.5,3) -- (9,-4.5,3) node[midway, below] {\scriptsize $d$};
        }
        \only<1-5>{
          \draw[AccentColor, fill=AccentColor!30] (9, -2, 3) -- (8, 1.5, -3) -- (13, -3, 4) -- cycle;
          \node[below] at (10,-4.5,0) {\scriptsize \textcolor{AccentColor}{Object}};
        }
      \end{tikzpicture}
    \end{column}
  \end{columns}
  \only<5>{
    \footnotesize
    Real pinhole cameras exist! They create sharp images but require very long exposure times due to tiny aperture.
  }
\end{frame}

\begin{frame}{Simplified Pinhole Camera}
  \begin{columns}
    \begin{column}{0.5\textwidth}
      \begin{mathbox}{Simplification}
        \footnotesize
        Place image plane in front! \\
        Equivalent to pinhole camera.
        \only<1>{
          The image plane will mirror the image of the pinhole camera if the plane is placed in the same distance in front of the eye.
        }

        \only<2->{
          \begin{itemize}
            \item \textbf{Physically unrealizable} \\
              \only<2> {A real cannot have a sensor/film in front of the pinhole.}
              \only<3->{
              \item \textbf{Non-inverted image} \\
                \only<3> {The image plane is in front of the pinhole, so the rays hit the image plane directly.}
              }
          \end{itemize}
        }
        \only<4->{
          \textbf{Ray Generation:}
          \begin{align*}
            \mathbf{R_o} & = \mathbf{eye}                  \\
            \mathbf{R_d} & = \frac{\mathbf{pixel} - \mathbf{eye}}{|\mathbf{pixel} - \mathbf{eye}|}
          \end{align*}
        }
      \end{mathbox}
    \end{column}
    \begin{column}{0.5\textwidth}
      \centering
      \begin{tikzpicture}[scale=0.5]
        \tikzset{
          screen/.style={fill=blue!10, draw=blue!50, opacity=0.8}
        }
        \node[eye] (eye) at (0,0,0) {\faIcon{eye}};
        \draw[AccentColor, fill=AccentColor!30] (7.5, -1, 3) -- (6.25, 2.25, -3) -- (11.5, -2, 4) -- cycle;
        \node[below] at (8,-3,0) {\scriptsize \textcolor{AccentColor}{Object}};
        \node[below] at (0,-0.8,0) {\scriptsize Eye};
        \node[above] at (3.5,4,0) {\scriptsize Screen};
        \begin{scope}[canvas is zy plane at x=3.5]
          \fill[screen] (-3,-3) rectangle (3,3);
          \foreach \x in {-3,-2,...,3} {
            \draw[gray!60, thin] (\x,-3) -- (\x,3);
          }
          \foreach \z in {-3,-2,...,3} {
            \draw[gray!60, thin] (-3,\z) -- (3,\z);
          }
          \foreach \x in {-2.5,-1.5,...,2.5} {
            \foreach \z in {-2.5,-1.5,...,2.5} {
              \fill[red!40] (\x, \z) circle (2pt);
              \coordinate (dir) at ($(\x,\z)-(eye.center)$);
              \draw[ray, thin, opacity=0.3] (eye.center) -- (\x,\z);
              \draw[ray, thin, opacity=0.3] (\x,\z) -- ($(\x,\z)+(dir)$);
            }
          }
          \only<3->{
            \draw[AccentColor, fill=AccentColor!30, opacity=0.7] (-0.5,1.5) -- (-1.5,-1.5) -- (1.5,-1.5) -- cycle;
          }
          \coordinate (p1) at (-0.5,1.5);
          \coordinate (dir1) at ($(p1)-(eye.center)$);
          \draw[ray, thick, opacity=0.9] (eye.center) -- ($(p1)+(dir1)$);
          \coordinate (p2) at (-0.5,-1.5);
          \coordinate (dir2) at ($(p2)-(eye.center)$);
          \draw[ray, thick, opacity=0.9] (eye.center) -- ($(p2)+(dir2)$);
        \end{scope}
      \end{tikzpicture}
    \end{column}
  \end{columns}
  \vspace{-0.5cm}
  \only<5->{
    \begin{conceptbox}{Advantage}
      \footnotesize
      \textbf{Upright image}, simpler ray generation, equivalent to real pinhole!
    \end{conceptbox}
  }
\end{frame}