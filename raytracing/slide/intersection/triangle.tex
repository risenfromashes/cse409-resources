
\begin{frame}{Ray-Triangle Intersection Overview}
    \begin{columns}
        \begin{column}{0.65\textwidth}
            \begin{conceptbox}{Two Main Approaches}
                \textbf{Method 1: Two-Step Process}
                \begin{enumerate}
                    \item Ray-plane intersection
                    \item Inside/outside triangle test
                \end{enumerate}
                
                \vspace{0.5cm}
                \textbf{Method 2: Direct Barycentric}
                \begin{enumerate}
                    \item Set up 3×3 linear system
                    \item Solve for t, $\beta$, $\gamma$ simultaneously
                \end{enumerate}
            \end{conceptbox}
        \end{column}
        \begin{column}{0.35\textwidth}
            \begin{tikzpicture}[scale=0.6]
                \begin{scope}[plane x={(0.707, 0, 0.707)}, plane y={(0, 0.9, -0.436)}, canvas is plane]
                    % Triangle vertices in 3D space
                    \coordinate (A) at (0,0);
                    \coordinate (B) at (3,0);
                    \coordinate (C) at (1.5,2.5);
                    
                    % Triangle
                    \draw[thick, ObjectColor, fill=ObjectColor!20, opacity=0.7] (A) -- (B) -- (C) -- cycle;
                    
                    % Vertices
                    \fill[ObjectColor] (A) circle (2pt);
                    \fill[ObjectColor] (B) circle (2pt);
                    \fill[ObjectColor] (C) circle (2pt);
                    \node[below] at (A) {$\mathbf{a}$};
                    \node[below] at (B) {$\mathbf{b}$};
                    \node[above] at (C) {$\mathbf{c}$};
                \end{scope}
                
                % Ray coming from above
                \begin{scope}[canvas is xy plane at z=2]
                    \coordinate (RayStart) at (-1,1);
                \end{scope}
                \begin{scope}[canvas is xy plane at z=0]
                    \coordinate (RayEnd) at (2,1);
                \end{scope}
                
                \draw[ray, very thick] (RayStart) -- (RayEnd);
                \node[above] at (RayStart) {\raycolor{Ray}};
                
                % Intersection point
                \fill[AccentColor] (0.9,0.75) circle (3pt);
                \node[below right, xshift=0.25cm] at (0.9,0.75) {\textcolor{AccentColor}{Intersection}};

                               
                \draw[->, thick, SecondaryColor] (0.9,0.75) -- (-0.3,1.8) node[midway, left] {$\mathbf{n}$};
            \end{tikzpicture}
        \end{column}
    \end{columns}
\end{frame}


\begin{frame}{What Are Barycentric Coordinates?}
    \begin{columns}
        \begin{column}{0.5\textwidth}
            \begin{mathbox}{Barycentric Definition}
                Any point P in the triangle's plane:
                \begin{align*}
                    \mathbf{P}(\alpha,\beta,\gamma) &= \alpha\mathbf{a} + \beta\mathbf{b} + \gamma\mathbf{c}
                \end{align*}
                where: $\alpha + \beta + \gamma = 1$

                \vspace{0.3cm}

                \pause
                \textbf{Physical Interpretation:}
                \begin{itemize}
                    \item $\alpha$, $\beta$, $\gamma$ are \textit{weights}
                    \item P is the \textit{center of mass}
                    \item Also called \textit{barycenter}
                \end{itemize}
            \end{mathbox}
        \end{column}
        \begin{column}{0.5\textwidth}
            \begin{tikzpicture}[scale=0.8]
                \begin{scope}[canvas is xy plane at z=0]
                    % Triangle
                    \coordinate (A) at (0,0);
                    \coordinate (B) at (4,0);
                    \coordinate (C) at (2,3);
                    
                    \draw[thick, ObjectColor, fill=ObjectColor!15] (A) -- (B) -- (C) -- cycle;
                    
                    % Vertices with weights
                    \fill[ObjectColor] (A) circle (3pt);
                    \fill[ObjectColor] (B) circle (3pt);
                    \fill[ObjectColor] (C) circle (3pt);
                    \node[below] at (A) {$\mathbf{a}$ ($\alpha=0.3$)};
                    \node[below] at (B) {$\mathbf{b}$ ($\beta=0.5$)};
                    \node[above] at (C) {$\mathbf{c}$ ($\gamma=0.2$)};
                    
                    % Point P
                    \coordinate (P) at (2.2,1.2);
                    \fill[AccentColor] (P) circle (3pt);
                    \node[left] at (P) {\textcolor{AccentColor}{$\mathbf{P}$}};
                    
                    % Sub-triangles for area interpretation
                    \draw[thin, SecondaryColor, dashed] (P) -- (A);
                    \draw[thin, SecondaryColor, dashed] (P) -- (B);
                    \draw[thin, SecondaryColor, dashed] (P) -- (C);
                    
                    % Area labels
                    \fill[red!15, opacity=0.5] (A) -- (B) -- (P) -- cycle;
                    \fill[green!15, opacity=0.5] (C) -- (B) -- (P) -- cycle;
                    \fill[blue!15, opacity=0.5] (A) -- (C) -- (P) -- cycle;

                    \node[red] at (2.2,0.4) {$A_{\gamma}$};
                    \node[green] at (2.6,1.7) {$A_{\alpha}$};
                    \node[blue] at (1.6,1.7) {$A_{\beta}$};


                \end{scope}
            \end{tikzpicture}
            \vspace{0.3cm}
            \only<3->{
                \begin{conceptbox}{Area Relationship}
                    $\alpha = \frac{A_{\alpha}}{A_{\text{total}}}$, $\beta = \frac{A_{\beta}}{A_{\text{total}}}$, \\
                    $\gamma = \frac{A_{\gamma}}{A_{\text{total}}}$
                \end{conceptbox}
            }
        \end{column}
    \end{columns}
    \href{https://www.desmos.com/calculator/rb2zeav9og}{Check out the Desmos demo.}
\end{frame}



\begin{frame}{Barycentric Coordinates: Inside vs Outside}
    \begin{columns}
        \begin{column}{0.5\textwidth}
            \begin{mathbox}{Triangle Interior Test}
                Point P is \textbf{inside} triangle if:
                \begin{align*}
                    \alpha, \beta, \gamma &\geq 0
                \end{align*}
                
                Since $\alpha + \beta + \gamma = 1$, we can rewrite as:
                \begin{align*}
                    \beta &\geq 0 \\
                    \gamma &\geq 0 \\
                    \alpha &\geq 0 \text{ or } \beta + \gamma \leq 1
                \end{align*}
            \end{mathbox}
        \end{column}
        \begin{column}{0.5\textwidth}
            \begin{tikzpicture}[scale=0.6]
                \begin{scope}[canvas is xy plane at z=0]
                    % Triangle
                    \coordinate (A) at (0,0);
                    \coordinate (B) at (4,0);
                    \coordinate (C) at (2,3);
                    
                    \draw[thick, ObjectColor, fill=ObjectColor!15] (A) -- (B) -- (C) -- cycle;
                    
                    % Vertices
                    \fill[ObjectColor] (A) circle (2pt);
                    \fill[ObjectColor] (B) circle (2pt);
                    \fill[ObjectColor] (C) circle (2pt);
                    \node[below] at (A) {$\mathbf{a}$};
                    \node[below] at (B) {$\mathbf{b}$};
                    \node[above] at (C) {$\mathbf{c}$};
                    
                    % Inside point
                    \fill[AccentColor] (1.5,1) circle (2pt);
                    \node[right, AccentColor] at (1.5,1) {Inside};
                    \node[right, AccentColor] at (0.6,0.3) {\scriptsize $0\geq \beta,\gamma \geq 1$};
                    
                    % Outside points
                    \fill[red] (-0.5,1) circle (2pt);
                    \node[left, red] at (-0.5,1) {Outside};
                    \node[left, red] at (-0.5,0.4) {$\beta<0$};
                    
                    \fill[red] (4,1.5) circle (2pt);
                    \node[above, red] at (5.2,1.1) {Outside};
                    \node[above, red] at (5,1.5) {$\alpha < 0, \beta+\gamma>1$};
                    
                    \fill[red] (1.5,-1.5) circle (2pt);
                    \node[right, red] at (1.5,-1.5) {Outside};
                    \node[right, red] at (1.5,-1) {$\gamma<0$};
                \end{scope}
            \end{tikzpicture}
        \end{column}
    \end{columns}
    \pause
    \begin{conceptbox}{Insight}
        Barycentric coordinates doesn't just tell us if a point is inside a triangle, but also it's position with respect to other vertices.
    \end{conceptbox}
\end{frame}

\begin{frame}{Barycentric Coordinates: Derivation}
    \begin{columns}
        \begin{column}{0.65\textwidth}
            \begin{mathbox}{Key Idea}
                \begin{itemize}
                    \item  The sides
                        $\mathbf{e_1} = \mathbf{b} - \mathbf{a}$ and $\mathbf{e_2} = \mathbf{c} - \mathbf{a}$ are linearly independent vectors on the triangle's plane.
                    \pause
                    \item Therefore, any vector in the triangle's plane (e.g. $\mathbf{P} - \mathbf{a}$) can be expressed as a linear combination of these vectors.
                    \pause
                    \item We can express $\mathbf{P}$ as:
                        \begin{align*}
                            \mathbf{P} &= \mathbf{a} + \beta(\mathbf{b}-\mathbf{a}) + \gamma(\mathbf{c}-\mathbf{a}) \\
                            &= \alpha \mathbf{a} + \beta \mathbf{b} + \gamma \mathbf{c}
                        \end{align*}
                        Where $\alpha = 1 - \beta - \gamma$.
                \end{itemize}
            \end{mathbox}
           
            

        \end{column}
        \begin{column}{0.35\textwidth}
            \begin{tikzpicture}[scale=0.7]
                \begin{scope}[canvas is xy plane at z=0]
                    % Triangle
                    \coordinate (A) at (0,0);
                    \coordinate (B) at (4,0);
                    \coordinate (C) at (2,3);
                    
                    \draw[thick, ObjectColor, fill=ObjectColor!15] (A) -- (B) -- (C) -- cycle;
                    
                    % Vertices with weights
                    \fill[ObjectColor] (A) circle (3pt);
                    \fill[ObjectColor] (B) circle (3pt);
                    \fill[ObjectColor] (C) circle (3pt);
                    \node[below] at (A) {$\mathbf{a}$};
                    \node[below] at (B) {$\mathbf{b}$};
                    \node[above] at (C) {$\mathbf{c}$};

                    \draw[->, red, very thick] (A) -- (B) node[midway, below] {$\mathbf{e_1} = \mathbf{b}-\mathbf{a}$}; 
                    \draw[->, blue, very thick] (A) -- (C) node[midway, left] {$\mathbf{e_2} = \mathbf{c}-\mathbf{a}$};
                    
                    % Point P
                    \coordinate (P) at (2.2,1.2);
                    \fill[AccentColor] (P) circle (3pt);
                    \node[right] at (P) {\textcolor{AccentColor}{$\mathbf{P}$}};
                    \draw[->, thin, PrimaryColor, dashed] (A) -- (P);
                \end{scope}
            \end{tikzpicture}
        \end{column}
    \end{columns}
\end{frame}


\begin{frame}{Barycentric Coordinates: Derivation}
    \begin{columns}
        \begin{column}{0.65\textwidth}
            \begin{mathbox}{Area Interpretation}
                \begin{itemize}
                    \small
                    \item  Let $\hat{\mathbf{u}}$ be an unit vector in the direction of the altitude towards $C$. 
                    \pause
                    \item  The height of the triangle is $h = \hat{\mathbf{u}} \cdot (\mathbf{c} - \mathbf{a})$ (projection).
                    \pause
                    \item  The height of the shaded triangle is $h' = \hat{\mathbf{u}} \cdot (\mathbf{P} - \mathbf{a})$.
                    \pause
                    \item  Hence, \begin{align*}
                        A_{\gamma} &= \frac{1}{2} \cdot h' \cdot |\mathbf{b} - \mathbf{a}| \\
                        &= \frac{1}{2} \cdot \left (\hat{\mathbf{u}} \cdot (\mathbf{P} - \mathbf{a}) \right ) \cdot |\mathbf{b} - \mathbf{a}|  \\
                        &= \frac{1}{2} \cdot \gamma \left (\hat{\mathbf{u}} \cdot (\mathbf{c} - \mathbf{a}) \right ) \cdot |\mathbf{b} - \mathbf{a}|  \\
                        &= \gamma \frac{1}{2} \cdot h \cdot |\mathbf{b} - \mathbf{a}| = \gamma A_{\text{total}}
                    \end{align*}
                \end{itemize}
            \end{mathbox}    
        \end{column}
        \begin{column}{0.35\textwidth}
            \begin{tikzpicture}[scale=0.7]
                \begin{scope}[canvas is xy plane at z=0]
                    % Triangle
                    \coordinate (A) at (0,0);
                    \coordinate (B) at (4,0);
                    \coordinate (C) at (2,3);
                    
                    \draw[thick, ObjectColor, fill=ObjectColor!15] (A) -- (B) -- (C) -- cycle;
                    
                    % Vertices with weights
                    \fill[ObjectColor] (A) circle (3pt);
                    \fill[ObjectColor] (B) circle (3pt);
                    \fill[ObjectColor] (C) circle (3pt);
                    \node[below] at (A) {$\mathbf{a}$};
                    \node[below] at (B) {$\mathbf{b}$};
                    \node[above] at (C) {$\mathbf{c}$};

                    \draw[->, red, very thick] (A) -- (B) node[midway, below] {$\mathbf{b}-\mathbf{a}$}; 
                    
                    % Point P
                    \coordinate (P) at (1.7,1.6);
                    \fill[AccentColor] (P) circle (3pt);
                    \node[left] at (P) {\textcolor{AccentColor}{$\mathbf{P}$}};
                    \fill[red!60, opacity=0.2] (A) -- (P) -- (B) -- cycle;
                    \draw[PrimaryColor, dashed, thin] (A) -- (P) -- (B);

                    \draw[->, blue, thick] (2, 0) -- (2, 1) node[midway, right] {$\hat{\mathbf{u}}$};
                    \draw[gray, thin] (2, 0) -- (2, 3) node[midway, right, yshift=0.3cm] {\small $h$};
                    \draw[gray, thin] (1.7, 0) -- (1.7, 1.6) node[midway, left, xshift=0.1cm] {\small $h'$};


                    \node[red] at (1,0.4) {$A_{\gamma}$};
                \end{scope}
            \end{tikzpicture}
            \vspace{0.3cm}
            \only<5->{
            \footnotesize
                Since, $\mathbf{P} = \mathbf{a} + \beta (\mathbf{b} - \mathbf{a}) + \gamma (\mathbf{c} - \mathbf{a})$, 
                \begin{align*}
                    \mathbf{P} - \mathbf{a} = \beta (\mathbf{b} - \mathbf{a}) + \gamma (\mathbf{c} - \mathbf{a}) \\
                    \hat{\mathbf{u}} \cdot (\mathbf{P} - \mathbf{a}) = \gamma (\hat{\mathbf{u}} \cdot (\mathbf{c} - \mathbf{a}))
                \end{align*}
                Since $\hat{\mathbf{u}}$ is perpendicular to $\mathbf{b} - \mathbf{a}$.
            }
        \end{column}
    \end{columns}
\end{frame}




\begin{frame}{Method 1: Two-Step Ray-Triangle Intersection}
    \begin{columns}
        \begin{column}{0.65\textwidth}
            \begin{mathbox}{Algorithm Steps}
                \textbf{Step 1:} Ray-Plane Intersection
                \begin{align*}
                   t = -\frac{D + \mathbf{n} \cdot \mathbf{R_o}}{\mathbf{n} \cdot \mathbf{R_d}}
                \end{align*}
                \only<2->{
                    \textbf{Step 2:} Inside/Outside Test
                    \begin{align*}
                        \mathbf{P} &= \mathbf{a} + \beta(\mathbf{b}-\mathbf{a}) + \gamma(\mathbf{c}-\mathbf{a})
                    \end{align*}
                    
                    Solve for $\beta$, $\gamma$ and check bounds.
                }
            \end{mathbox}
        \end{column}
        \begin{column}{0.35\textwidth}
            \begin{tikzpicture}[scale=0.6]
                % Step 1: Ray-plane intersection

                % Rays
                \coordinate (RayStart1) at (2.5,-1, 4);
                \draw[ray, orange] (RayStart1) -- (3.5,0);

                \begin{scope}[plane x={(0.707, 0, 0.707)}, plane y={(0, 1, 0)}, canvas is plane]
                    % Extended plane
                    \fill[blue!30, opacity=0.3] (-2,-1) rectangle (7,4);
                    \draw[blue!60, thin] (-2,-1) rectangle (7,4);
                    
                    % Triangle
                    \coordinate (A) at (0,0);
                    \coordinate (B) at (4,0);
                    \coordinate (C) at (2,3);
                    
                    \draw[thick, ObjectColor, fill=ObjectColor!20] (A) -- (B) -- (C) -- cycle;
                    
                    % Vertices
                    \fill[ObjectColor] (A) circle (2pt);
                    \fill[ObjectColor] (B) circle (2pt);
                    \fill[ObjectColor] (C) circle (2pt);
                    
                    % Plane intersection point (outside triangle)
                    \coordinate (PlaneHit) at (5,0.05);
                    \fill[orange] (PlaneHit) circle (3pt);
                    \node[right, orange] at (PlaneHit) {\small Step 1: Plane hit};
                    
                    \pause
                    % Triangle intersection point
                    \coordinate (TriHit) at (2,1);
                    \fill[AccentColor] (TriHit) circle (3pt);
                    \draw[ray, AccentColor] (PlaneHit) -- (TriHit);
                    \node[right, AccentColor, xshift=0.3cm] at (TriHit) {\small Step 2: Triangle hit};
                \end{scope}
                
            
            \end{tikzpicture}
        \end{column}
    \end{columns}
\end{frame}



\begin{frame}{Method 2: Direct Barycentric Intersection}
    \begin{columns}
        \begin{column}{0.65\textwidth}
            \begin{mathbox}{Direct Approach}
                \small
                Set ray equation equal to barycentric form:
                \begin{align*}
                    \mathbf{R_o} + t\mathbf{R_d} &= \mathbf{a} + \beta(\mathbf{b}-\mathbf{a}) + \gamma(\mathbf{c}-\mathbf{a})
                \end{align*}
                
                Rearrange to linear system:
                \begin{align*}
                    \begin{bmatrix}
                        -\mathbf{R_d} & (\mathbf{b}-\mathbf{a}) & (\mathbf{c}-\mathbf{a})
                    \end{bmatrix}
                    \begin{bmatrix}
                        t \\ \beta \\ \gamma
                    \end{bmatrix}
                    = \mathbf{R_o} - \mathbf{a}
                \end{align*}
                
                Solve using Cramer's rule or \\ LU decomposition.
            \end{mathbox}
        \end{column}
        \begin{column}{0.35\textwidth}
                \begin{tikzpicture}[scale=0.6]
                \begin{scope}[plane x={(0.707, 0, 0.707)}, plane y={(0, 1, 0)}, canvas is plane]
                    % Extended plane
                    \fill[blue!30, opacity=0.3] (-2,-1) rectangle (7,4);
                    \draw[blue!60, thin] (-2,-1) rectangle (7,4);
                    \coordinate (A) at (0,0);
                    \coordinate (B) at (4,0);
                    \coordinate (C) at (2,3);
                    \draw[thick, ObjectColor, fill=ObjectColor!20] (A) -- (B) -- (C) -- cycle;
                    
                    \fill[ObjectColor] (A) circle (2pt);
                    \fill[ObjectColor] (B) circle (2pt);
                    \fill[ObjectColor] (C) circle (2pt);
                    
                    \coordinate (PlaneHit) at (2.5,0.75);
                    \fill[orange] (PlaneHit) circle (3pt);
                    
                \end{scope}
                \coordinate (RayStart1) at (1,0, 4);
                \draw[ray, orange] (RayStart1) -- (2,1);
                           
            \end{tikzpicture}
        \end{column}
    \end{columns}
\end{frame}

\begin{frame}{Cramer's Rule Solution}
    \begin{mathbox}{Matrix Form}
        \vspace{-0.5cm}
        \small
        \begin{align*}
            \underbrace{
                \begin{bmatrix}
                    -R_{dx} & b_x-a_x & c_x-a_x \\
                    -R_{dy} & b_y-a_y & c_y-a_y \\
                    -R_{dz} & b_z-a_z & c_z-a_z
                \end{bmatrix}
            }_{A}
            \begin{bmatrix}
                t \\ \beta \\ \gamma
            \end{bmatrix}
            =
            \begin{bmatrix}
                R_{ox} - a_x \\
                R_{oy} - a_y \\
                R_{oz} - a_z
            \end{bmatrix}
        \end{align*}
    \end{mathbox}
    \vspace{-0.1cm}
    \pause
    \begin{columns}
        \begin{column}{0.6\textwidth}
            \begin{mathbox}{Cramer's Rule}
                \vspace{-0.3cm}
                \footnotesize
                \begin{align*}
                    t &= \frac{1}{|A|} \begin{vmatrix}
                        R_{ox}-a_x & b_x-a_x & c_x-a_x \\
                        R_{oy}-a_y & b_y-a_y & c_y-a_y \\
                        R_{oz}-a_z & b_z-a_z & c_z-a_z
                    \end{vmatrix} \\
                    \beta &= \frac{1}{|A|} \begin{vmatrix}
                        -R_{dx} & R_{ox}-a_x & c_x-a_x \\
                        -R_{dy} & R_{oy}-a_y & c_y-a_y \\
                        -R_{dz} & R_{oz}-a_z & c_z-a_z
                    \end{vmatrix} \\
                    \gamma &= \frac{1}{|A|} \begin{vmatrix}
                        -R_{dx} & b_x-a_x & R_{ox}-a_x \\
                        -R_{dy} & b_y-a_y & R_{oy}-a_y \\
                        -R_{dz} & b_z-a_z & R_{oz}-a_z
                    \end{vmatrix}
                \end{align*}
            \end{mathbox}
        \end{column}
        \pause
        \begin{column}{0.4\textwidth}
            \begin{conceptbox}{Checks}
                \footnotesize
                \begin{itemize}
                    \item $t_{\text{\text{min}}} < t < t_{\text{\text{current}}}$ (valid intersection)
                    \item $\beta, \gamma \geq 0$ and $\beta + \gamma \leq 1$ \\ (inside triangle)
                \end{itemize}
            \end{conceptbox}
        \end{column}
    \end{columns}
\end{frame}

%% optional
{
    \setbeamercolor{background canvas}{bg=PrimaryColor!30, fg=black}
    \begin{frame}{Cramer's Rule Intuition}
        \begin{columns}
            \begin{column}{0.65\textwidth}
                \begin{mathbox}{Determinant Interpretation}
                    Remember, $\beta = \frac{A_{\beta}}{A_{\text{total}}}$? \\

                    Determinants measure areas of projected triangles.
                    \begin{align*}
                        \begin{vmatrix}
                            -R_{dx} & (b_x-a_x) & (c_x-a_x) \\
                            -R_{dy} & (b_y-a_y) & (c_y-a_y) \\
                            -R_{dz} & (b_z-a_z) & (c_z-a_z)
                        \end{vmatrix} &\propto A_{\text{total}} \\
                        \begin{vmatrix}
                            -R_{dx} & R_ox-a_x & c_x-a_x \\
                            -R_{dy} & R_oy-a_y & c_y-a_y \\
                            -R_{dz} & R_oz-a_z & c_z-a_z
                        \end{vmatrix} &\propto A_{\beta}
                    \end{align*}
                    
                    \textbf{Why?}
                \end{mathbox}
                
            \end{column}
            \begin{column}{0.35\textwidth}
                \centering
                \begin{tikzpicture}[scale=0.6]
                    \begin{scope}[canvas is xy plane at z=0]
                        % Triangle
                        \coordinate (A) at (0,0);
                        \coordinate (B) at (4,0);
                        \coordinate (C) at (2,3);
                        
                        \draw[thick, ObjectColor, fill=ObjectColor!15] (A) -- (B) -- (C) -- cycle;
                        
                        % Vertices with weights
                        \fill[ObjectColor] (A) circle (3pt);
                        \fill[ObjectColor] (B) circle (3pt);
                        \fill[ObjectColor] (C) circle (3pt);
                        \node[below] at (A) {$\mathbf{a}$};
                        \node[below] at (B) {$\mathbf{b}$};
                        \node[above] at (C) {$\mathbf{c}$};
                        
                        % Point P
                        \coordinate (P) at (2.2,1.2);
                        \fill[AccentColor] (P) circle (3pt);
                        \node[below] at (P) {\textcolor{AccentColor}{$\mathbf{P}$}};
                        
                        % Sub-triangles for area interpretation
                        \draw[thin, SecondaryColor, dashed] (P) -- (A);
                        \draw[thin, SecondaryColor, dashed] (P) -- (B);
                        \draw[thin, SecondaryColor, dashed] (P) -- (C);
                        
                        % Area labels
                        \fill[blue!25, opacity=0.5] (A) -- (C) -- (P) -- cycle;

                        \node[blue] at (1.6,1.7) {$A_{\beta}$};


                    \end{scope}
                \end{tikzpicture}
            \end{column}
        \end{columns}
    \end{frame}

        \begin{frame}{Cramer's Rule Intuition}
        \begin{columns}
            \begin{column}{0.65\textwidth}
                \begin{mathbox}{Determinant Interpretation}
                    \only<1>{
                        The areas can be found by cross products of vectors. Consider |
                        \begin{align*}
                            \mathbf{N} &= \mathbf{(\mathbf{b} - \mathbf{a})} \times \mathbf{(\mathbf{c} - \mathbf{a})} \\
                            &= \begin{vmatrix}
                                \mathbf{i} & \mathbf{j} & \mathbf{k} \\
                                b_x - a_x & b_y - a_y & b_z - a_z \\
                                c_x - a_x & c_y - a_y & c_z - a_z
                            \end{vmatrix} \\
                            &= 2 A_{\text{total}} \hat{\mathbf{n}} 
                        \end{align*}
                        $\mathbf{N}$ is the normal vector of the triangle scaled by twice the area.
                    }
                    \only<2>{
                        \begin{itemize}
                            \small
                            \item We project the triangle onto the plane perpendicular to the ray direction $\mathbf{R_d}$. 
                            \item Which we can do if you just multiply the original area with the cosine of the angle between the plane of the triangle and the ray direction.
                            \begin{align*}
                                A'_{\text{total}} = A_{\text{total}} \cdot \cos(\theta)
                            \end{align*}
                            Where $\theta$ is the angle between the triangle's plane and the ray direction.
                            \item $\theta$ is also the angle between the normal vector $\mathbf{N}$ and the opposite of the ray direction $-\mathbf{R}_d$. So we can write:
                            \begin{align*}
                                A'_{\text{total}} = \frac{1}{2} \mathbf{N} \cdot (-\mathbf{R}_d)
                            \end{align*}
                        \end{itemize}
                    }
                    \only<3>{
                        Therefore, the first determinant in Cramer's can be interpreted as follows:                    
                        \begin{align*}
                            \mathbf{N} \cdot (-\mathbf{R}_d)
                            &= \begin{vmatrix}
                                -R_x & -R_y & -R_z \\
                                b_x - a_x & b_y - a_y & b_z - a_z \\
                                c_x - a_x & c_y - a_y & c_z - a_z
                            \end{vmatrix} \\
                            &= \begin{vmatrix}
                                -R_x & b_x - a_x & c_x - a_x \\
                                -R_y & b_y - a_y & c_y - a_y \\
                                -R_z & b_z - a_z & c_z - a_z
                            \end{vmatrix} \\
                            &= 2 A'_{\text{total}} \propto A_{\text{total}} 
                        \end{align*}
                    }
                    \only<4>{
                        Now what about the $A_\gamma$ triangle? \\
                        We also find $A'_{\gamma}$ by projecting the triangle $A_\gamma$ onto the plane perpendicular to the ray direction $\mathbf{R}_d$.
                    }
                    \only<5>{
                        \begin{itemize}
                            \item Instead of projecting the triangle $\triangle (\mathbf{a}, \mathbf{c}, \mathbf{P})$,
                            we project the triangle $\triangle (\mathbf{a}, \mathbf{c}, \mathbf{R}_o)$.
                            \item From the perspective of the ray,
                        points $\mathbf{P}$ and $\mathbf{R}_o$ are the same point, as both fall on the ray.
                            \item So, both triangles have the same the same area $A'_\gamma$ when projected to the plane perpendicular to the ray direction $\mathbf{R}_d$.
                        \end{itemize}
                    }
                    \only<6>{
                        The projected area of the triangle can again be found using cross product followed by the dot product with the ray direction.
                        \begin{align*}
                            & (\mathbf{R}_o - \mathbf{a}) \times (\mathbf{c} - \mathbf{a}) \cdot (-\mathbf{R}_d) \\
                            =& \begin{vmatrix}
                                -R_x & -R_y & -R_z \\
                                R_{ox} - a_x & R_{oy} - a_y & R_{oz} - a_z \\
                                c_x - a_x & c_y - a_y & c_z - a_z
                            \end{vmatrix} \\
                            =& \begin{vmatrix}
                                -R_x & R_{ox} - a_x & c_x - a_x \\
                                -R_y & R_{oy} - a_y & c_y - a_y \\
                                -R_z & R_{oz} - a_z & c_z - a_z
                            \end{vmatrix} \\
                            =& 2 A'_\gamma \propto A_{\gamma}
                        \end{align*}
                    }
                \end{mathbox}
                
            \end{column}
            \begin{column}{0.35\textwidth}
                \centering
                \begin{tikzpicture}[scale=0.5]
                    % Point P
                    \coordinate (P) at (2.2,1.2,0);
                    \coordinate (Ro) at ($(2.2,1.2,0)-8*(-0.707,0,-0.707)$);
                    \coordinate (Rd) at ($(2.2,1.2,0)-6*(-0.707,0,-0.707)$);
                    \coordinate (Rd2) at ($(2.2,1.2,0)-2*(-0.707,0,-0.707)$);
                    \coordinate (Rf) at ($(2.2,1.2,0)+8*(-0.707,0,-0.707)$);
                    \begin{scope}[canvas is xy plane at z=0]
                        % Triangle
                        \coordinate (A) at (0,0);
                        \coordinate (B) at (4,0);
                        \coordinate (C) at (2,3);
                        
                        \draw[thick, ObjectColor, fill=ObjectColor!15] (A) -- (B) -- (C) -- cycle;
                        
                        % Vertices with weights
                        \fill[ObjectColor] (A) circle (3pt);
                        \fill[ObjectColor] (B) circle (3pt);
                        \fill[ObjectColor] (C) circle (3pt);
                        \node[below] at (A) {$\mathbf{a}$};
                        \node[below] at (B) {$\mathbf{b}$};
                        \node[above] at (C) {$\mathbf{c}$};
                        
                        
                        % Sub-triangles for area interpretation
                        \draw[thin, SecondaryColor, dashed] (P) -- (A);
                        \draw[thin, SecondaryColor, dashed] (P) -- (B);
                        \draw[thin, SecondaryColor, dashed] (P) -- (C);
                        
                        % Area labels
                        \fill[blue!25, opacity=0.5] (A) -- (C) -- (P) -- cycle;

                        \node[blue] at (1.6,1.7) {$A_{\beta}$};
                    \end{scope}

                    \fill[AccentColor] (P) circle (3pt);
                    \node[below] at (P) {\textcolor{AccentColor}{$\mathbf{P}$}};
                    \draw[blue, ->] (P) --++ (0, 0, 6) node[right] {$\mathbf{N}$};
                    \only<5->{
                        \draw[thick, green, fill=green!30, opacity=0.5] (Ro) -- (A) -- (C) -- cycle;
                    }
                    \only<4->{
                        \fill[blue!50, opacity=0.5] (A) -- (C) -- (P) -- cycle;
                        \node[blue] at (1.6,1.7) {$A_{\beta}$};
                    }
                    \only<2->{    
                        \begin{scope}[
                            plane origin={(2.2, 1.2, 0)},
                            plane x={(2.907, 1.2, -0.707)},
                            plane y={(2.2, 2.2, 0)},
                            canvas is plane    
                        ]
                            \draw[PrimaryColor!60, opacity=0.8] (-3,-3) rectangle (3,3);
                            \fill[PrimaryColor!60, opacity=0.5] (-3,-3) rectangle (3,3);

                            \coordinate (A1) at (-1.556,-1.2);
                            \coordinate (B1) at (1.273,-1.2);
                            \coordinate (C1) at (-0.141,1.8);
                            \only<3,6->{
                                \draw[thick, red, fill=red!15, opacity=0.3] (A1) -- (B1) -- (C1) -- cycle;
                            }
                            \only<3>{
                                \node[red, right] at (P) {$A'_{\text{total}}$};
                            }
                            \only<4,6->{
                                \draw[thick, orange, fill=orange!80, opacity=0.3] (P) -- (A1) -- (C1) -- cycle;
                            }
                            \only<4>{
                                \node[orange, right] at (P) {$A'_{\gamma}$};
                            }
                        \end{scope}
                        \draw[PrimaryColor,fill=PrimaryColor] (Ro) circle (2pt) node[below] {$\mathbf{R_o}$};
                        \draw[ray] (Ro) -- (Rf);
                        \draw[->, red, thick] (Ro) -- (Rd) node[midway, above] {$\mathbf{R_d}$};
                        \only<2>{
                            \draw[->, red, thick] (P) -- (Rd2) node[midway, above right] {$-\mathbf{R_d}$};
                        }
                    }
                \end{tikzpicture}
            \end{column}
        \end{columns}
    \end{frame}
}


\begin{frame}{Bonus of Using Barycentric Coordinates}
    \begin{columns}
        \begin{column}{0.5\textwidth}
            \begin{conceptbox}{Advantages}
                \begin{itemize}
                    \item Efficient to compute
                    \item Get Barycentric coordinates for free
                    \item Enables interpolation of vertex attributes\\
                    Used in |
                    \begin{itemize}
                        \item Textures
                        \item Normals
                        \item Colors
                    \end{itemize}
                \end{itemize}
            \end{conceptbox}
        \end{column}
        \begin{column}{0.5\textwidth}
            \begin{tikzpicture}[scale=0.6]
            \coordinate (A) at (90:4);    % top
            \coordinate (B) at (210:4);   % left
            \coordinate (C) at (-30:4);   % right
            \fill[green] (A) -- (B) -- (C) -- cycle;
            \fill[blue,
                    path fading=west      % fades toward the left
                ]
                (A) -- (B) -- (C) -- cycle;
            \fill[red,
                    path fading=south      % fades toward the bottom
                ]
                (A) -- (B) -- (C) -- cycle;
            \draw[black, thick] (A) -- (B) -- (C) -- cycle;
            \fill[red]   (A) circle (3pt) node[above]      {$A$};
            \fill[green] (B) circle (3pt) node[below left] {$B$};
            \fill[blue]  (C) circle (3pt) node[below right]{$C$};
            \coordinate (P) at (0, 0);
            \fill[white] (P) circle (2pt) node[right] {$P$};
            \node[below, white, yshift=-0.2cm] at (P) {\small $\alpha R + \beta G + \gamma B$};
            \end{tikzpicture}
        \end{column}
    \end{columns}
\end{frame}
