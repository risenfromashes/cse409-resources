
\begin{frame}{What Are Barycentric Coordinates?}
  \begin{columns}
    \begin{column}{0.5\textwidth}
      \begin{mathbox}{Barycentric Definition}
        Any point P in the triangle's plane:
        \begin{align*}
          \mathbf{P}(\alpha,\beta,\gamma) &= \alpha\mathbf{a} + \beta\mathbf{b} + \gamma\mathbf{c}
        \end{align*}
        where: $\alpha + \beta + \gamma = 1$

        \vspace{0.3cm}

        \pause
        \textbf{Physical Interpretation:}
        \begin{itemize}
          \item $\alpha$, $\beta$, $\gamma$ are \textit{weights}
          \item P is the \textit{center of mass}
          \item Also called \textit{barycenter}
        \end{itemize}
      \end{mathbox}
    \end{column}
    \begin{column}{0.5\textwidth}
      \begin{tikzpicture}[scale=0.8]
        \begin{scope}[canvas is xy plane at z=0]
          % Triangle
          \coordinate (A) at (0,0);
          \coordinate (B) at (4,0);
          \coordinate (C) at (2,3);

          \draw[thick, ObjectColor, fill=ObjectColor!15] (A) -- (B) -- (C) -- cycle;

          % Vertices with weights
          \fill[ObjectColor] (A) circle (3pt);
          \fill[ObjectColor] (B) circle (3pt);
          \fill[ObjectColor] (C) circle (3pt);
          \node[below] at (A) {$\mathbf{a}$ ($\alpha=0.3$)};
          \node[below] at (B) {$\mathbf{b}$ ($\beta=0.5$)};
          \node[above] at (C) {$\mathbf{c}$ ($\gamma=0.2$)};

          % Point P
          \coordinate (P) at (2.2,1.2);
          \fill[AccentColor] (P) circle (3pt);
          \node[left] at (P) {\textcolor{AccentColor}{$\mathbf{P}$}};

          % Sub-triangles for area interpretation
          \draw[thin, SecondaryColor, dashed] (P) -- (A);
          \draw[thin, SecondaryColor, dashed] (P) -- (B);
          \draw[thin, SecondaryColor, dashed] (P) -- (C);

          % Area labels
          \fill[red!15, opacity=0.5] (A) -- (B) -- (P) -- cycle;
          \fill[green!15, opacity=0.5] (C) -- (B) -- (P) -- cycle;
          \fill[blue!15, opacity=0.5] (A) -- (C) -- (P) -- cycle;

          \node[red] at (2.2,0.4) {$A_{\gamma}$};
          \node[green] at (2.6,1.7) {$A_{\alpha}$};
          \node[blue] at (1.6,1.7) {$A_{\beta}$};

        \end{scope}
      \end{tikzpicture}
      \vspace{0.3cm}
      \only<3->{
        \begin{conceptbox}{Area Relationship}
          $\alpha = \frac{A_{\alpha}}{A_{\text{total}}}$, $\beta = \frac{A_{\beta}}{A_{\text{total}}}$, \\
          $\gamma = \frac{A_{\gamma}}{A_{\text{total}}}$
        \end{conceptbox}
      }
    \end{column}
  \end{columns}
  \href{https://www.desmos.com/calculator/rb2zeav9og}{Check out the Desmos demo.}
\end{frame}

\begin{frame}{Barycentric Coordinates: Inside vs Outside}
  \begin{columns}
    \begin{column}{0.5\textwidth}
      \begin{mathbox}{Triangle Interior Test}
        Point P is \textbf{inside} triangle if:
        \begin{align*}
          \alpha, \beta, \gamma &\geq 0
        \end{align*}

        Since $\alpha + \beta + \gamma = 1$, we can rewrite as:
        \begin{align*}
          \beta &\geq 0 \\
          \gamma &\geq 0 \\
          \alpha &\geq 0 \text{ or } \beta + \gamma \leq 1
        \end{align*}
      \end{mathbox}
    \end{column}
    \begin{column}{0.5\textwidth}
      \begin{tikzpicture}[scale=0.6]
        \begin{scope}[canvas is xy plane at z=0]
          % Triangle
          \coordinate (A) at (0,0);
          \coordinate (B) at (4,0);
          \coordinate (C) at (2,3);

          \draw[thick, ObjectColor, fill=ObjectColor!15] (A) -- (B) -- (C) -- cycle;

          % Vertices
          \fill[ObjectColor] (A) circle (2pt);
          \fill[ObjectColor] (B) circle (2pt);
          \fill[ObjectColor] (C) circle (2pt);
          \node[below] at (A) {$\mathbf{a}$};
          \node[below] at (B) {$\mathbf{b}$};
          \node[above] at (C) {$\mathbf{c}$};

          % Inside point
          \fill[AccentColor] (1.5,1) circle (2pt);
          \node[right, AccentColor] at (1.5,1) {Inside};
          \node[right, AccentColor] at (0.6,0.3) {\scriptsize $0\leq \beta,\gamma \leq 1$};

          % Outside points
          \fill[red] (-0.5,1) circle (2pt);
          \node[left, red] at (-0.5,1) {Outside};
          \node[left, red] at (-0.5,0.4) {$\beta<0$};

          \fill[red] (4,1.5) circle (2pt);
          \node[above, red] at (5.2,1.1) {Outside};
          \node[above, red] at (5,1.5) {$\alpha < 0, \beta+\gamma>1$};

          \fill[red] (1.5,-1.5) circle (2pt);
          \node[right, red] at (1.5,-1.5) {Outside};
          \node[right, red] at (1.5,-1) {$\gamma<0$};
        \end{scope}
      \end{tikzpicture}
    \end{column}
  \end{columns}
  \pause
  \begin{conceptbox}{Insight}
    Barycentric coordinates doesn't just tell us if a point is inside a triangle, but also it's position with respect to other vertices.
  \end{conceptbox}
\end{frame}

\begin{frame}{Barycentric Coordinates: Derivation}
  \begin{columns}
    \begin{column}{0.65\textwidth}
      \begin{mathbox}{Key Idea}
        \begin{itemize}
          \item  The sides
            $\mathbf{e_1} = \mathbf{b} - \mathbf{a}$ and $\mathbf{e_2} = \mathbf{c} - \mathbf{a}$ are linearly independent vectors on the triangle's plane.
            \pause
          \item Therefore, any vector in the triangle's plane (e.g. $\mathbf{P} - \mathbf{a}$) can be expressed as a linear combination of these vectors.
            \pause
          \item We can express $\mathbf{P}$ as:
            \begin{align*}
              \mathbf{P} &= \mathbf{a} + \beta(\mathbf{b}-\mathbf{a}) + \gamma(\mathbf{c}-\mathbf{a}) \\
              &= \alpha \mathbf{a} + \beta \mathbf{b} + \gamma \mathbf{c}
            \end{align*}
            Where $\alpha = 1 - \beta - \gamma$.
        \end{itemize}
      \end{mathbox}



    \end{column}
    \begin{column}{0.35\textwidth}
      \begin{tikzpicture}[scale=0.7]
        \begin{scope}[canvas is xy plane at z=0]
          % Triangle
          \coordinate (A) at (0,0);
          \coordinate (B) at (4,0);
          \coordinate (C) at (2,3);

          \draw[thick, ObjectColor, fill=ObjectColor!15] (A) -- (B) -- (C) -- cycle;

          % Vertices with weights
          \fill[ObjectColor] (A) circle (3pt);
          \fill[ObjectColor] (B) circle (3pt);
          \fill[ObjectColor] (C) circle (3pt);
          \node[below] at (A) {$\mathbf{a}$};
          \node[below] at (B) {$\mathbf{b}$};
          \node[above] at (C) {$\mathbf{c}$};

          \draw[->, red, very thick] (A) -- (B) node[midway, below] {$\mathbf{e_1} = \mathbf{b}-\mathbf{a}$};
          \draw[->, blue, very thick] (A) -- (C) node[midway, left] {$\mathbf{e_2} = \mathbf{c}-\mathbf{a}$};

          % Point P
          \coordinate (P) at (2.2,1.2);
          \fill[AccentColor] (P) circle (3pt);
          \node[right] at (P) {\textcolor{AccentColor}{$\mathbf{P}$}};
          \draw[->, thin, PrimaryColor, dashed] (A) -- (P);
        \end{scope}
      \end{tikzpicture}
    \end{column}
  \end{columns}
\end{frame}

\begin{frame}{Barycentric Coordinates: Derivation}
  \begin{columns}
    \begin{column}{0.65\textwidth}
      \begin{mathbox}{Area Interpretation}
        \begin{itemize}
            \small
          \item  Let $\hat{\mathbf{u}}$ be an unit vector in the direction of the altitude towards $C$.
            \pause
          \item  The height of the triangle is $h = \hat{\mathbf{u}} \cdot (\mathbf{c} - \mathbf{a})$ (projection).
            \pause
          \item  The height of the shaded triangle is $h' = \hat{\mathbf{u}} \cdot (\mathbf{P} - \mathbf{a})$.
            \pause
          \item  Hence,
            \begin{align*}
              A_{\gamma} &= \frac{1}{2} \cdot h' \cdot |\mathbf{b} - \mathbf{a}| \\
              &= \frac{1}{2} \cdot \left (\hat{\mathbf{u}} \cdot (\mathbf{P} - \mathbf{a}) \right ) \cdot |\mathbf{b} - \mathbf{a}|  \\
              &= \frac{1}{2} \cdot \gamma \left (\hat{\mathbf{u}} \cdot (\mathbf{c} - \mathbf{a}) \right ) \cdot |\mathbf{b} - \mathbf{a}|  \\
              &= \gamma \frac{1}{2} \cdot h \cdot |\mathbf{b} - \mathbf{a}| = \gamma A_{\text{total}}
            \end{align*}
        \end{itemize}
      \end{mathbox}
    \end{column}
    \begin{column}{0.35\textwidth}
      \begin{tikzpicture}[scale=0.7]
        \begin{scope}[canvas is xy plane at z=0]
          % Triangle
          \coordinate (A) at (0,0);
          \coordinate (B) at (4,0);
          \coordinate (C) at (2,3);

          \draw[thick, ObjectColor, fill=ObjectColor!15] (A) -- (B) -- (C) -- cycle;

          % Vertices with weights
          \fill[ObjectColor] (A) circle (3pt);
          \fill[ObjectColor] (B) circle (3pt);
          \fill[ObjectColor] (C) circle (3pt);
          \node[below] at (A) {$\mathbf{a}$};
          \node[below] at (B) {$\mathbf{b}$};
          \node[above] at (C) {$\mathbf{c}$};

          \draw[->, red, very thick] (A) -- (B) node[midway, below] {$\mathbf{b}-\mathbf{a}$};

          % Point P
          \coordinate (P) at (1.7,1.6);
          \fill[AccentColor] (P) circle (3pt);
          \node[left] at (P) {\textcolor{AccentColor}{$\mathbf{P}$}};
          \fill[red!60, opacity=0.2] (A) -- (P) -- (B) -- cycle;
          \draw[PrimaryColor, dashed, thin] (A) -- (P) -- (B);

          \draw[->, blue, thick] (2, 0) -- (2, 1) node[midway, right] {$\hat{\mathbf{u}}$};
          \draw[gray, thin] (2, 0) -- (2, 3) node[midway, right, yshift=0.3cm] {\small $h$};
          \draw[gray, thin] (1.7, 0) -- (1.7, 1.6) node[midway, left, xshift=0.1cm] {\small $h'$};

          \node[red] at (1,0.4) {$A_{\gamma}$};
        \end{scope}
      \end{tikzpicture}
      \vspace{0.3cm}
      \only<5->{
        \footnotesize
        Since, $\mathbf{P} = \mathbf{a} + \beta (\mathbf{b} - \mathbf{a}) + \gamma (\mathbf{c} - \mathbf{a})$,
        \begin{align*}
          \mathbf{P} - \mathbf{a} = \beta (\mathbf{b} - \mathbf{a}) + \gamma (\mathbf{c} - \mathbf{a}) \\
          \hat{\mathbf{u}} \cdot (\mathbf{P} - \mathbf{a}) = \gamma (\hat{\mathbf{u}} \cdot (\mathbf{c} - \mathbf{a}))
        \end{align*}
        Since $\hat{\mathbf{u}}$ is perpendicular to $\mathbf{b} - \mathbf{a}$.
      }
    \end{column}
  \end{columns}
\end{frame}
