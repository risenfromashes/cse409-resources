\subsubsection{Ray-AABB Intersection}
\begin{frame}{Ray-AABB Intersection: Overview}
    \begin{columns}
        \begin{column}{0.65\textwidth}
            \begin{conceptbox}{Axis-Aligned Bounding Box}
                A simple 3D box or rectangle aligned with the coordinate axes.
                \begin{itemize}
                    \item The sides are parallel to the axes, that's why it's called \textbf{axis-aligned}.
                    \item Usually used to enclose complex objects, which is why it's called a \textbf{bounding box}.
                    \item Very efficient for intersection tests. \\
                          (Just test 6 planes)
                \end{itemize}
            \end{conceptbox}
        \end{column}
        \begin{column}{0.35\textwidth}
            \begin{tikzpicture}[->, scale=0.7]
                \draw[->] (0,0) -- (3,0) node[right] {$x$};
                \draw[->] (0,0) -- (0,3) node[above] {$y$};
                \draw[->] (0,0) -- (-1.2,-1.2) node[below left] {$z$};

                \begin{scope}[canvas is zx plane at y=2]
                    \draw[green, fill=green!20, opacity=0.6] (2,2) rectangle (4,4);
                \end{scope}
                \begin{scope}[canvas is xy plane at z=2]
                    \draw[red, fill=red!20, opacity=0.6] (2,2) rectangle (4,4);
                \end{scope}
                \begin{scope}[canvas is yz plane at x=2]
                    \draw[blue, fill=blue!20, opacity=0.6] (2,2) rectangle (4,4);
                \end{scope}
                \begin{scope}[canvas is xy plane at z=4]
                    \draw[red, fill=red!20, opacity=0.6] (2,2) rectangle (4,4);
                \end{scope}
                \begin{scope}[canvas is yz plane at x=4]
                    \draw[blue, fill=blue!20, opacity=0.6] (2,2) rectangle (4,4);
                \end{scope}
                \begin{scope}[canvas is zx plane at y=4]
                    \draw[green, fill=green!20, opacity=0.6] (2,2) rectangle (4,4);
                \end{scope}
            \end{tikzpicture}
        \end{column}
    \end{columns}
\end{frame}

\begin{frame}{AABB Mathematical Representation}
    \begin{columns}
        \begin{column}{0.6\textwidth}
            \begin{mathbox}{AABB Representation}
                \small
                \begin{align*}
                    \text{AABB} = \left\{ 
                    (x, y, z) \,\middle|\,
                    \begin{array}{c}
                    x_{\min} \leq x \leq x_{\max} \\
                    y_{\min} \leq y \leq y_{\max} \\
                    z_{\min} \leq z \leq z_{\max}
                    \end{array}
                    \right\}
                \end{align*}
                
                We can store,
                \begin{align*}
                    \mathbf{P}_{\text{min}} &= (x_{\text{min}}, y_{\text{min}}, z_{\text{min}}) \\
                    \mathbf{P}_{\text{max}} &= (x_{\text{max}}, y_{\text{max}}, z_{\text{max}}) \\
                \end{align*} 
            \end{mathbox}
        \end{column}
        \begin{column}{0.4\textwidth}
            \begin{tikzpicture}[->, scale=0.7]
                \draw[->] (0,0) -- (3,0) node[right] {$x$};
                \draw[->] (0,0) -- (0,3) node[above] {$y$};
                \draw[->] (0,0) -- (-1.2,-1.2) node[below left] {$z$};

                \begin{scope}[canvas is zx plane at y=2]
                    \draw[green, fill=green!20, opacity=0.6] (2,2) rectangle (4,4);
                \end{scope}
                \begin{scope}[canvas is xy plane at z=2]
                    \draw[red, fill=red!20, opacity=0.6] (2,2) rectangle (4,4);
                \end{scope}
                \begin{scope}[canvas is yz plane at x=2]
                    \draw[blue, fill=blue!20, opacity=0.6] (2,2) rectangle (4,4);
                \end{scope}
                \begin{scope}[canvas is xy plane at z=4]
                    \draw[red, fill=red!20, opacity=0.6] (2,2) rectangle (4,4);
                \end{scope}
                \begin{scope}[canvas is yz plane at x=4]
                    \draw[blue, fill=blue!20, opacity=0.6] (2,2) rectangle (4,4);
                \end{scope}
                \begin{scope}[canvas is zx plane at y=4]
                    \draw[green, fill=green!20, opacity=0.6] (2,2) rectangle (4,4);
                \end{scope}
                \draw[PrimaryColor!80, fill=PrimaryColor!80] (2,2,2) circle (2pt) node[below right] {$\mathbf{P}_{\text{min}}$};
                \draw[PrimaryColor!80, fill=PrimaryColor!80] (4,4,4) circle (2pt) node[above left] {$\mathbf{P}_{\text{max}}$};
            \end{tikzpicture}
        \end{column}
    \end{columns}
\end{frame}

\begin{frame}{Ray-AABB Intersection: Slab Method}
    \begin{columns}
        \begin{column}{0.55\textwidth}
            \begin{mathbox}{Approach}
                \only<1-2>{
                    \textbf{Step 1:} Consider $z$ axis first. There are two planes:
                    \alt<2>{
                        \begin{align*}
                            \underbrace{z}_{\mathbf{n}=(0,0,1)} \quad &\underbrace{- z_{\text{min}}}_{D=-z_{\text{min}}} &= 0 \\
                            z \quad \quad \  &- z_{\text{max}} &= 0
                        \end{align*}
                    }{
                        \begin{align*}
                            z &= z_{\text{min}} \\
                            z &= z_{\text{max}}
                        \end{align*}
                    }
                }
                \only<3>{
                    \textbf{Step 2:} Compute intersection with ray:
                    \begin{align*}
                        t_{\text{min},z} &= - \frac{D + \mathbf{n}\cdot \mathbf{R_o}}{\mathbf{n}\cdot \mathbf{R_d}} \\
                                &= - \frac{-z_{\text{min}} + R_{oz}}{R_{dz}} \\
                        t_{\text{min},z} &= \frac{z_{\text{min}} - R_{oz}}{R_{dz}} 
                    \end{align*}
                    Similarly,
                    \begin{align*}
                        t_{\text{max},z} &= \frac{z_{\text{max}} - R_{oz}}{R_{dz}}
                    \end{align*}
                }
                \only<4>{
                    \textbf{Step 3:} Similarly for $y$ and $z$ axes:
                    \begin{align*}
                        t_{\text{min},x} &= \frac{x_{\text{min}} - R_{ox}}{R_{dx}} \\
                        t_{\text{max},x} &= \frac{x_{\text{max}} - R_{ox}}{R_{dx}} \\
                        t_{\text{min},y} &= \frac{y_{\text{min}} - R_{oy}}{R_{dy}} \\
                        t_{\text{max},y} &= \frac{y_{\text{max}} - R_{oy}}{R_{dy}} 
                    \end{align*}
                }

                \only<5>{
                    \textbf{Step 4:} Find the overlap of the intervals:
                    \begin{align*}
                        t_{\text{enter}} &= \max(t_{\text{min},x}, t_{\text{min},y}, t_{\text{min},z}) \\
                        t_{\text{exit}} &= \min(t_{\text{max},x}, t_{\text{max},y}, t_{\text{max},z})
                    \end{align*}
                    \begin{tikzpicture}
                        \draw[<->, PrimaryColor, thick] (-2.5,0) -- (2.5,0) node[midway, above] {$t$};
                        \draw[<->, PrimaryColor, thick] (-2.5,-0.5) -- (2.5,-0.5);
                        \draw[<->, PrimaryColor, thick] (-2.5,-1) -- (2.5,-1);
                        \draw[<->, PrimaryColor, thick] (-2.5,-1.5) -- (2.5,-1.5);
                        % x‐slab interval (blue): from -1.8 to 2.2
                        \draw[blue, fill=blue!80] (-1.8,-0.05) rectangle (2.2,0.05);
                        \node[blue, below] at (2.2,0)   {\footnotesize $t_{\max,x}$};
                        \node[blue, below] at (-1.8,0)  {\footnotesize $t_{\min,x}$};

                        % y‐slab interval (green): from -1.2 to 1.3
                        \draw[green, fill=green!80] (-1.2,-0.55) rectangle (1.3,-0.45);
                        \node[green, below] at (-1.2,-0.5) {\footnotesize $t_{\min,y}$};
                        \node[green, below] at (1.3,-0.5)  {\footnotesize $t_{\max,y}$};

                        % z‐slab interval (red): from 0.5 to 1.7
                        \draw[red, fill=red!80] (0.5,-1.05) rectangle (1.7,-0.95);
                        \node[red, below] at (0.5,-1)   {\footnotesize $t_{\min,z}$};
                        \node[red, below] at (1.7,-1)   {\footnotesize $t_{\max,z}$};

                        % enter/exit (AccentColor): exact overlap = [0.5, 1.3]
                        \draw[AccentColor, fill=AccentColor!80] (0.5,-1.55) rectangle (1.3,-1.45);
                        \node[AccentColor, below] at (0.5,-1.5) {\footnotesize $t_{\text{enter}}$};
                        \node[AccentColor, below] at (1.3,-1.5) {\footnotesize $t_{\text{exit}}$};
                    \end{tikzpicture}
                }
                \only<6>{
                    \textbf{Edge cases:}
                    \begin{itemize}
                        \item $R_{dx} = 0$ or $R_{dy} = 0$ or $R_{dz} = 0$: ray parallel to a faces \\
                        Automatically handled by division by zero 
                        \begin{itemize}
                            \item  $t_{\min} = -\infty$ and $t_{\max} = +\infty$ if valid
                            \item $t_{\min} = t_{\max} = \pm \infty$ if invalid 
                        \end{itemize}
                        \item $t_{\min} = t_{\max}$: 
                        \begin{itemize}
                            \item Just swap the values if $t_{\min} > t_{\max}$ \\
                                $t_{\min} \leftrightarrow t_{\max}$
                        \end{itemize}
                    \end{itemize}
                }
            \end{mathbox}
        \end{column}
        \begin{column}{0.45\textwidth}
            \small
            \begin{tikzpicture}[->, scale=0.5]
                \draw[->] (0,0) -- (3,0) node[right] {$x$};
                \draw[->] (0,0) -- (0,3) node[above] {$y$};
                \draw[->] (0,0) -- (-1.2,-1.2) node[below left] {$z$};
                \coordinate (RD) at (0.5,0.5,-2);
                \coordinate (Rd) at ($0.4714*(RD)$);
                \coordinate (Px1) at (2, 1.5, 8);
                \coordinate (Px2) at (4, 3.5, 0);
                \coordinate (Py1) at (2.5, 2, 6);
                \coordinate (Py2) at (4.5, 4, -2);
                \coordinate (Pz1) at (3.5, 3, 2); 
                \coordinate (Pz2) at (3,  2.5, 4); 
                \coordinate (Ro) at ($(Px1)-4*(Rd)$);
                \coordinate (Rf) at ($(Px2)+4*(Rd)$);
                \only<4->{
                \begin{scope}[canvas is zx plane at y=2]
                    \draw[green, fill=green!50, opacity=0.15] (-2,0) rectangle (6,6);
                    \draw[green, fill=green!20, opacity=0.6] (2,2) rectangle (4,4);
                \end{scope}
                }
                \only<1->{
                \begin{scope}[canvas is xy plane at z=2]
                    \draw[red, fill=red!50, opacity=0.15] (0,0) rectangle (6,6);
                    \draw[red, fill=red!20, opacity=0.6] (2,2) rectangle (4,4);
                    \only<1-2>{
                        \node[red, below right ] at (6,0) {$z = z_{\text{max}}$};
                    }
                \end{scope}
                }
                \only<2>{
                    \draw[blue,->] (3,3,2) -- (3,3,4) node[below] {$\mathbf{n}$};  
                }
                \only<4->{
                \begin{scope}[canvas is yz plane at x=2]
                    \draw[blue, fill=blue!50, opacity=0.15] (0,0) rectangle (6,8);
                    \draw[blue, fill=blue!20, opacity=0.6] (2,2) rectangle (4,4);
                \end{scope}
                }
                \only<1->{
                \begin{scope}[canvas is xy plane at z=4]
                    \draw[red, fill=red!50, opacity=0.15] (0,0) rectangle (6,6);
                    \draw[red, fill=red!20, opacity=0.6] (2,2) rectangle (4,4);
                    \only<1-2>{
                        \node[red, below right ] at (6,0) {$z = z_{\text{min}}$};
                    }
                \end{scope}
                }
                \only<4->{
                \begin{scope}[canvas is yz plane at x=4]
                    \draw[blue, fill=blue!50, opacity=0.15] (0,0) rectangle (6,8);
                    \draw[blue, fill=blue!20, opacity=0.6] (2,2) rectangle (4,4);
                \end{scope}
                }
                \only<4->{
                \begin{scope}[canvas is zx plane at y=4]
                    \draw[green, fill=green!50, opacity=0.15] (-2,0) rectangle (6,6);
                    \draw[green, fill=green!20, opacity=0.6] (2,2) rectangle (4,4);
                \end{scope}
                }
                \only<3->{
                    \draw[ray, very thick] (Ro) -- (Rf);
                    \draw[PrimaryColor, fill=PrimaryColor!60] (Ro) circle (2pt) node[below right] {$\mathbf{R_o}$};
                    \draw[red, ->, thick] (Ro) --++ ($2*(Rd)$) node[below,xshift=0.3cm] {$\mathbf{R_d}$};
                }
                \only<3->{
                    \fill[PrimaryColor!80] (Pz1) circle (2pt) node[below right] {$\mathbf{t}_{\text{max},z}$};
                    \fill[PrimaryColor!80] (Pz2) circle (2pt) node[below right] {$\mathbf{t}_{\text{min},z}$};
                }
                \only<4->{
                    \fill[PrimaryColor!80] (Py1) circle (2pt)
                            node[below right] {$\mathbf{t_{\min,y}}$};
                    \fill[PrimaryColor!80] (Py2) circle (2pt)
                        node[below right] {$\mathbf{t_{\max,y}}$};
                    \fill[PrimaryColor!80] (Px1) circle (2pt)
                        node[above left] {$\mathbf{t_{\min,x}}$};
                    \fill[PrimaryColor!80] (Px2) circle (2pt)
                        node[above left] {$\mathbf{t_{\max,x}}$};
                }
                \only<5->{
                    \draw[-, AccentColor, very thick] (Pz1) -- (Pz2);
                }

            \end{tikzpicture}
        \end{column}
    \end{columns}
\end{frame}

\endinput

\begin{frame}{AABB Intersection: Final Algorithm}
    \begin{columns}
        \begin{column}{0.5\textwidth}
            \begin{mathbox}{Ray-AABB Test}
                \textbf{Step 1:} Compute all slab intersections
                \begin{align*}
                    t_{enter} &= \max(t_{min,x}, t_{min,y}, t_{min,z}) \\
                    t_{exit} &= \min(t_{max,x}, t_{max,y}, t_{max,z})
                \end{align*}
                
                \textbf{Step 2:} Check intersection conditions
                \begin{itemize}
                    \item \alert{Hit if:} $t_{enter} \leq t_{exit}$ and $t_{exit} \geq 0$
                    \item \alert{Miss if:} $t_{enter} > t_{exit}$ or $t_{exit} < 0$
                \end{itemize}
                
                \textbf{Step 3:} Return closest intersection
                \begin{itemize}
                    \item If $t_{enter} \geq 0$: return $t_{enter}$
                    \item Else: return $t_{exit}$ (ray starts inside)
                \end{itemize}
            \end{mathbox}
        \end{column}
        \begin{column}{0.5\textwidth}
            \begin{tikzpicture}[scale=0.8]
                % 3D AABB with detailed ray intersection
                \begin{scope}[
                    x={(1cm,0cm)},
                    y={(0cm,1cm)},
                    z={(-0.5cm,-0.5cm)}
                ]
                    % AABB faces with transparency
                    \fill[ObjectColor!15, opacity=0.8] (1,1,1) -- (3,1,1) -- (3,3,1) -- (1,3,1) -- cycle;
                    \fill[ObjectColor!15, opacity=0.8] (1,1,3) -- (3,1,3) -- (3,3,3) -- (1,3,3) -- cycle;
                    \fill[ObjectColor!15, opacity=0.6] (1,1,1) -- (1,3,1) -- (1,3,3) -- (1,1,3) -- cycle;
                    \fill[ObjectColor!15, opacity=0.6] (3,1,1) -- (3,3,1) -- (3,3,3) -- (3,1,3) -- cycle;
                    \fill[ObjectColor!15, opacity=0.6] (1,1,1) -- (3,1,1) -- (3,1,3) -- (1,1,3) -- cycle;
                    \fill[ObjectColor!15, opacity=0.6] (1,3,1) -- (3,3,1) -- (3,3,3) -- (1,3,3) -- cycle;
                    
                    % AABB wireframe
                    \draw[ObjectColor, thick] (1,1,1) -- (3,1,1) -- (3,3,1) -- (1,3,1) -- cycle;
                    \draw[ObjectColor, thick] (1,1,3) -- (3,1,3) -- (3,3,3) -- (1,3,3) -- cycle;
                    \draw[ObjectColor, thick] (1,1,1) -- (1,1,3);
                    \draw[ObjectColor, thick] (3,1,1) -- (3,1,3);
                    \draw[ObjectColor, thick] (3,3,1) -- (3,3,3);
                    \draw[ObjectColor, thick] (1,3,1) -- (1,3,3);
                    
                    % Ray
                    \draw[ray, very thick] (0,0,0) -- (4,4,4);
                    \fill[PrimaryColor] (0,0,0) circle (2pt);
                    
                    % Intersection points
                    \fill[AccentColor] (1,1,1) circle (3pt);
                    \fill[AccentColor] (3,3,3) circle (3pt);
                    
                    % Labels
                    \node[below] at (1,1,1) {$t_{enter}$};
                    \node[above] at (3,3,3) {$t_{exit}$};
                \end{scope}
            \end{tikzpicture}
        \end{column}
    \end{columns}
\end{frame}

\begin{frame}{AABB Implementation Optimizations}
    \begin{columns}
        \begin{column}{0.6\textwidth}
            \begin{conceptbox}{Efficient Implementation}
                \textbf{Early Exit Strategies:}
                \begin{enumerate}
                    \item Check each axis incrementally
                    \item Exit immediately if $t_{enter} > t_{exit}$
                    \item Use IEEE floating point properties
                \end{enumerate}
                
                \vspace{0.3cm}
                \textbf{Handle Edge Cases:}
                \begin{itemize}
                    \item $R_d = 0$: ray parallel to slab faces
                    \item Negative ray direction: swap $t_{\text{min}}$ and $t_{\text{max}}$
                    \item Ray starting inside AABB
                \end{itemize}
                
                \vspace{0.3cm}
                \textbf{Performance Notes:}
                \begin{itemize}
                    \item AABB test is \alert{much faster} than triangle test
                    \item Enables BVH traversal with early pruning
                    \item Critical for real-time ray tracing
                \end{itemize}
            \end{conceptbox}
        \end{column}
        \begin{column}{0.4\textwidth}
            \begin{tikzpicture}[scale=0.6]
                % BVH tree visualization
                \node[draw, ObjectColor, thick] at (0,3) {AABB Root};
                \node[draw, ObjectColor] at (-1.5,1.5) {AABB L};
                \node[draw, ObjectColor] at (1.5,1.5) {AABB R};
                \node[draw, SecondaryColor] at (-2.5,0) {Tri 1};
                \node[draw, SecondaryColor] at (-0.5,0) {Tri 2};
                \node[draw, SecondaryColor] at (0.5,0) {Tri 3};
                \node[draw, SecondaryColor] at (2.5,0) {Tri 4};
                
                % Connections
                \draw[thick] (0,3) -- (-1.5,1.5);
                \draw[thick] (0,3) -- (1.5,1.5);
                \draw[thick] (-1.5,1.5) -- (-2.5,0);
                \draw[thick] (-1.5,1.5) -- (-0.5,0);
                \draw[thick] (1.5,1.5) -- (0.5,0);
                \draw[thick] (1.5,1.5) -- (2.5,0);
                
                % Ray traversal
                \draw[ray, very thick] (-3,2) -- (3,2);
                \fill[AccentColor] (0,2) circle (2pt);
                \node[above] at (0,2.3) {\small Hit Root};
                \fill[red] (1.5,2) circle (2pt);
                \node[above] at (1.5,2.3) {\small Hit R};
                \draw[red, thick, opacity=0.5] (-1.5,1.5) circle (0.3);
                \node[below] at (-1.5,1) {\small \color{red} Miss L};
                
                \node[below] at (0,-1) {\small BVH Acceleration};
            \end{tikzpicture}
        \end{column}
    \end{columns}
\end{frame}