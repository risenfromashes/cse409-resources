\section{Coordinate System Transformations}

\begin{frame}{The MVP Matrix Pipeline}
    \begin{center}
        \begin{tikzpicture}[scale=0.8, node distance=2cm]
            % Coordinate spaces
            \node[process, fill=ObjectColor!20] (model) {Model Space};
            \node[process, right of=model, fill=PrimaryColor!20] (world) {World Space};
            \node[process, right of=world, fill=SecondaryColor!20] (view) {View Space};
            \node[process, right of=view, fill=AccentColor!20] (clip) {Clip Space};
            
            % Transformations
            \draw[arrow] (model) -- (world) node[midway, above] {\scriptsize \textbf{M}};
            \draw[arrow] (world) -- (view) node[midway, above] {\scriptsize \textbf{V}};
            \draw[arrow] (view) -- (clip) node[midway, above] {\scriptsize \textbf{P}};
            
            % Next steps
            \node[process, below of=clip, fill=RayColor!20] (ndc) {NDC};
            \node[process, left of=ndc, fill=LightGray] (screen) {Screen Space};
            
            \draw[arrow] (clip) -- (ndc) node[midway, right] {\scriptsize \textbf{÷w}};
            \draw[arrow] (ndc) -- (screen) node[midway, above] {\scriptsize \textbf{Viewport}};
            
            % Labels
            \node[below] at (model.south) {\scriptsize Local object};
            \node[below] at (world.south) {\scriptsize Scene};
            \node[below] at (view.south) {\scriptsize Camera};
            \node[below] at (clip.south) {\scriptsize Projection};
            \node[below] at (ndc.south) {\scriptsize [-1,1]³};
            \node[below] at (screen.south) {\scriptsize Pixels};
        \end{tikzpicture}
    \end{center>
    
    \pause
    \begin{mathbox}{The Complete Transformation}
        \begin{align}
            \mathbf{v}_{\text{clip}} &= \mathbf{P} \times \mathbf{V} \times \mathbf{M} \times \mathbf{v}_{\text{model}} \\
            \mathbf{v}_{\text{ndc}} &= \frac{\mathbf{v}_{\text{clip}}}{w_{\text{clip}}} \\
            \mathbf{v}_{\text{screen}} &= \text{Viewport}(\mathbf{v}_{\text{ndc}})
        \end{align}
    \end{mathbox>
\end{frame>

\begin{frame}>{Why Homogeneous Coordinates?}
    \begin{columns}
        \begin{column}>{0.5\textwidth>
            \begin{conceptbox}>{The W Component}
                \textbf{4D coordinates} $(x, y, z, w)$ enable:
                \begin{itemize}
                    \item \textbf{Perspective projection:} $w \neq 1$
                    \item \textbf{Uniform transformations:} All as matrices
                    \item \textbf{Efficient GPU processing:} 4-component vectors
                    \item \textbf{Clipping in homogeneous space:} Before division
                \end{itemize>
                
                \vspace{0.3cm}
                \textbf{The perspective divide:} $\frac{(x,y,z,w)}{w} = \left(\frac{x}{w}, \frac{y}{w}, \frac{z}{w}, 1\right)$
            \end{conceptbox>
        \end{column>
        \begin{column}>{0.5\textwidth>
            \begin{tikzpicture}[scale=0.7]
                % Perspective projection visualization
                \node[above] at (0,3) {\textbf{Perspective Effect}};
                
                % Frustum
                \draw[thick, gray] (0,0) -- (3,1.5) -- (3,-1.5) -- cycle;
                \draw[thick, gray] (0,0) -- (3,0.75) -- (3,-0.75) -- cycle;
                
                % Near plane
                \draw[thick, PrimaryColor] (1,0.5) -- (1,-0.5);
                \node[above] at (1,0.7) {\scriptsize Near};
                
                % Far plane  
                \draw[thick, AccentColor] (3,1.5) -- (3,-1.5);
                \node[above] at (3,1.7) {\scriptsize Far};
                
                % Objects at different depths
                \fill[SecondaryColor] (1.5,0.2) circle (0.1);
                \fill[SecondaryColor] (2.5,0.5) circle (0.1);
                \node[right] at (1.6,0.2) {\scriptsize Same size};
                \node[right] at (2.6,0.5) {\scriptsize in world};
                
                % After projection
                \begin{scope}[shift={(0,-3)}]
                    \draw[thick, gray] (1,-1) rectangle (1,1);
                    \node[above] at (1,1.2) {\textbf{After Projection}};
                    
                    \fill[SecondaryColor] (1,0.2) circle (0.1);
                    \fill[SecondaryColor] (1,0.05) circle (0.05);
                    \node[right] at (1.1,0.2) {\scriptsize Near: larger};
                    \node[right] at (1.1,0.05) {\scriptsize Far: smaller};
                \end{scope>
            \end{tikzpicture>
        \end{column>
    \end{columns>
\end{frame>

\begin{frame}>{NDC: The Normalized Device Coordinates}
    \begin{columns}
        \begin{column}>{0.6\textwidth>
            \begin{raybox}>{NDC Properties}
                \textbf{Cube from [-1, 1] in all axes:}
                \begin{itemize}
                    \item \textbf{Device independent:} Same for all GPUs
                    \item \textbf{Clipping friendly:} Easy bounds checking
                    \item \textbf{Symmetric:} Origin at center
                    \item \textbf{Z-depth:} Usually [0,1] or [-1,1]
                \end{itemize>
                
                \vspace{0.3cm}
                \textbf{Anything outside [-1,1] gets clipped}
            \end{raybox>
            
            \pause
            \begin{mathbox}>{Viewport Transformation}
                \footnotesize
                From NDC to screen coordinates:
                \begin{align}
                    x_{\text{screen}} &= \frac{x_{\text{ndc}} + 1}{2} \times \text{width} + x_{\text{offset}} \\
                    y_{\text{screen}} &= \frac{y_{\text{ndc}} + 1}{2} \times \text{height} + y_{\text{offset}}
                \end{align>
            \end{mathbox>
        \end{column>
        \begin{column}>{0.4\textwidth>
            \begin{tikzpicture}[scale=0.8]
                % NDC cube
                \node[above] at (0,3) {\textbf{NDC Space}};
                \draw[thick, PrimaryColor] (-1,-1) rectangle (1,1);
                \node[below left] at (-1,-1) {(-1,-1)};
                \node[above right] at (1,1) {(1,1)};
                \node[below] at (0,-1.3) {\scriptsize All devices};
                
                % Arrow down
                \draw[->, thick] (0,-1.5) -- (0,-2.5);
                \node[right] at (0.2,-2) {\textbf{Viewport}};
                
                % Screen space
                \node[above] at (0,-2.7) {\textbf{Screen Space}};
                \draw[thick, AccentColor] (-1.5,-3) rectangle (1.5,-1);
                \node[below left] at (-1.5,-3) {(0,0)};
                \node[above right] at (1.5,-1) {(w,h)};
                \node[below] at (0,-3.3) {\scriptsize Device pixels};
                
                % Grid overlay
                \draw[gray, very thin] (-1,-1) grid[step=0.5] (1,1);
                \draw[gray, very thin] (-1.5,-3) grid[step=0.5] (1.5,-1);
            \end{tikzpicture>
        \end{column>
    \end{columns>
\end{frame>

\begin{frame}>{Modern Clipping: Guard Bands}
    \begin{columns}
        \begin{column}>{0.5\textwidth>
            \begin{conceptbox}>{Traditional Clipping}
                \textbf{Sutherland-Hodgman algorithm:}
                \begin{itemize}
                    \item Clip against each plane
                    \item Generate new vertices
                    \item Complex geometry cases
                    \item Software implementation
                \end{itemize>
            \end{conceptbox>
            
            \vspace{0.3cm}
            
            \begin{raybox}>{Modern Guard Bands}
                \textbf{Hardware approach:}
                \begin{itemize}
                    \item Larger clipping region
                    \item Rasterizer handles boundaries
                    \item Fewer geometry modifications
                    \item Better performance
                \end{itemize>
            \end{raybox>
        \end{column>
        \begin{column}>{0.5\textwidth>
            \begin{tikzpicture}[scale=0.8]
                % Traditional clipping
                \node[above] at (0,3.5) {\textbf{Traditional}};
                \draw[thick, PrimaryColor] (-1,2) rectangle (1,3);
                \node[below] at (0,1.8) {\scriptsize Exact clipping};
                
                % Triangle that needs clipping
                \draw[thick, red] (-1.5,2.2) -- (0.5,3.5) -- (0.2,1.8) -- cycle;
                \draw[red, dashed] (-1,2.3) -- (0.3,2.1);
                \node[right] at (0.6,2.7) {\scriptsize \textcolor{red}{Complex}};
                
                % Modern approach
                \begin{scope}[shift={(0,-2.5)}]
                    \node[above] at (0,3.5) {\textbf{Guard Band}};
                    
                    % Actual viewport
                    \draw[thick, PrimaryColor] (-1,2) rectangle (1,3);
                    \node[below] at (0,1.8) {\scriptsize Viewport};
                    
                    % Guard band
                    \draw[thick, AccentColor, dashed] (-1.5,1.5) rectangle (1.5,3.5);
                    \node[right] at (1.6,2.5) {\scriptsize Guard band};
                    
                    % Same triangle - no clipping needed
                    \draw[thick, PrimaryColor] (-1.5,2.2) -- (0.5,3.5) -- (0.2,1.8) -- cycle;
                    \node[right] at (0.6,2.7) {\scriptsize \textcolor{PrimaryColor}{Simple}};
                \end{scope>
            \end{tikzpicture>
        \end{column>
    \end{columns>
\end{frame>
