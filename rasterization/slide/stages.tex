\section{Pipeline Stages in Detail}

\begin{frame}[fragile]{Vertex Processing: The Transformation Stage}
  \begin{columns}
    \begin{column}{0.5\textwidth}
      \begin{raybox}{Vertex Shader Responsibilities}
        \textbf{Must do:}
        \begin{itemize}
          \item Transform vertices to clip space
          \item Set \texttt{gl\_Position}
        \end{itemize}

        \textbf{Can also do:}
        \begin{itemize}
          \item Compute lighting per vertex
          \item Transform normals
          \item Generate texture coordinates
          \item Vertex animation/skinning
        \end{itemize}
      \end{raybox}
    \end{column}
    \begin{column}{0.5\textwidth}
      \begin{tikzpicture}[scale=0.8]
        % 3D model
        \node[above] at (0,3) {\textbf{Model Space}};
        \begin{scope}[canvas is xy plane at z=0, scale=0.6]
          \draw[thick, ObjectColor] (0,0) -- (2,0) -- (1,2) -- cycle;
          \fill[ObjectColor] (0,0) circle (0.1);
          \fill[ObjectColor] (2,0) circle (0.1);
          \fill[ObjectColor] (1,2) circle (0.1);
          \node[below] at (1,-0.3) {\scriptsize Local coordinates};
        \end{scope}

        \draw[->, thick, PrimaryColor] (0,1.8) -- (0,0.8);
        \node[right] at (0.2,1.3) {\textcolor{PrimaryColor}{\textbf{Vertex Shader}}};

        % Screen space
        \node[below] at (0,-0.5) {\textbf{Clip Space}};
        \begin{scope}[shift={(0,-1.5)}, scale=0.6]
          \draw[thick, PrimaryColor] (0,0) -- (2,0) -- (1,1.5) -- cycle;
          \fill[PrimaryColor] (0,0) circle (0.1);
          \fill[PrimaryColor] (2,0) circle (0.1);
          \fill[PrimaryColor] (1,1.5) circle (0.1);
          \node[below] at (1,-0.3) {\scriptsize Ready for rasterization};
        \end{scope}
      \end{tikzpicture}
    \end{column}
  \end{columns}

  \pause
  \begin{mathbox}{Typical Vertex Shader (GLSL)}
    \footnotesize
        \begin{verbatim}
#version 330 core
layout (location = 0) in vec3 position;
layout (location = 1) in vec3 normal;

uniform mat4 mvpMatrix;
uniform mat4 modelMatrix;

out vec3 worldNormal;

void main() {
    gl_Position = mvpMatrix * vec4(position, 1.0);
    worldNormal = mat3(modelMatrix) * normal;
}
        \end{verbatim}
  \end{mathbox}
\end{frame}

\begin{frame}{Primitive Assembly \& Clipping}
  \begin{columns}
    \begin{column}{0.6\textwidth}
      \begin{conceptbox}{Primitive Assembly}
        \textbf{Fixed Function Stage:}
        \begin{itemize}
          \item Group vertices into primitives (triangles, lines, points)
          \item Handle different primitive types
          \item Prepare for clipping
        \end{itemize}
      \end{conceptbox}

      \vspace{0.3cm}

      \begin{conceptbox}{Clipping \& Culling}
        \textbf{Hardware optimizations:}
        \begin{itemize}
          \item \textbf{Frustum clipping:} Remove off-screen geometry
          \item \textbf{Backface culling:} Remove back-facing triangles
          \item \textbf{Guard bands:} Modern approach vs Sutherland-Hodgman
        \end{itemize}
      \end{conceptbox}
    \end{column}
    \begin{column}{0.4\textwidth}
      \begin{tikzpicture}[scale=0.7]
        % View frustum
        \draw[thick, gray] (-1.5,-1) -- (1.5,-1) -- (1.5,1) -- (-1.5,1) -- cycle;
        \node[above] at (0,1.2) {\scriptsize View Frustum};

        % Triangles - some inside, some outside
        \draw[thick, PrimaryColor] (-1,0) -- (0,-0.5) -- (0,0.5) -- cycle;
        \fill[PrimaryColor!30] (-1,0) -- (0,-0.5) -- (0,0.5) -- cycle;
        \node[right] at (0.2,0) {\scriptsize \textcolor{PrimaryColor}{Keep}};

        \draw[thick, red] (0.5,0) -- (2,0) -- (1.5,1.5) -- cycle;
        \fill[red!30] (0.5,0) -- (2,0) -- (1.5,1.5) -- cycle;
        \node[above] at (1.5,1.7) {\scriptsize \textcolor{red}{Clip}};

        \draw[thick, gray] (-2.5,0.5) -- (-2,0) -- (-2.5,-0.5) -- cycle;
        \fill[gray!30] (-2.5,0.5) -- (-2,0) -- (-2.5,-0.5) -- cycle;
        \node[left] at (-2.7,0) {\scriptsize \textcolor{gray}{Cull}};

        % Backface culling illustration
        \begin{scope}[shift={(0,-2.5)}]
          \node[above] at (0,0.5) {\textbf{Backface Culling}};

          % Front face
          \draw[thick, PrimaryColor] (-1,0) -- (0,-0.5) -- (0,0.5) -- cycle;
          \fill[PrimaryColor!30] (-1,0) -- (0,-0.5) -- (0,0.5) -- cycle;
          \draw[->, thick] (-0.3,0) -- (-0.8,0);
          \node[below] at (-0.5,-0.7) {\scriptsize Front (render)};

          % Back face
          \draw[thick, red, dashed] (1,0) -- (0.5,-0.5) -- (0.5,0.5) -- cycle;
          \draw[->, thick] (0.7,0) -- (1.2,0);
          \node[below] at (0.75,-0.7) {\scriptsize Back (cull)};
        \end{scope}
      \end{tikzpicture}
    \end{column}
  \end{columns}
\end{frame}

\begin{frame}{Rasterization: From Triangles to Pixels}
  \begin{center}
    \begin{tikzpicture}[scale=0.9]
      % Triangle
      \coordinate (A) at (0,0);
      \coordinate (B) at (4,0);
      \coordinate (C) at (2,3);

      \draw[thick, PrimaryColor] (A) -- (B) -- (C) -- cycle;
      \fill[PrimaryColor!20] (A) -- (B) -- (C) -- cycle;

      % Grid overlay
      \draw[gray, very thin] (0,0) grid[step=0.5] (4,3);

      % Highlight pixels inside triangle
      \foreach \x in {1,1.5,2,2.5} {
        \foreach \y in {0.5,1,1.5} {
          \pgfmathparse{(\x-0)/2 + (\y-0)/3 <= 1 && (\x-4)/(-2) + (\y-0)/3 <= 1 && (\x-2)/2 + (\y-3)/(-3) <= 1 ? 1 : 0}
          \ifnum\pgfmathresult>0
          \fill[AccentColor] (\x,\y) rectangle ++(0.5,0.5);
          \fi
        }
      }

      % Additional pixels
      \fill[AccentColor] (1.5,2) rectangle ++(0.5,0.5);
      \fill[AccentColor] (2,2) rectangle ++(0.5,0.5);
      \fill[AccentColor] (2.5,2) rectangle ++(0.5,0.5);
      \fill[AccentColor] (2,2.5) rectangle ++(0.5,0.5);
      \fill[AccentColor] (1.5,1) rectangle ++(0.5,0.5);
      \fill[AccentColor] (2.5,1) rectangle ++(0.5,0.5);

      % Labels
      \node[below] at (A) {$(x_1, y_1)$};
      \node[below] at (B) {$(x_2, y_2)$};
      \node[above] at (C) {$(x_3, y_3)$};

      \draw[->, thick] (5,1.5) -- (7,1.5);

      % Pixel grid result
      \begin{scope}[shift={(8,0)}]
        \foreach \x in {0,0.5,...,2} {
          \foreach \y in {0,0.5,...,1.5} {
            \draw[AccentColor] (\x,\y) rectangle ++(0.5,0.5);
          }
        }
        \node[below] at (1,-0.5) {\textbf{Fragment Grid}};
      \end{scope}
    \end{tikzpicture}
  \end{center}

  \begin{conceptbox}{Hardware Rasterization}
    \footnotesize
    \textbf{What the hardware does:}
    \begin{itemize}
      \item \textbf{Edge equations:} Determine which pixels are inside the triangle
      \item \textbf{Interpolation:} Compute attributes (color, texture coords) per pixel
      \item \textbf{Fragment generation:} Create fragments for pixel shader processing
      \item \textbf{Early culling:} Discard fragments that fail depth test early
    \end{itemize}
  \end{conceptbox}
\end{frame}

\begin{frame}[fragile]{Fragment Processing: The Heavy Lifting}
  \begin{columns}
    \begin{column}{0.5\textwidth}
      \begin{raybox}{Fragment Shader Tasks}
        \textbf{Compute final pixel color:}
        \begin{itemize}
          \item Texture sampling
          \item Lighting calculations
          \item Material properties
          \item Special effects
          \item Transparency
        \end{itemize}

        \vspace{0.2cm}
        \textcolor{PrimaryColor}{\textbf{Most expensive stage!}}
      \end{raybox}
    \end{column}
    \begin{column}{0.5\textwidth}
      \begin{tikzpicture}[scale=0.8]
        % Fragment input
        \node[rectangle, draw, fill=AccentColor!20] (frag) at (0,2) {Fragment};

        % Inputs
        \node[left] at (-1.5,2.5) {\scriptsize Position};
        \node[left] at (-1.5,2) {\scriptsize Normals};
        \node[left] at (-1.5,1.5) {\scriptsize UV coords};

        \draw[->, thin] (-1.2,2.5) -- (frag.west);
        \draw[->, thin] (-1.2,2) -- (frag.west);
        \draw[->, thin] (-1.2,1.5) -- (frag.west);

        % Processing
        \node[rectangle, draw, fill=PrimaryColor!20] (shader) at (0,0) {
          \begin{minipage}{1.5cm}
            \centering
            \textbf{Fragment} \\
            \textbf{Shader}
          \end{minipage}
        };

        \draw[->, thick] (frag) -- (shader);

        % Resources
        \node[right] at (1.5,0.5) {\scriptsize Textures};
        \node[right] at (1.5,0) {\scriptsize Uniforms};
        \node[right] at (1.5,-0.5) {\scriptsize Samplers};

        \draw[->, thin] (1.2,0.5) -- (shader.east);
        \draw[->, thin] (1.2,0) -- (shader.east);
        \draw[->, thin] (1.2,-0.5) -- (shader.east);

        % Output
        \node[rectangle, draw, fill=LightColor!20] (color) at (0,-2) {
          \begin{minipage}{1.5cm}
            \centering
            \textbf{Final} \\
            \textbf{Color}
          \end{minipage}
        };

        \draw[->, thick] (shader) -- (color);
      \end{tikzpicture}
    \end{column}
  \end{columns}

  \pause
  \begin{mathbox}{Typical Fragment Shader (GLSL)}
    \footnotesize
    \begin{minted}[fontsize=\small, linenos]{glsl}
      #version 330 core
      in vec3 worldNormal;
      in vec2 texCoord;

      uniform sampler2D diffuseTexture;
      uniform vec3 lightDirection;

      out vec4 fragColor;

      void main() {
        vec3 normal = normalize(worldNormal);
        float NdotL = max(dot(normal, -lightDirection), 0.0);
        vec3 texColor = texture(diffuseTexture, texCoord).rgb;
        fragColor = vec4(texColor * NdotL, 1.0);
      }
    \end{minted}
  \end{mathbox}
\end{frame}

\begin{frame}{Per-Fragment Operations: The Final Steps}
  \begin{center}
    \begin{tikzpicture}[scale=0.8, node distance=1.2cm]
      % Fragment from shader
      \node[process, fill=PrimaryColor!20] (frag) {Fragment};

      % Tests and operations
      \node[process, below of=frag, fill=SecondaryColor!20] (scissor) {Scissor Test};
      \node[process, below of=scissor, fill=AccentColor!20] (stencil) {Stencil Test};
      \node[process, below of=stencil, fill=RayColor!20] (depth) {Depth Test};
      \node[process, below of=depth, fill=ObjectColor!20] (blend) {Blending};
      \node[process, below of=blend, fill=LightGray] (fb) {Frame Buffer};

      % Arrows with conditional paths
      \draw[arrow] (frag) -- (scissor);
      \draw[arrow] (scissor) -- (stencil);
      \draw[arrow] (stencil) -- (depth);
      \draw[arrow] (depth) -- (blend);
      \draw[arrow] (blend) -- (fb);

      % Discard paths
      \draw[->, red, dashed] (scissor.east) -- ++(2,0) node[right] {\textcolor{red}{Discard}};
      \draw[->, red, dashed] (stencil.east) -- ++(2,0) node[right] {\textcolor{red}{Discard}};
      \draw[->, red, dashed] (depth.east) -- ++(2,0) node[right] {\textcolor{red}{Discard}};

      % Configurations
      \node[right, text width=3cm] at (6,-1.2) {
        \scriptsize \textbf{Configurable:} \\
        Enable/disable each test
      };
      \node[right, text width=3cm] at (6,-2.4) {
        \scriptsize \textbf{Stencil:} \\
        Complex masking operations
      };
      \node[right, text width=3cm] at (6,-3.6) {
        \scriptsize \textbf{Depth:} \\
        LESS, EQUAL, ALWAYS, etc.
      };
      \node[right, text width=3cm] at (6,-4.8) {
        \scriptsize \textbf{Blend:} \\
        Alpha, additive, multiply
      };
    \end{tikzpicture}
  \end{center}
\end{frame}
