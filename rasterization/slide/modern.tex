\section{Modern Rendering Techniques}

\begin{frame}{Forward vs Deferred Rendering}
  \begin{columns}
    \begin{column}{0.5\textwidth}
      \begin{raybox}{Forward Rendering}
        \textbf{Traditional approach:}
        \begin{itemize}
          \item Render objects front-to-back
          \item Compute lighting per fragment
          \item Multiple passes for multiple lights
          \item Good for few lights
        \end{itemize}

        \vspace{0.3cm}
        \textbf{Complexity:} $O(\text{fragments} \times \text{lights})$
      \end{raybox}
    \end{column}
    \begin{column}{0.5\textwidth}
      \begin{conceptbox}{Deferred Rendering}
        \textbf{Modern approach:}
        \begin{itemize}
          \item Render geometry to G-buffer
          \item Store material properties
          \item Lighting pass operates on screen
          \item Excellent for many lights
        \end{itemize}

        \vspace{0.3cm}
        \textbf{Complexity:} $O(\text{pixels} \times \text{lights})$
      \end{conceptbox}
    \end{column}
  \end{columns}

  \pause
  \begin{center}
    \begin{tikzpicture}[scale=0.8]
      % Forward rendering
      \node[above] at (-3,2) {\textbf{Forward}};
      \node[process, fill=ObjectColor!20] (geom1) at (-3,1) {Geometry};
      \node[process, fill=PrimaryColor!20] (light1) at (-3,0) {+ Lighting};
      \node[process, fill=LightGray] (fb1) at (-3,-1) {Frame Buffer};

      \draw[arrow] (geom1) -- (light1);
      \draw[arrow] (light1) -- (fb1);

      % Multiple light passes
      \draw[->, dashed] (-1.5,0) -- (-1.5,-0.5) -- (-4.5,-0.5) -- (-4.5,0.5);
      \node[right] at (-1.3,0) {\scriptsize For each light};

      % Deferred rendering
      \node[above] at (3,2) {\textbf{Deferred}};
      \node[process, fill=ObjectColor!20] (geom2) at (3,1) {Geometry};
      \node[process, fill=AccentColor!20] (gbuf) at (3,0) {G-Buffer};
      \node[process, fill=PrimaryColor!20] (light2) at (3,-1) {Lighting Pass};
      \node[process, fill=LightGray] (fb2) at (3,-2) {Frame Buffer};

      \draw[arrow] (geom2) -- (gbuf);
      \draw[arrow] (gbuf) -- (light2);
      \draw[arrow] (light2) -- (fb2);

      % G-buffer breakdown
      \node[right, text width=2cm] at (4.5,0) {
        \scriptsize Albedo, Normal, Depth, Material params
      };
    \end{tikzpicture}
  \end{center}
\end{frame}

\begin{frame}{Screen-Space Techniques}
  \begin{columns}
    \begin{column}{0.6\textwidth}
      \begin{raybox}{Screen-Space Philosophy}
        \textbf{Operate on the final image:}
        \begin{itemize}
          \item Use depth buffer as 3D information
          \item Process in screen space (2D)
          \item Approximate 3D effects cheaply
          \item GPU-friendly parallel processing
        \end{itemize}

        \vspace{0.3cm}
        \textbf{Examples:}
        \begin{itemize}
          \item \textbf{SSAO:} Ambient occlusion
          \item \textbf{SSR:} Screen-space reflections
          \item \textbf{SSGI:} Global illumination
          \item \textbf{Temporal effects:} Motion blur, TAA
        \end{itemize}
      \end{raybox}
    \end{column}
    \begin{column}{0.4\textwidth}
      \begin{tikzpicture}[scale=0.7]
        % Screen-space processing
        \node[above] at (0,3) {\textbf{Screen-Space}};

        % Input buffers
        \draw[thick, ObjectColor] (-1,2) rectangle (1,2.5);
        \node[right] at (1.1,2.25) {\scriptsize Color};

        \draw[thick, PrimaryColor] (-1,1.5) rectangle (1,2);
        \node[right] at (1.1,1.75) {\scriptsize Depth};

        \draw[thick, AccentColor] (-1,1) rectangle (1,1.5);
        \node[right] at (1.1,1.25) {\scriptsize Normals};

        % Processing
        \node[process, fill=RayColor!20] (process) at (0,0) {
          \begin{minipage}{1.5cm}
            \centering
            \textbf{Screen} \\
            \textbf{Effect}
          \end{minipage}
        };

        \draw[->, thick] (0,1) -- (process);

        % Output
        \draw[thick, LightColor!50] (-1,-1) rectangle (1,-0.5);
        \node[right] at (1.1,-0.75) {\scriptsize Enhanced};

        \draw[->, thick] (process) -- (0,-0.5);

        % SSAO example
        \begin{scope}[shift={(0,-2.5)}]
          \node[above] at (0,0.5) {\textbf{SSAO Example}};

          % Sample points
          \foreach \angle in {0,45,90,135,180,225,270,315} {
            \draw[->, thin] (0,0) -- (\angle:0.4);
          }
          \fill[red] (0,0) circle (0.05);
          \node[below] at (0,-0.7) {\scriptsize Sample around pixel};
        \end{scope}
      \end{tikzpicture}
    \end{column}
  \end{columns}
\end{frame}

\begin{frame}{Multi-Pass Rendering Strategies}
  \begin{center}
    \begin{tikzpicture}[scale=0.7, node distance=1.5cm]
      % Shadow mapping pass
      \node[process, fill=DarkGray!20] (shadow) {Shadow Map Pass};
      \node[below of=shadow, process, fill=ObjectColor!20] (geom) {Geometry Pass};
      \node[below of=geom, process, fill=PrimaryColor!20] (lighting) {Lighting Pass};
      \node[below of=lighting, process, fill=AccentColor!20] (post) {Post-Processing};
      \node[below of=post, process, fill=LightGray] (final) {Final Image};

      % Arrows
      \draw[arrow] (shadow) -- (geom);
      \draw[arrow] (geom) -- (lighting);
      \draw[arrow] (lighting) -- (post);
      \draw[arrow] (post) -- (final);

      % Side information
      \node[right, text width=3cm] at (3,2) {
        \scriptsize \textbf{From light's POV} \\
        Store depth information
      };

      \node[right, text width=3cm] at (3,0.5) {
        \scriptsize \textbf{Render scene} \\
        Fill G-buffers
      };

      \node[right, text width=3cm] at (3,-1) {
        \scriptsize \textbf{Use shadow maps} \\
        Compute illumination
      };

      \node[right, text width=3cm] at (3,-2.5) {
        \scriptsize \textbf{Screen effects} \\
        Bloom, tone mapping
      };

      % Render targets
      \node[left, text width=2.5cm] at (-3,2) {
        \scriptsize \textcolor{DarkGray}{\textbf{Shadow RT}} \\
        Depth only
      };

      \node[left, text width=2.5cm] at (-3,0.5) {
        \scriptsize \textcolor{ObjectColor}{\textbf{G-Buffer}} \\
        Multiple RTs
      };

      \node[left, text width=2.5cm] at (-3,-1) {
        \scriptsize \textcolor{PrimaryColor}{\textbf{HDR Buffer}} \\
        High dynamic range
      };

      \node[left, text width=2.5cm] at (-3,-2.5) {
        \scriptsize \textcolor{AccentColor}{\textbf{LDR Buffer}} \\
        Final processing
      };
    \end{tikzpicture}
  \end{center}
\end{frame}

\begin{frame}{Tessellation: Modern Geometry Processing}
  \begin{columns}
    \begin{column}{0.5\textwidth}
      \begin{conceptbox}{Why Tessellation?}
        \textbf{Adaptive detail:}
        \begin{itemize}
          \item Generate geometry on GPU
          \item Level-of-detail based on distance
          \item Displacement mapping
          \item Smooth curved surfaces
        \end{itemize}

        \vspace{0.3cm}
        \textbf{Two-stage process:}
        \begin{itemize}
          \item \textbf{Hull Shader:} Control tessellation
          \item \textbf{Domain Shader:} Generate vertices
        \end{itemize}
      \end{conceptbox}
    \end{column}
    \begin{column}{0.5\textwidth}
      \begin{tikzpicture}[scale=0.8]
        % Input patch
        \node[above] at (0,3) {\textbf{Tessellation Pipeline}};

        \draw[thick, ObjectColor] (-1,2) -- (1,2) -- (0,2.5) -- cycle;
        \node[below] at (0,1.8) {\scriptsize Input Patch};

        \draw[->, thick] (0,1.7) -- (0,1.3);
        \node[right] at (0.2,1.5) {\scriptsize Hull Shader};

        % Tessellated result
        \begin{scope}[shift={(0,-0.5)}]
          % Subdivided triangle
          \draw[thick, PrimaryColor] (-1,1) -- (1,1) -- (0,1.5) -- cycle;
          \draw[thin, PrimaryColor] (-0.5,1) -- (0,1.25);
          \draw[thin, PrimaryColor] (0.5,1) -- (0,1.25);
          \draw[thin, PrimaryColor] (-0.5,1) -- (0.5,1);

          \node[below] at (0,0.8) {\scriptsize Tessellated};
        \end{scope}

        \draw[->, thick] (0,0.7) -- (0,0.3);
        \node[right] at (0.2,0.5) {\scriptsize Domain Shader};

        % Final displaced geometry
        \begin{scope}[shift={(0,-1.5)}]
          \draw[thick, AccentColor] (-1,0) .. controls (-0.5,0.3) and (0.5,0.3) .. (1,0)
          .. controls (0.3,0.8) .. (0,0.7)
          .. controls (-0.3,0.8) .. cycle;
          \node[below] at (0,-0.3) {\scriptsize Displaced Surface};
        \end{scope}

        % LOD illustration
        \begin{scope}[shift={(2.5,0)}]
          \node[above] at (0,1.5) {\textbf{Adaptive LOD}};

          % High detail (close)
          \foreach \x in {0,0.2,0.4} {
            \foreach \y in {0,0.2} {
              \draw[AccentColor] (\x,\y) rectangle ++(0.15,0.15);
            }
          }
          \node[below] at (0.3,-0.3) {\scriptsize High (close)};

          % Low detail (far)
          \begin{scope}[shift={(0,-1)}]
            \draw[AccentColor] (0,0) rectangle ++(0.6,0.4);
            \node[below] at (0.3,-0.3) {\scriptsize Low (far)};
          \end{scope}
        \end{scope}
      \end{tikzpicture}
    \end{column}
  \end{columns}
\end{frame}
