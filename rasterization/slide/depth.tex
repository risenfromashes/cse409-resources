\section{Depth \& Visibility}

\begin{frame}{Z-Buffer: The Depth Solution}
    \begin{columns}
        \begin{column}{0.5\textwidth}
            \begin{raybox}{Z-Buffer Algorithm}
                \textbf{For each pixel:}
                \begin{enumerate}
                    \item Initialize depth buffer to far plane
                    \item For each fragment:
                        \begin{itemize}
                            \item Compare fragment depth to stored depth
                            \item If closer, update depth \& color
                            \item If farther, discard fragment
                        \end{itemize}
                \end{enumerate>
                
                \vspace{0.2cm}
                \textbf{Hardware accelerated!}
            \end{raybox>
        \end{column>
        \begin{column}{0.5\textwidth}
            \begin{tikzpicture}[scale=0.7]
                % Scene setup
                \node[above] at (0,3.5) {\textbf{Scene}};
                
                % Multiple triangles at different depths
                \fill[PrimaryColor!50] (-1,2) -- (1,2) -- (0,3) -- cycle;
                \node[right] at (1.1,2.3) {\scriptsize Triangle A (z=0.3)};
                
                \fill[AccentColor!50] (-0.5,1.5) -- (1.5,1.5) -- (0.5,2.5) -- cycle;
                \node[right] at (1.6,2) {\scriptsize Triangle B (z=0.7)};
                
                % Pixel location
                \draw[red, thick] (0.3,1.8) circle (0.1);
                \node[below] at (0.3,1.6) {\scriptsize Pixel};
                
                % Depth buffer illustration
                \begin{scope}[shift={(0,-2)}]
                    \node[above] at (0,2) {\textbf{Z-Buffer Process}};
                    
                    % Step 1
                    \draw[gray] (-1,1) rectangle (2,1.5);
                    \node[left] at (-1.2,1.25) {\scriptsize 1.};
                    \node[right] at (2.1,1.25) {\scriptsize Init: ∞};
                    
                    % Step 2
                    \draw[PrimaryColor] (-1,0.5) rectangle (2,1);
                    \node[left] at (-1.2,0.75) {\scriptsize 2.};
                    \node[right] at (2.1,0.75) {\scriptsize A: 0.3 < ∞ → store};
                    
                    % Step 3
                    \draw[PrimaryColor] (-1,0) rectangle (2,0.5);
                    \node[left] at (-1.2,0.25) {\scriptsize 3.};
                    \node[right] at (2.1,0.25) {\scriptsize B: 0.7 > 0.3 → discard};
                \end{scope>
            \end{tikzpicture>
        \end{column>
    \end{columns>
\end{frame>

\begin{frame}>{Early-Z and Hierarchical Z-Buffer}
    \begin{columns}
        \begin{column}>{0.5\textwidth>
            \begin{conceptbox}>{Early-Z Optimization}
                \textbf{Test depth before fragment shader:}
                \begin{itemize}
                    \item Save expensive fragment processing
                    \item Requires depth not modified in shader
                    \item Massive performance gain for complex shaders
                \end{itemize>
                
                \vspace{0.3cm}
                \textbf{Traditional:} Raster → Fragment → Depth \\
                \textbf{Early-Z:} Raster → Depth → Fragment
            \end{conceptbox>
            
            \pause
            \begin{raybox}>{Hierarchical Z-Buffer}
                \textbf{Multi-resolution depth testing:}
                \begin{itemize}
                    \item Test large blocks first
                    \item Reject entire regions quickly
                    \item Pyramid of depth values
                \end{itemize>
            \end{raybox>
        \end{column>
        \begin{column}>{0.5\textwidth>
            \begin{tikzpicture}[scale=0.8]
                % Early-Z illustration
                \node[above] at (0,4) {\textbf{Early-Z Pipeline}};
                
                \node[process, fill=AccentColor!20, minimum width=1.5cm] (raster) at (0,3) {Raster};
                \node[process, fill=RayColor!20, minimum width=1.5cm] (earlyz) at (0,2) {Early-Z};
                \node[process, fill=PrimaryColor!20, minimum width=1.5cm] (frag) at (0,1) {Fragment};
                \node[process, fill=LightGray, minimum width=1.5cm] (output) at (0,0) {Output};
                
                \draw[->, thick] (raster) -- (earlyz);
                \draw[->, thick] (earlyz) -- (frag);
                \draw[->, thick] (frag) -- (output);
                
                % Reject path
                \draw[->, red, dashed] (earlyz.east) -- ++(1.5,0) node[right] {\textcolor{red}{Reject}};
                
                % HiZ illustration
                \begin{scope}[shift={(3,-1)}]
                    \node[above] at (0,3) {\textbf{Hierarchical Z}};
                    
                    % Level 0 (full resolution)
                    \foreach \x in {0,0.3,0.6,0.9} {
                        \foreach \y in {0,0.3,0.6,0.9} {
                            \draw[gray] (\x,\y) rectangle ++(0.2,0.2);
                        }
                    }
                    \node[below] at (0.6,-0.3) {\scriptsize Level 0};
                    
                    % Level 1 (half resolution)
                    \foreach \x in {0,0.6} {
                        \foreach \y in {1.5,1.8} {
                            \draw[PrimaryColor] (\x,\y) rectangle ++(0.4,0.2);
                        }
                    }
                    \node[below] at (0.6,1.3) {\scriptsize Level 1};
                    
                    % Level 2 (quarter resolution)
                    \draw[AccentColor] (0,2.2) rectangle ++(0.8,0.4);
                    \node[below] at (0.4,2.1) {\scriptsize Level 2};
                \end{scope>
            \end{tikzpicture>
        \end{column>
    \end{columns>
\end{frame>

\begin{frame}>{Z-Fighting and Precision Issues}
    \begin{columns>
        \begin{column}>{0.6\textwidth>
            \begin{conceptbox}>{The Z-Fighting Problem}
                \textbf{When two surfaces are very close:}
                \begin{itemize>
                    \item Limited depth buffer precision
                    \item Floating point rounding errors
                    \item Flickering between surfaces
                    \item Artifacts at grazing angles
                \end{itemize>
                
                \vspace{0.3cm>
                \textbf{Solutions:}
                \begin{itemize>
                    \item Polygon offset
                    \item Better depth distribution
                    \item Reverse-Z for better precision
                \end{itemize>
            \end{conceptbox>
        \end{column>
        \begin{column}>{0.4\textwidth>
            \begin{tikzpicture}[scale=0.8]
                % Z-fighting visualization
                \node[above] at (0,3) {\textbf{Z-Fighting}};
                
                % Two very close surfaces
                \draw[thick, PrimaryColor] (-1,1.5) -- (1,1.5);
                \draw[thick, AccentColor, dashed] (-1,1.52) -- (1,1.52);
                \node[right] at (1.1,1.5) {\scriptsize Surface A};
                \node[right] at (1.1,1.52) {\scriptsize Surface B};
                
                % Precision illustration
                \draw[<->] (-0.5,1.5) -- (-0.5,1.52);
                \node[left] at (-0.6,1.51) {\scriptsize ≈ Precision};
                
                % Result
                \begin{scope}[shift={(0,-1.5)}]
                    \node[above] at (0,1) {\textbf{Result}};
                    \foreach \x in {-1,-0.7,-0.4,-0.1,0.2,0.5,0.8} {
                        \pgfmathparse{int(2*rnd)}
                        \ifnum\pgfmathresult=0
                            \fill[PrimaryColor] (\x,0) rectangle ++(0.2,0.3);
                        \else
                            \fill[AccentColor] (\x,0) rectangle ++(0.2,0.3);
                        \fi
                    }
                    \node[below] at (0,-0.3) {\scriptsize Random flickering};
                \end{scope>
            \end{tikzpicture>
        \end{column>
    \end{columns>
    
    \pause
    \begin{mathbox>>{Reverse-Z for Better Precision}
        \footnotesize
        \textbf{Traditional:} $z \in [0, 1]$ with poor precision near far plane \\
        \textbf{Reverse-Z:} $z \in [1, 0]$ with better precision distribution
        
        \begin{center}
            \begin{tikzpicture}[scale=0.6]
                % Traditional Z
                \draw[->] (0,0) -- (3,0) node[right] {Distance};
                \draw[->] (0,0) -- (0,2) node[above] {Precision};
                \draw[PrimaryColor, thick] (0.2,1.5) -- (2.8,0.2);
                \node[below] at (1.5,-0.3) {\textbf{Traditional Z}};
                \node[right] at (3,1) {\scriptsize Poor at far};
                
                % Reverse Z
                \begin{scope}[shift={(4,0)}]
                    \draw[->] (0,0) -- (3,0) node[right] {Distance};
                    \draw[->] (0,0) -- (0,2) node[above] {Precision};
                    \draw[AccentColor, thick] (0.2,0.2) -- (2.8,1.5);
                    \node[below] at (1.5,-0.3) {\textbf{Reverse Z}};
                    \node[right] at (3,1) {\scriptsize Better distribution};
                \end{scope>
            \end{tikzpicture>
        \end{center>
    \end{mathbox>
\end{frame>
