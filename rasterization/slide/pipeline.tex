\section{The Modern Graphics Pipeline}

\begin{frame}{The Big Picture: 3D to Pixels}
  \begin{center}
    \begin{tikzpicture}[scale=0.8, node distance=1.25cm]
      % Input
      \node[process, fill=ObjectColor!20] (input) {3D Models};

      % GPU Pipeline stages
      \node[process, below of=input, fill=PrimaryColor!20] (vertex) {Vertex Processing};
      \node[process, below of=vertex, fill=SecondaryColor!20] (assembly) {Primitive Assembly};
      \node[process, below of=assembly, fill=AccentColor!20] (raster) {Rasterization};
      \node[process, below of=raster, fill=PrimaryColor!20] (fragment) {Fragment Processing};
      \node[process, below of=fragment, fill=LightGray] (output) {Frame Buffer};

      % Arrows
      \draw[arrow] (input) -- (vertex);
      \draw[arrow] (vertex) -- (assembly);
      \draw[arrow] (assembly) -- (raster);
      \draw[arrow] (raster) -- (fragment);
      \draw[arrow] (fragment) -- (output);

      % Side annotations
      \begin{scope}[shift={(-3,-0.25)}]
        \node[right, text width=3cm] at (6,0) {
          \footnotesize \textcolor{ObjectColor}{\textbf{Application}} \\
          \scriptsize Meshes, textures, shaders
        };
        \node[right, text width=3cm] at (6,-1.5) {
          \footnotesize \textcolor{PrimaryColor}{\textbf{Programmable}} \\
          \scriptsize Transform vertices
        };
        \node[right, text width=3cm] at (6,-3) {
          \footnotesize \textcolor{SecondaryColor}{\textbf{Fixed Function}} \\
          \scriptsize Group into triangles
        };
        \node[right, text width=3cm] at (6,-4.5) {
          \footnotesize \textcolor{AccentColor}{\textbf{Fixed Function}} \\
          \scriptsize Convert to pixels
        };
        \node[right, text width=3cm] at (6,-6) {
          \footnotesize \textcolor{PrimaryColor}{\textbf{Programmable}} \\
          \scriptsize Compute pixel colors
        };
        \node[right, text width=3cm] at (6,-7.5) {
          \footnotesize \textcolor{DarkGray}{\textbf{Display}} \\
          \scriptsize Final image
        };
      \end{scope}
    \end{tikzpicture}
  \end{center}
\end{frame}

\begin{frame}{Modern GPU Pipeline Architecture}
  \begin{center}
    \footnotesize
    \begin{tikzpicture}[scale=0.65]
      % Input Assembly
      \node[rectangle, draw, fill=ObjectColor!20, minimum width=2cm, minimum height=1cm] (input) at (0,8) {
        \begin{minipage}{1.8cm}
          \centering
          \textbf{Input} \\
          \textbf{Assembly}
        \end{minipage}
      };

      % Vertex Shader
      \node[rectangle, draw, fill=PrimaryColor!30, minimum width=2cm, minimum height=1cm] (vs) at (0,6) {
        \begin{minipage}{1.8cm}
          \centering
          \textbf{Vertex} \\
          \textbf{Shader}
        \end{minipage}
      };

      % Tessellation (optional)
      \node[rectangle, draw, fill=SecondaryColor!20, minimum width=2cm, minimum height=1cm] (tess) at (4,6) {
        \begin{minipage}{1.8cm}
          \centering
          \textbf{Tessellation} \\
          \scriptsize (Optional)
        \end{minipage}
      };

      % Geometry Shader (optional)
      \node[rectangle, draw, fill=AccentColor!20, minimum width=2cm, minimum height=1cm] (gs) at (8,6) {
        \begin{minipage}{1.8cm}
          \centering
          \textbf{Geometry} \\
          \textbf{Shader} \\
          \scriptsize (Optional)
        \end{minipage}
      };

      % Clipping & Culling
      \node[rectangle, draw, fill=LightGray, minimum width=2cm, minimum height=1cm] (clip) at (4,4) {
        \begin{minipage}{1.8cm}
          \centering
          \textbf{Clipping \&} \\
          \textbf{Culling}
        \end{minipage}
      };

      % Rasterization
      \node[rectangle, draw, fill=RayColor!20, minimum width=2cm, minimum height=1cm] (raster) at (4,2) {
        \begin{minipage}{1.8cm}
          \centering
          \textbf{Rasterization}
        \end{minipage}
      };

      % Fragment Shader
      \node[rectangle, draw, fill=PrimaryColor!30, minimum width=2cm, minimum height=1cm] (fs) at (4,0) {
        \begin{minipage}{1.8cm}
          \centering
          \textbf{Fragment} \\
          \textbf{Shader}
        \end{minipage}
      };

      % Per-Fragment Ops
      \node[rectangle, draw, fill=LightGray, minimum width=2cm, minimum height=1cm] (ops) at (8,0) {
        \begin{minipage}{1.8cm}
          \centering
          \textbf{Per-Fragment} \\
          \textbf{Operations}
        \end{minipage}
      };

      % Frame Buffer
      \node[rectangle, draw, fill=DarkGray!20, minimum width=2cm, minimum height=1cm] (fb) at (4,-2) {
        \begin{minipage}{1.8cm}
          \centering
          \textbf{Frame Buffer}
        \end{minipage}
      };

      % Arrows - main path
      \draw[->, thick, PrimaryColor] (input) -- (vs);
      \draw[->, thick] (vs) -- (tess);
      \draw[->, thick] (tess) -- (gs);
      \draw[->, thick] (gs) -- (clip);
      \draw[->, thick] (clip) -- (raster);
      \draw[->, thick] (raster) -- (fs);
      \draw[->, thick] (fs) -- (ops);
      \draw[->, thick] (ops) -- (fb);

      % Optional bypasses
      \draw[->, dashed, gray] (vs) to[bend left=20] (clip);
      \draw[->, dashed, gray] (tess) to[bend left=10] (clip);

      % Annotations
      \node[left, text width=2cm] at (-1,6) {\scriptsize \textcolor{PrimaryColor}{\textbf{Programmable}}};
      \node[right, text width=2cm] at (9,6) {\scriptsize \textcolor{SecondaryColor}{\textbf{Programmable}}};
      \node[right, text width=2cm] at (9,4) {\scriptsize \textcolor{DarkGray}{\textbf{Configurable}}};
      \node[right, text width=2cm] at (6,2) {\scriptsize \textcolor{RayColor}{\textbf{Fixed Function}}};
      \node[left, text width=2cm] at (3,0) {\scriptsize \textcolor{PrimaryColor}{\textbf{Programmable}}};
      \node[right, text width=2cm] at (10,0) {\scriptsize \textcolor{DarkGray}{\textbf{Configurable}}};
    \end{tikzpicture}
  \end{center}
\end{frame}

\begin{frame}{Programmable vs Fixed Function Stages}
  \begin{columns}
    \begin{column}{0.5\textwidth}
      \begin{raybox}{Programmable Stages}
        \textbf{You write the code:}
        \begin{itemize}
          \item \textbf{Vertex Shader:} Transform positions, compute lighting
          \item \textbf{Tessellation:} Subdivide surfaces adaptively
          \item \textbf{Geometry Shader:} Generate/modify primitives
          \item \textbf{Fragment Shader:} Compute final pixel colors
        \end{itemize}

        \vspace{0.2cm}
        \textcolor{PrimaryColor}{\textbf{Maximum flexibility}}
      \end{raybox}
    \end{column}
    \begin{column}{0.5\textwidth}
      \begin{conceptbox}{Fixed Function Stages}
        \textbf{Hardware handles it:}
        \begin{itemize}
          \item \textbf{Primitive Assembly:} Group vertices into triangles
          \item \textbf{Clipping:} Remove off-screen geometry
          \item \textbf{Rasterization:} Convert triangles to pixels
          \item \textbf{Depth Testing:} Z-buffer comparisons
        \end{itemize}

        \vspace{0.2cm}
        \textcolor{SecondaryColor}{\textbf{Maximum performance}}
      \end{conceptbox}
    \end{column}
  \end{columns}

  \pause
  \vspace{0.5cm}
  \begin{center}
    \begin{tikzpicture}[scale=0.8]
      \draw[->, very thick, PrimaryColor] (0,0) -- (4,0) node[midway, above] {Data Flow};
      \node[below] at (0,-0.3) {\scriptsize Vertices};
      \node[below] at (1,-0.3) {\scriptsize Primitives};
      \node[below] at (2,-0.3) {\scriptsize Fragments};
      \node[below] at (3,-0.3) {\scriptsize Pixels};
      \node[below] at (4,-0.3) {\scriptsize Frame};
    \end{tikzpicture}
  \end{center}
\end{frame}

\begin{frame}{Data Flow Through the Pipeline}
  \begin{center}
    \begin{tikzpicture}[scale=0.7]
      % Data representations at each stage
      \node[above] at (0,3) {\textbf{Input Data}};
      \node[below] at (0,2.5) {
        \begin{tikzpicture}[scale=0.3]
          % 3D vertices
          \draw[thick, ObjectColor] (0,0,0) -- (2,0,0) -- (1,2,0) -- cycle;
          \fill[ObjectColor] (0,0,0) circle (0.1);
          \fill[ObjectColor] (2,0,0) circle (0.1);
          \fill[ObjectColor] (1,2,0) circle (0.1);
        \end{tikzpicture}
      };
      \node[below] at (0,1.5) {\scriptsize 3D Vertices};

      \draw[->, thick] (1,2.5) -- (2,2.5);

      \node[above] at (3,3) {\textbf{After Vertex}};
      \node[below] at (3,2.5) {
        \begin{tikzpicture}[scale=0.3]
          % Transformed vertices
          \draw[thick, PrimaryColor] (0,0) -- (2,0) -- (1,1.5) -- cycle;
          \fill[PrimaryColor] (0,0) circle (0.1);
          \fill[PrimaryColor] (2,0) circle (0.1);
          \fill[PrimaryColor] (1,1.5) circle (0.1);
        \end{tikzpicture}
      };
      \node[below] at (3,1.5) {\scriptsize Screen Space};

      \draw[->, thick] (4,2.5) -- (5,2.5);

      \node[above] at (6,3) {\textbf{After Raster}};
      \node[below] at (6,2.5) {
        \begin{tikzpicture}[scale=0.3]
          % Rasterized pixels
          \foreach \x in {0,0.3,0.6,0.9,1.2} {
            \foreach \y in {0,0.3,0.6} {
              \fill[AccentColor] (\x,\y) rectangle ++(0.2,0.2);
            }
          }
        \end{tikzpicture}
      };
      \node[below] at (6,1.5) {\scriptsize Fragments};

      \draw[->, thick] (7,2.5) -- (8,2.5);

      \node[above] at (9,3) {\textbf{Final Image}};
      \node[below] at (9,2.5) {
        \begin{tikzpicture}[scale=0.3]
          % Final colored pixels
          \foreach \x in {0,0.3,0.6,0.9,1.2} {
            \foreach \y in {0,0.3,0.6} {
              \pgfmathsetmacro{\shade}{20 + 40*rnd}
              \fill[PrimaryColor!\shade] (\x,\y) rectangle ++(0.2,0.2);
            }
          }
        \end{tikzpicture}
      };
      \node[below] at (9,1.5) {\scriptsize Colored Pixels};

      % Stage labels
      \node[below] at (1.5,1) {\textcolor{PrimaryColor}{\textbf{Vertex}}};
      \node[below] at (4.5,1) {\textcolor{SecondaryColor}{\textbf{Clipping}}};
      \node[below] at (7.5,1) {\textcolor{AccentColor}{\textbf{Rasterization}}};
      \node[below] at (8.5,0.5) {\textcolor{PrimaryColor}{\textbf{Fragment}}};
    \end{tikzpicture}
  \end{center}

  \pause
  \begin{conceptbox}{Key Insight}
    Each stage transforms data from one representation to another, optimized for the next stage's processing.
  \end{conceptbox}
\end{frame}
