\section{The Modern Graphics Pipeline}

\subsection{Definition}
% Slide 1: Overview of the Pipeline
\begin{frame}{Why a Graphics Pipeline?}
  \begin{raybox}{The Rasterization Process}
    \small
    \begin{itemize}
      \item<1-> GPU process vast numbers of \textbf{vertices} and \textbf{pixels} every frame \\
        (millions per second)
      \item<2-> Each undergoes \textbf{a series of identical operations}
      \item<3-> The operations can be divided into \textbf{stages}
      \item<4-> Each stage can be \textbf{efficiently} implemented in hardware
      \item<5-> Stages can be \textbf{parallelized} for high throughput
    \end{itemize}
  \end{raybox}
  \only<6->{
    \begin{conceptbox}{The Graphics Pipeline}
      The graphics pipeline is a sequence of stages that process vertices and fragments in parallel, transforming 3D models into 2D images.
    \end{conceptbox}
  }
\end{frame}

\subsection{Stages}
% Slide 2: High-Level Block Diagram
\begin{frame}{Pipeline Stages at a Glance}
  \small
  \begin{columns}[T]
    \begin{column}{0.3\textwidth}
      \begin{tikzpicture}[scale=0.7,node distance=0.5cm and 1cm,auto]
        \node<1->[draw, fill=ObjectColor!20] (ia) {Input Assembly};
        \node<2->[draw, fill=PrimaryColor!30, below=of ia] (vs) {Vertex Shader};
        \node<3->[draw, fill=SecondaryColor!20, below=of vs] (pa) {Primitive Assembly};
        \node<4->[draw, fill=LightGray, below=of pa] (rc) {Clipping \& Culling};
        \node<5->[draw, fill=RayColor!20, below=of rc] (rz) {Rasterization};
        \node<6->[draw, fill=PrimaryColor!30, below=of rz] (fs) {Fragment Shader};
        \node<7->[draw, fill=DarkGray!20, below=of fs] (fb) {Frame Buffer};
        \foreach \i/\j/\n in {ia/vs/2, vs/pa/3, pa/rc/4, rc/rz/5, rz/fs/6, fs/fb/7} {
        \only<\n->{\draw[->, thick] (\i) -- (\j);}  }
      \end{tikzpicture}
    \end{column}
    \begin{column}{0.7\textwidth}
      \only<1->{\textbf{Input Assembly:} Pull vertex data (positions, normals, UVs) into the pipeline. \\ \vspace{0.2cm}}
      \only<2->{\textbf{Vertex Shader:} \texttt{Programmable stage} — transform each vertex from model to clip space. \\ \vspace{0.2cm}}
      \only<3->{\textbf{Primitive Assembly:} Group vertices into triangles (or other primitives). \\ \vspace{0.2cm}}
      \only<4->{\textbf{Clipping \& Culling:} Discard or trim primitives outside the view frustum. \\ \vspace{0.2cm}}
      \only<5->{\textbf{Rasterization:} Convert triangles into a grid of fragments (potential pixels). \\ \vspace{0.2cm}}
      \only<6->{\textbf{Fragment Shader:} \texttt{Programmable stage} — compute final color of each fragment. \\ \vspace{0.2cm}}
      \only<7->{\textbf{Frame Buffer:} Blend, depth‑test, and write pixels to the screen.}
    \end{column}
  \end{columns}
\end{frame}

\subsection{Advanced Pipeline}
\begin{frame}{Modern Advanced Pipeline}
  \begin{center}
    \footnotesize
    \vspace*{-0.5cm}
    \begin{tikzpicture}[scale=0.6]
      % Input Assembly
      \node[rectangle, draw, fill=ObjectColor!20, minimum width=2cm, minimum height=1cm] (input) at (0,8.5) {
        \begin{minipage}{1.8cm}
          \centering
          \textbf{Input} \\
          \textbf{Assembly}
        \end{minipage}
      };

      % Vertex Shader
      \node[rectangle, draw, fill=PrimaryColor!30, minimum width=2cm, minimum height=1cm] (vs) at (0,6.5) {
        \begin{minipage}{1.8cm}
          \centering
          \textbf{Vertex} \\
          \textbf{Shader}
        \end{minipage}
      };

      % Tessellation (optional)
      \node[rectangle, draw, fill=SecondaryColor!20, minimum width=2cm, minimum height=1cm] (tess) at (4,6.5) {
        \begin{minipage}{1.8cm}
          \centering
          \textbf{Tessellation} \\
          \scriptsize (Optional)
        \end{minipage}
      };

      % Geometry Shader (optional)
      \node[rectangle, draw, fill=AccentColor!20, minimum width=2cm, minimum height=1cm] (gs) at (8,6.5) {
        \begin{minipage}{1.8cm}
          \centering
          \textbf{Geometry} \\
          \textbf{Shader} \\
          \scriptsize (Optional)
        \end{minipage}
      };

      % Primitive Assembly
      \node[rectangle, draw, fill=LightGray!40, minimum width=2.5cm, minimum height=0.75cm] (pa) at (4,4.5) {
        \begin{minipage}{2.5cm}
          \centering
          \textbf{Primitive Assembly}
        \end{minipage}
      };

      % Clipping & Culling
      \node[rectangle, draw, fill=LightGray, minimum width=2cm, minimum height=0.75cm] (clip) at (4,2.75) {
        \begin{minipage}{2.5cm}
          \centering
          \textbf{Clipping \& Culling}
        \end{minipage}
      };

      % Rasterization
      \node[rectangle, draw, fill=RayColor!20, minimum width=2cm, minimum height=0.75cm] (raster) at (4,1) {
        \begin{minipage}{1.8cm}
          \centering
          \textbf{Rasterization}
        \end{minipage}
      };

      % Fragment Shader
      \node[rectangle, draw, fill=PrimaryColor!30, minimum width=2cm, minimum height=1cm] (fs) at (4,-1) {
        \begin{minipage}{1.8cm}
          \centering
          \textbf{Fragment} \\
          \textbf{Shader}
        \end{minipage}
      };

      % Per-Fragment Ops
      \node[rectangle, draw, fill=LightGray, minimum width=2cm, minimum height=1cm] (ops) at (8,-1) {
        \begin{minipage}{1.8cm}
          \centering
          \textbf{Per-Fragment} \\
          \textbf{Operations}
        \end{minipage}
      };

      % Frame Buffer
      \node[rectangle, draw, fill=DarkGray!20, minimum width=2cm, minimum height=0.75cm] (fb) at (4,-3) {
        \begin{minipage}{1.8cm}
          \centering
          \textbf{Frame Buffer}
        \end{minipage}
      };

      % Arrows - main path
      \draw[->, thick, PrimaryColor] (input) -- (vs);
      \draw[->, thick] (vs) -- (tess);
      \draw[->, thick] (tess) -- (gs);
      \draw[->, thick] (gs) to[bend left=20] (pa);
      \draw[->, thick] (pa) -- (clip);
      \draw[->, thick] (clip) -- (raster);
      \draw[->, thick] (raster) -- (fs);
      \draw[->, thick] (fs) -- (ops);
      \draw[->, thick] (ops) to[bend left=20] (fb);

      % Optional bypasses via Primitive Assembly
      \draw[->, dashed, gray] (vs) to[bend right=20] (pa);
      \draw[->, dashed, gray] (tess) to[bend left=10] (pa);

      % Annotations
      \node[left, text width=2cm] at (-1,6.5) {\scriptsize \textcolor{PrimaryColor}{\textbf{Programmable}}};
      \node[right, text width=2cm] at (9.75,6.5) {\scriptsize \textcolor{SecondaryColor}{\textbf{Programmable}}};
      \node[right, text width=2cm] at (6.25,2.75) {\scriptsize \textcolor{DarkGray}{\textbf{Configurable}}};
      \node[right, text width=2cm] at (5.75,1) {\scriptsize \textcolor{RayColor}{\textbf{Fixed Function}}};
      \node[left, text width=2cm] at (3,-1) {\scriptsize \textcolor{PrimaryColor}{\textbf{Programmable}}};
      \node[right, text width=2cm] at (9.75,-1) {\scriptsize \textcolor{DarkGray}{\textbf{Configurable}}};
    \end{tikzpicture}
  \end{center}
\end{frame}

\subsection{Data Flow Through the Pipeline}
\begin{frame}{Data Flow Through the Pipeline}
  \begin{center}
    \footnotesize
    \begin{tikzpicture}[scale=0.6]
      % Stage 1: Input Data (3D vertices)
      \node (inputLabel) at (0,3) {\textbf{Input Data}};
      \node (inputPic) at (0,1.5) {%
        \begin{tikzpicture}[scale=0.3]
          \draw[thick, ObjectColor] (0,0,0) -- (2,0,0) -- (1,2,0) -- cycle;
          \foreach \p in {(0,0,0),(2,0,0),(1,2,0)} {\fill[ObjectColor] \p circle (0.1);}
        \end{tikzpicture}%
      };
      \node at (0,0) {\scriptsize 3D Vertices};
      \pause

      \draw[->, thick] (1.5,1.5) -- (2.5,1.5);
      % Stage 2: After Vertex (screen space, partially out of bounds)
      \node (vertexLabel) at (4,3) {\textbf{Vertex Stage}};
      \node (vertexPic) at (4,1.5) {%
        \begin{tikzpicture}[scale=0.3]
          % draw screen boundary
          \draw[thin, orange] (0,0) rectangle (3,2);
          % transformed triangle, partly outside
          \draw[thick, PrimaryColor] (-0.5,1) -- (2.2,0.2) -- (1,2.5) -- cycle;
          \foreach \p in {(-0.5,1),(2.2,0.2),(1,2.5)} {\fill[PrimaryColor] \p circle (0.1);}
        \end{tikzpicture}%
      };
      \node[align=center] at (4,0) {\scriptsize Screen Space \\ (Unclipped)};

      \pause
      \draw[->, thick] (5.5,1.5) -- (6.5,1.5);
      % Stage 3: After Clipping
      \node (clipLabel) at (8,3) {\textbf{Clipping}};
      \node (clipPic) at (8,1.5) {%
        \begin{tikzpicture}[scale=0.3]
          % screen boundary
          \draw[thin, orange] (0,0) rectangle (3,2);
          % clipped triangle
          \clip (0,0) rectangle (3,2);
          \draw[thick, SecondaryColor] (-0.5,1) -- (2.2,0.2) -- (1,2.5) -- cycle;
          \foreach \p in {(-0.5,1),(2.2,0.2),(1,2.5)} {\fill[SecondaryColor] \p circle (0.1);}
        \end{tikzpicture}%
      };
      \node[align=center] at (8,0) {\scriptsize Screen Space \\ (Clipped)};
      \pause
      \draw[->, thick] (9.5,1.5) -- (10.5,1.5);
      % Stage 4: Rasterization (Fragments in/out)
      \node (rasterLabel) at (12,3) {\textbf{Rasterization}};
      \node (rasterPic) at (12,1.5) {%
        \begin{tikzpicture}[scale=0.3]
          % draw screen boundary
          \draw[thin, orange] (0,0) rectangle (3,2);
          % draw base grid
          \foreach \x in {0,0.25,...,2.75} {
            \foreach \y in {0,0.25,...,1.75} {
              \fill[gray!20] ($(\x,\y)+(0.025, 0.025)$) rectangle ++(0.2,0.2);
            }
          }
          % shade cells inside triangle
          \begin{scope}
            \clip (-0.5,1) -- (2.2,0.2) -- (1,2.5) -- cycle;
            \foreach \x in {0,0.25,...,2.75} {
              \foreach \y in {0,0.25,...,1.75} {
                \fill[AccentColor!40] ($(\x,\y)+(0.025, 0.025)$) rectangle ++(0.2,0.2);
              }
            }
          \end{scope}
        \end{tikzpicture}%
      };
      \node[align=center] at (12,0) {\scriptsize Inside/Outside\\ Fragments};

      \pause
      \draw[->, thick] (13.5,1.5) -- (14.5,1.5);

      % Stage 5: Fragment Stage (Colored by in/out)
      \node (fragLabel) at (16,3) {\textbf{Fragment Stage}};
      \node (fragPic) at (16,1.5) {%
        \begin{tikzpicture}[scale=0.3]
          % draw screen boundary
          \draw[thin, orange] (0,0) rectangle (3,2);
          % base grid for outside
          \foreach \x in {0,0.25,...,2.75} {
            \foreach \y in {0,0.25,...,1.75} {
              \fill[AccentColor!20] ($(\x,\y)+(0.025, 0.025)$) rectangle ++(0.2,0.2);
            }
          }
          % fill inside triangle fragments
          \begin{scope}
            \clip (-0.5,1) -- (2.2,0.2) -- (1,2.5) -- cycle;
            \foreach \x in {0,0.25,...,2.75} {
              \foreach \y in {0,0.25,...,1.75} {
                \fill[PrimaryColor!60] ($(\x,\y)+(0.025, 0.025)$) rectangle ++(0.2,0.2);
              }
            }
          \end{scope}
        \end{tikzpicture}%
      };
      \node[align=center] at (16,0) {\scriptsize Colored \\ Fragments};

      % Arrows between stages

    \end{tikzpicture}
  \end{center}
\end{frame}

\subsection{Pipeline Stage Types}
\begin{frame}{Programmable vs Fixed Function Stages}
  \begin{columns}
    \small
    \begin{column}{0.5\textwidth}
      \begin{raybox}{Programmable Stages}
        \textbf{You write the code:}
        \begin{itemize}
          \item \textbf{Vertex Shader:} Transform positions, compute lighting
          \item \textbf{Tessellation:} Subdivide surfaces adaptively
          \item \textbf{Geometry Shader:} Generate/modify primitives
          \item \textbf{Fragment Shader:} Compute final pixel colors
        \end{itemize}

        \vspace{0.2cm}
        \textcolor{PrimaryColor}{\textbf{Maximum flexibility}}
      \end{raybox}
    \end{column}
    \pause
    \begin{column}{0.5\textwidth}
      \begin{conceptbox}{Fixed Function Stages}
        \textbf{Hardware handles it:}
        \begin{itemize}
          \item \textbf{Primitive Assembly:} Group vertices into triangles
          \item \textbf{Clipping:} Remove off-screen geometry
          \item \textbf{Rasterization:} Convert triangles to pixels
          \item \textbf{Depth Testing:} Z-buffer comparisons
        \end{itemize}

        \vspace{0.2cm}
        \textcolor{SecondaryColor}{\textbf{Maximum performance}}
      \end{conceptbox}
    \end{column}
  \end{columns}
\end{frame}
