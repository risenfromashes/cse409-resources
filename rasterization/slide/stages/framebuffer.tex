\section{Framebuffer Output}

\subsection{Overview}
\begin{frame}{Framebuffer Output Stage}
  \begin{columns}
    \begin{column>{0.6\textwidth}
      \begin{raybox}{Framebuffer Output}
        \textbf{Input:} Final fragment colors and depth values \\
        \textbf{Output:} Pixels written to display or render targets
        
        \vspace{0.3cm}
        \textbf{Purpose:}
        \begin{itemize}
          \item Write final pixel data to memory
          \item Handle multiple render targets
          \item Manage color format conversions
          \item Support double buffering for smooth display
        \end{itemize>
        
        \vspace{0.2cm}
        \textcolor{DarkGray}{\textbf{Fixed-function stage}}
      \end{raybox>
    \end{column>
    \begin{column>{0.4\textwidth}
      \begin{tikzpicture}[scale=0.7]
        % Fragment data
        \node at (0,4) {\textbf{Fragment Data}};
        \node[rectangle, draw, fill=PrimaryColor!20, minimum width=1.5cm, minimum height=0.4cm] (color) at (0,3.5) {
          \tiny Color (RGBA)
        };
        \node[rectangle, draw, fill=SecondaryColor!20, minimum width=1.5cm, minimum height=0.4cm] (depth) at (0,3) {
          \tiny Depth (Z)
        };
        
        \draw[->, thick] (0,2.7) -- (0,2.3);
        
        % Framebuffer
        \node at (0,2) {\textbf{Framebuffer}};
        \node[rectangle, draw, fill=LightGray, minimum width=2cm, minimum height=1.2cm] (fb) at (0,1.2) {
          \begin{minipage>{1.8cm}
            \centering
            \tiny
            Color Buffer \\
            Depth Buffer \\
            Stencil Buffer
          \end{minipage>
        };
        
        \draw[->, thick] (0,0.5) -- (0,0.1);
        
        % Display
        \node[rectangle, draw, fill=AccentColor!20, minimum width=1.5cm, minimum height=0.4cm] (display) at (0,-0.3) {
          \tiny Display
        };
      \end{tikzpicture>
    \end{column>
  \end{columns>
\end{frame>

\subsection{Framebuffer Components}
\begin{frame}{Framebuffer Structure}
  \begin{columns}
    \begin{column>{0.5\textwidth}
      \begin{conceptbox}{Buffer Types}
        \textbf{Color Buffer:}
        \begin{itemize>
          \item Stores final pixel colors
          \item Multiple render targets (MRT) support
          \item Various formats: RGBA8, RGBA16F, etc.
        \end{itemize>
        
        \textbf{Depth Buffer:}
        \begin{itemize>
          \item Stores depth values for Z-testing
          \item Typically 16, 24, or 32-bit precision
          \item Essential for 3D rendering
        \end{itemize>
        
        \textbf{Stencil Buffer:}
        \begin{itemize>
          \item 8-bit per-pixel values
          \item Often packed with depth (D24S8)
          \item Used for advanced effects
        \end{itemize>
      \end{conceptbox>
    \end{column>
    \begin{column>{0.5\textwidth}
      \begin{tikzpicture}[scale=0.7]
        % Framebuffer visualization
        \node at (0,3.5) {\textbf{Framebuffer Layout}};
        
        % Color buffer
        \draw[thick, PrimaryColor] (0,2.5) rectangle (2.5,3);
        \node[left] at (-0.2,2.75) {\scriptsize Color};
        \foreach \x in {0.1,0.3,...,2.3} {
          \foreach \y in {2.6,2.8} {
            \fill[PrimaryColor!60] (\x,\y) rectangle ++(0.15,0.15);
          }
        }
        
        % Depth buffer
        \draw[thick, SecondaryColor] (0,1.8) rectangle (2.5,2.3);
        \node[left] at (-0.2,2.05) {\scriptsize Depth};
        \foreach \x in {0.1,0.3,...,2.3} {
          \foreach \y in {1.9,2.1} {
            \pgfmathparse{rnd}
            \fill[SecondaryColor!\pgfmathresult*60+20] (\x,\y) rectangle ++(0.15,0.15);
          }
        }
        
        % Stencil buffer
        \draw[thick, AccentColor] (0,1.1) rectangle (2.5,1.6);
        \node[left] at (-0.2,1.35) {\scriptsize Stencil};
        \foreach \x in {0.1,0.3,...,2.3} {
          \foreach \y in {1.2,1.4} {
            \fill[AccentColor!40] (\x,\y) rectangle ++(0.15,0.15);
          }
        }
        
        % Pixel coordinates
        \node[below] at (1.25,0.8) {\scriptsize Pixel (x, y)};
        \draw[dashed] (1.25,1.1) -- (1.25,3);
      \end{tikzpicture>
    \end{column>
  \end{columns>
\end{frame>

\subsection{Double Buffering}
\begin{frame}{Double Buffering}
  \begin{columns}
    \begin{column>{0.6\textwidth}
      \begin{raybox}{Double Buffering Concept}
        \textbf{Problem without double buffering:}
        \begin{itemize>
          \item Rendering directly to display buffer
          \item Visible tearing and flickering
          \item Partial frame updates visible to user
        \end{itemize>
        
        \vspace{0.3cm}
        \textbf{Double buffering solution:}
        \begin{enumerate>
          \item Render to back buffer (off-screen)
          \item When frame complete, swap buffers
          \item Front buffer displayed while back buffer rendered
          \item Atomic swap prevents tearing
        \end{enumerate>
        
        \vspace{0.2cm}
        \textbf{Result:} Smooth, tear-free animation
      \end{raybox>
    \end{column>
    \begin{column>{0.4\textwidth}
      \begin{tikzpicture}[scale=0.6]
        % Timeline showing buffer swaps
        \node at (0,4) {\textbf{Buffer Swapping}};
        
        % Frame 1
        \node at (0,3.2) {\scriptsize Frame 1};
        \node[rectangle, draw, fill=PrimaryColor!30, minimum width=0.8cm, minimum height=0.4cm] (front1) at (-0.8,2.8) {\tiny Front};
        \node[rectangle, draw, fill=SecondaryColor!30, minimum width=0.8cm, minimum height=0.4cm] (back1) at (0.8,2.8) {\tiny Back};
        \node[above] at (-0.8,3.2) {\tiny Display};
        \node[above] at (0.8,3.2) {\tiny Render};
        
        \draw[->, thick] (0,2.5) -- (0,2.1);
        \node[right] at (0.2,2.3) {\scriptsize Swap};
        
        % Frame 2
        \node at (0,1.7) {\scriptsize Frame 2};
        \node[rectangle, draw, fill=SecondaryColor!30, minimum width=0.8cm, minimum height=0.4cm] (front2) at (-0.8,1.3) {\tiny Front};
        \node[rectangle, draw, fill=PrimaryColor!30, minimum width=0.8cm, minimum height=0.4cm] (back2) at (0.8,1.3) {\tiny Back};
        \node[above] at (-0.8,1.7) {\tiny Display};
        \node[above] at (0.8,1.7) {\tiny Render};
        
        \draw[->, thick] (0,1) -- (0,0.6);
        \node[right] at (0.2,0.8) {\scriptsize Swap};
        
        % Frame 3
        \node at (0,0.2) {\scriptsize Frame 3};
        \node[rectangle, draw, fill=PrimaryColor!30, minimum width=0.8cm, minimum height=0.4cm] (front3) at (-0.8,-0.2) {\tiny Front};
        \node[rectangle, draw, fill=SecondaryColor!30, minimum width=0.8cm, minimum height=0.4cm] (back3) at (0.8,-0.2) {\tiny Back};
        \node[above] at (-0.8,0.2) {\tiny Display};
        \node[above] at (0.8,0.2) {\tiny Render};
      \end{tikzpicture>
    \end{column>
  \end{columns>
\end{frame>

\begin{frame}{VSync \& Refresh Rates}
  \begin{columns}
    \begin{column>{0.5\textwidth}
      \begin{conceptbox}{Vertical Synchronization}
        \textbf{VSync purpose:}
        \begin{itemize>
          \item Synchronize buffer swaps with display refresh
          \item Prevent screen tearing
          \item Limit frame rate to display refresh rate
        \end{itemize>
        
        \vspace{0.3cm}
        \textbf{Common refresh rates:}
        \begin{itemize>
          \item 60 Hz (16.67 ms per frame)
          \item 120 Hz (8.33 ms per frame)
          \item 144 Hz (6.94 ms per frame)
          \item Variable refresh (G-Sync, FreeSync)
        \end{itemize>
      \end{conceptbox>
    \end{column>
    \begin{column>{0.5\textwidth}
      \begin{mathbox}{VSync Trade-offs}
        \textbf{With VSync enabled:}
        \begin{itemize>
          \item \textcolor{SecondaryColor}{\textbf{Pro:}} No tearing
          \item \textcolor{SecondaryColor}{\textbf{Pro:}} Consistent frame timing
          \item \textcolor{AccentColor}{\textbf{Con:}} Input latency increase
          \item \textcolor{AccentColor}{\textbf{Con:}} Frame rate capped
        \end{itemize>
        
        \vspace{0.2cm}
        \textbf{Without VSync:}
        \begin{itemize}
          \item \textcolor{SecondaryColor}{\textbf{Pro:}} Lower input latency
          \item \textcolor{SecondaryColor}{\textbf{Pro:}} Uncapped frame rate
          \item \textcolor{AccentColor}{\textbf{Con:}} Screen tearing
          \item \textcolor{AccentColor}{\textbf{Con:}} Inconsistent timing
        \end{itemize>
      \end{mathbox>
    \end{column>
  \end{columns>
  
  \vspace{0.3cm}
  
  \begin{tikzpicture}
    % VSync timing diagram
    \node at (0,0.5) {\textbf{VSync Timing}};
    
    % Display refresh
    \draw[thick, LightGray] (0,0) -- (6,0);
    \foreach \x in {0,1,2,3,4,5,6} {
      \draw[thick, LightGray] (\x,-0.1) -- (\x,0.1);
    }
    \node[below] at (3,-0.3) {\scriptsize Display Refresh (60 Hz)};
    
    % GPU frame completion
    \draw[thick, PrimaryColor] (0,-0.7) -- (6,-0.7);
    \fill[PrimaryColor] (0.8,-0.7) circle (2pt);
    \fill[PrimaryColor] (1.6,-0.7) circle (2pt);
    \fill[PrimaryColor] (2.9,-0.7) circle (2pt);
    \fill[PrimaryColor] (4.1,-0.7) circle (2pt);
    \fill[PrimaryColor] (5.2,-0.7) circle (2pt);
    \node[below] at (3,-1) {\scriptsize GPU Frame Completion};
    
    % Buffer swaps (aligned with refresh)
    \draw[->, thick, AccentColor] (1,-0.5) -- (1,-0.2);
    \draw[->, thick, AccentColor] (3,-0.5) -- (3,-0.2);
    \draw[->, thick, AccentColor] (5,-0.5) -- (5,-0.2);
    \node[above] at (3,0.3) {\scriptsize Buffer Swaps};
  \end{tikzpicture>
\end{frame>

\subsection{Render Targets}
\begin{frame}{Multiple Render Targets (MRT)}
  \begin{columns}
    \begin{column>{0.5\textwidth}
      \begin{raybox}{Multiple Render Targets}
        \textbf{Render to multiple buffers simultaneously:}
        \begin{itemize>
          \item Fragment shader outputs multiple colors
          \item Each output goes to different render target
          \item Efficient for deferred rendering
          \item Single geometry pass, multiple outputs
        \end{itemize>
        
        \vspace{0.3cm}
        \textbf{Common applications:}
        \begin{itemize}
          \item G-buffer for deferred shading
          \item Motion vectors for effects
          \item Normal maps for post-processing
          \item Depth peeling for transparency
        \end{itemize>
      \end{raybox>
    \end{column>
    \begin{column>{0.5\textwidth}
      \begin{tikzpicture}[scale=0.6]
        % Fragment shader
        \node[rectangle, draw, fill=PrimaryColor!30, minimum width=1.5cm, minimum height=0.8cm] (fs) at (0,3) {
          \begin{minipage>{1.3cm}
            \centering
            \textbf{Fragment} \\
            \textbf{Shader}
          \end{minipage>
        };
        
        % Multiple outputs
        \draw[->, thick] (fs) -- (2,3.4);
        \draw[->, thick] (fs) -- (2,3);
        \draw[->, thick] (fs) -- (2,2.6);
        
        % Render targets
        \node[rectangle, draw, fill=ObjectColor!20, minimum width=1cm, minimum height=0.5cm] (rt0) at (3,3.4) {\tiny Albedo};
        \node[rectangle, draw, fill=SecondaryColor!20, minimum width=1cm, minimum height=0.5cm] (rt1) at (3,3) {\tiny Normal};
        \node[rectangle, draw, fill=AccentColor!20, minimum width=1cm, minimum height=0.5cm] (rt2) at (3,2.6) {\tiny Depth};
        
        % Labels
        \node[left] at (1.8,3.4) {\scriptsize Color0};
        \node[left] at (1.8,3) {\scriptsize Color1};
        \node[left] at (1.8,2.6) {\scriptsize Color2};
        
        \node[below] at (1.5,2.2) {\scriptsize Deferred Shading G-Buffer};
      \end{tikzpicture>
    \end{column>
  \end{columns>
\end{frame>

\begin{frame>[fragile]{OpenGL Framebuffer Operations}
  \begin{columns}
    \begin{column>{0.5\textwidth}
      \begin{minted}[fontsize=\tiny, bgcolor=LightGray!20]{c}
// Clear framebuffer
glClear(GL_COLOR_BUFFER_BIT | GL_DEPTH_BUFFER_BIT | GL_STENCIL_BUFFER_BIT);

// Set clear colors
glClearColor(0.0f, 0.0f, 0.0f, 1.0f);  // Black
glClearDepth(1.0f);                     // Far plane
glClearStencil(0);                      // Zero

// Control which buffers are written
glColorMask(GL_TRUE, GL_TRUE, GL_TRUE, GL_TRUE);  // RGBA
glDepthMask(GL_TRUE);                             // Depth
glStencilMask(0xFF);                              // Stencil

// Viewport transformation
glViewport(0, 0, windowWidth, windowHeight);

// Buffer swapping (using GLFW)
glfwSwapBuffers(window);

// VSync control
glfwSwapInterval(1);  // Enable VSync
glfwSwapInterval(0);  // Disable VSync
      \end{minted>
    \end{column>
    \begin{column>{0.5\textwidth}
      \begin{minted}[fontsize=\tiny, bgcolor=LightGray!20]{c}
// Creating custom framebuffer
unsigned int framebuffer;
glGenFramebuffers(1, &framebuffer);
glBindFramebuffer(GL_FRAMEBUFFER, framebuffer);

// Color attachment
unsigned int colorTexture;
glGenTextures(1, &colorTexture);
glBindTexture(GL_TEXTURE_2D, colorTexture);
glTexImage2D(GL_TEXTURE_2D, 0, GL_RGB, width, height, 0, 
             GL_RGB, GL_UNSIGNED_BYTE, NULL);
glFramebufferTexture2D(GL_FRAMEBUFFER, GL_COLOR_ATTACHMENT0, 
                       GL_TEXTURE_2D, colorTexture, 0);

// Depth attachment
unsigned int depthTexture;
glGenTextures(1, &depthTexture);
glBindTexture(GL_TEXTURE_2D, depthTexture);
glTexImage2D(GL_TEXTURE_2D, 0, GL_DEPTH_COMPONENT24, width, height, 0,
             GL_DEPTH_COMPONENT, GL_FLOAT, NULL);
glFramebufferTexture2D(GL_FRAMEBUFFER, GL_DEPTH_ATTACHMENT,
                       GL_TEXTURE_2D, depthTexture, 0);

// Check completeness
if (glCheckFramebufferStatus(GL_FRAMEBUFFER) != GL_FRAMEBUFFER_COMPLETE) {
    // Handle error
}

// Render to custom framebuffer
glBindFramebuffer(GL_FRAMEBUFFER, framebuffer);
// ... render scene ...
glBindFramebuffer(GL_FRAMEBUFFER, 0);  // Back to default
      \end{minted>
    \end{column>
  \end{columns>
\end{frame>

\subsection{Performance Considerations}
\begin{frame>{Framebuffer Performance}
  \begin{columns}
    \begin{column>{0.5\textwidth}
      \begin{conceptbox}{Memory Bandwidth}
        \textbf{Framebuffer operations are bandwidth-intensive:}
        \begin{itemize>
          \item High resolution increases bandwidth needs
          \item Multiple render targets multiply bandwidth
          \item Color format affects bandwidth (RGBA8 vs RGBA16F)
          \item Depth/stencil adds additional overhead
        \end{itemize}
        
        \vspace{0.2cm}
        \textbf{Example (1920×1080):}
        \begin{itemize}
          \item RGBA8: 8.3 MB per frame
          \item RGBA16F: 16.6 MB per frame
          \item With depth: +2.5 MB (D24S8)
          \item At 60 FPS: 500-1200 MB/s
        \end{itemize>
      \end{conceptbox>
    \end{column>
    \begin{column>{0.5\textwidth}
      \begin{raybox>{Optimization Strategies}
        \textbf{Reduce bandwidth:}
        \begin{itemize}
          \item Use appropriate color formats
          \item Minimize MRT count when possible
          \item Consider lower resolution for effects
          \item Use texture compression when suitable
        \end{itemize>
        
        \textbf{Tile-based rendering:}
        \begin{itemize>
          \item Process small screen tiles
          \item Keep working set in fast memory
          \item Reduce external memory bandwidth
          \item Common on mobile GPUs
        \end{itemize>
        
        \textbf{Early optimizations:}
        \begin{itemize>
          \item Early Z to skip fragment processing
          \item Occlusion culling to skip objects
          \item LOD to reduce triangle count
        \end{itemize>
      \end{raybox>
    \end{column>
  \end{columns>
\end{frame>

\begin{frame>{Framebuffer Summary}
  \begin{columns}
    \begin{column>{0.5\textwidth}
      \begin{mathbox>{Key Concepts}
        \textbf{Framebuffer is the final destination:}
        \begin{itemize>
          \item Collection of buffers (color, depth, stencil)
          \item Can render to screen or off-screen targets
          \item Supports multiple render targets (MRT)
          \item Managed through graphics API
        \end{itemize>
        
        \vspace{0.2cm}
        \textbf{Double buffering essential for:}
        \begin{itemize>
          \item Smooth animation
          \item Preventing visual artifacts
          \item Professional presentation quality
        \end{itemize>
      \end{mathbox}
    \end{column>
    \begin{column>{0.5\textwidth}
      \begin{conceptbox>{Modern Applications}
        \textbf{Advanced rendering techniques:}
        \begin{itemize>
          \item Deferred shading with G-buffer
          \item Post-processing effects
          \item Shadow mapping
          \item Screen-space reflections
          \item Temporal effects (motion blur, TAA)
        \end{itemize>
        
        \vspace{0.2cm}
        \textbf>Mobile considerations:}
        \begin{itemize>
          \item Tile-based deferred rendering (TBDR)
          \item Bandwidth-efficient formats
          \item Power consumption optimization
        \end{itemize>
      \end{conceptbox>
    \end{column>
  \end{columns>
  
  \vspace{0.3cm}
  
  \begin{raybox>{Pipeline Completion}
    \textbf{Framebuffer output completes the graphics pipeline:}
    
    Input Assembly → Vertex Shader → [Tessellation] → [Geometry] → Primitive Assembly → Clipping & Culling → Rasterization → Fragment Shader → Per-Fragment Ops → \textbf{Framebuffer Output}
    
    \vspace{0.2cm}
    \textbf{From 3D models to pixels on screen - the journey is complete!}
  \end{raybox>
\end{frame>
