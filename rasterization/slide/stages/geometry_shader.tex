\section{Geometry Shader}

\subsection{Overview}
\begin{frame}{Geometry Shader Stage}
  \begin{columns}
    \begin{column}{0.6\textwidth}
      \begin{raybox}{Geometry Shader}
        \textbf{Input:} Complete primitives (points, lines, triangles) \\
        \textbf{Output:} New primitives (can generate or discard)

        \vspace{0.3cm}
        \textbf{Purpose:}
        \begin{itemize}
          \item Generate new geometry from existing primitives
          \item Discard primitives based on conditions
          \item Transform primitive types
          \item Add detail or effects
          \end{itemize>

            \vspace{0.2cm}
            \textcolor{AccentColor}{\textbf{Optional programmable stage}}
          \end{raybox}
        \end{column>
          \begin{column}{0.4\textwidth}
            \begin{tikzpicture}[scale=0.7]
              % Input primitive
              \node at (0,3.5) {\textbf{Input}};
              \draw[thick, ObjectColor] (0,3) -- (0.8,2.4) -- (0,1.8) -- cycle;
              \fill[ObjectColor] (0,3) circle (2pt);
              \fill[ObjectColor] (0.8,2.4) circle (2pt);
              \fill[ObjectColor] (0,1.8) circle (2pt);
              \node[below] at (0.4,1.6) {\scriptsize 1 Triangle};

              \draw[->, thick, PrimaryColor] (0.4,1.4) -- (0.4,0.8);
              \node[right] at (0.6,1.1) {\scriptsize Geometry Shader};

              % Output primitives
              \node at (0,0.4) {\textbf{Output}};
              % Multiple triangles
              \draw[thick, SecondaryColor] (-0.5,-0.2) -- (0,-0.8) -- (-1,-0.8) -- cycle;
              \draw[thick, SecondaryColor] (0.5,-0.2) -- (1,-0.8) -- (0,-0.8) -- cycle;
              \draw[thick, SecondaryColor] (0,-0.2) -- (0.5,-0.5) -- (-0.5,-0.5) -- cycle;
              \node[below] at (0,-1) {\scriptsize 3 Triangles};
            \end{tikzpicture>
            \end{column>
            \end{columns>
            \end{frame>

              \subsection{Capabilities}
              \begin{frame}{Geometry Shader Capabilities}
                \begin{columns}
                  \begin{column}{0.5\textwidth}
                    \begin{conceptbox}{Primitive Generation}
                      \textbf{Can output:}
                      \begin{itemize>
                        \item Points
                        \item Line strips
                        \item Triangle strips
                        \end{itemize>

                          \vspace{0.2cm}
                          \textbf{Generation patterns:}
                          \begin{itemize}
                            \item One-to-many (subdivision)
                            \item One-to-one (pass-through)
                            \item Many-to-one (primitive merging)
                            \item Many-to-none (culling)
                            \end{itemize>
                            \end{conceptbox>
                            \end{column>
                              \begin{column>{0.5\textwidth}
                                  \begin{mathbox>{Input/Output Declaration}
                                      \small
                                      \textbf{Input layout specifiers:}
                                      \begin{itemize}
                                        \item \texttt{points} - Point primitives
                                        \item \texttt{lines} - Line primitives
                                        \item \texttt{triangles} - Triangle primitives
                                        \item \texttt{lines\_adjacency} - Lines with adjacency
                                        \item \texttt{triangles\_adjacency} - Triangles with adjacency
                                        \end{itemize>

                                          \vspace{0.2cm}
                                          \textbf{Output layout specifiers:}
                                          \begin{itemize>
                                            \item \texttt{points}
                                            \item \texttt{line\_strip}
                                            \item \texttt{triangle\_strip}
                                            \end{itemize>
                                            \end{mathbox>
                                            \end{column>
                                            \end{columns>
                                            \end{frame>

                                              \subsection{Use Cases}
                                              \begin{frame}{Common Geometry Shader Use Cases}
                                                \begin{columns}
                                                  \begin{column>{0.5\textwidth}
                                                      \begin{raybox>{Billboarding}
                                                          \textbf{Convert points to camera-facing quads}

                                                          \vspace{0.2cm}
                                                          \textbf{Applications:}
                                                          \begin{itemize}
                                                            \item Particle systems
                                                            \item Vegetation (grass, leaves)
                                                            \item UI elements in 3D space
                                                            \item Sprites and icons
                                                            \end{itemize>

                                                              \vspace{0.2cm}
                                                              \textbf{Process:}
                                                              \begin{enumerate}
                                                                \item Input: Single point
                                                                \item Calculate camera-facing orientation
                                                                \item Output: 4 vertices forming a quad
                                                                \end{enumerate>
                                                                \end{raybox>
                                                                \end{column>
                                                                  \begin{column>{0.5\textwidth}
                                                                      \begin{tikzpicture}[scale=0.6]
                                                                        % Camera
                                                                        \node[circle, fill=PrimaryColor!30, minimum size=0.5cm] (cam) at (0,2) {\tiny \faIcon{video}};

                                                                        % Point particle
                                                                        \fill[ObjectColor] (3,2) circle (2pt);
                                                                        \node[below] at (3,1.7) {\scriptsize Point};

                                                                        \draw[->, thick, AccentColor] (3,1.5) -- (3,0.8);
                                                                        \node[right] at (3.2,1.15) {\scriptsize Geometry Shader};

                                                                        % Generated quad
                                                                        \draw[thick, SecondaryColor, fill=SecondaryColor!20] (2.5,0.5) rectangle (3.5,0);
                                                                        \node[below] at (3,-0.2) {\scriptsize Camera-facing Quad};

                                                                        % Viewing direction
                                                                        \draw[dashed, gray] (cam) -- (3,0.25);
                                                                      \end{tikzpicture>
                                                                      \end{column>
                                                                      \end{columns>

                                                                        \vspace{0.3cm}

                                                                        \begin{columns}
                                                                          \begin{column>{0.5\textwidth}
                                                                              \begin{mathbox>{Primitive Amplification}
                                                                                  \textbf{Examples:}
                                                                                  \begin{itemize}
                                                                                    \item Fur/hair generation
                                                                                    \item Procedural geometry
                                                                                    \item Subdivision surfaces
                                                                                    \item Extrusion effects
                                                                                    \end{itemize>
                                                                                    \end{mathbox>
                                                                                    \end{column>
                                                                                      \begin{column>{0.5\textwidth}
                                                                                          \begin{conceptbox>{Primitive Culling}
                                                                                              \textbf{Discard primitives based on:}
                                                                                              \begin{itemize}
                                                                                                \item Viewing frustum
                                                                                                \item Back-face orientation
                                                                                                \item Distance from camera
                                                                                                \item Custom criteria
                                                                                                \end{itemize>
                                                                                                \end{conceptbox>
                                                                                                \end{column>
                                                                                                \end{columns>
                                                                                                \end{frame>

                                                                                                  \begin{frame>[fragile]{Geometry Shader Example: Billboarding}
                                                                                                      \begin{columns}
                                                                                                        \begin{column>{0.55\textwidth}
      \begin{minted}[fontsize=\tiny, bgcolor=LightGray!20]{glsl}
#version 330 core

// Input: points, Output: triangle strip (quad)
layout (points) in;
layout (triangle_strip, max_vertices = 4) out;

// Input from vertex shader
in vec3 vPosition[];
in float vSize[];
in vec4 vColor[];

// Output to fragment shader
out vec2 gTexCoord;
out vec4 gColor;

// Uniforms
uniform mat4 uView;
uniform mat4 uProjection;

void main() {
    vec3 worldPos = vPosition[0];
    float size = vSize[0];
    vec4 color = vColor[0];

    // Get camera right and up vectors from view matrix
    vec3 right = vec3(uView[0][0], uView[1][0], uView[2][0]);
    vec3 up = vec3(uView[0][1], uView[1][1], uView[2][1]);

    // Scale by particle size
    right *= size * 0.5;
    up *= size * 0.5;
      \end{minted>
    \end{column>
    \begin{column>{0.45\textwidth}
      \begin{minted}[fontsize=\tiny, bgcolor=LightGray!20]{glsl}
    // Generate quad vertices
    vec3 va = worldPos - right - up; // Bottom-left
    gl_Position = uProjection * uView * vec4(va, 1.0);
    gTexCoord = vec2(0.0, 0.0);
    gColor = color;
    EmitVertex();

    vec3 vb = worldPos + right - up; // Bottom-right
    gl_Position = uProjection * uView * vec4(vb, 1.0);
    gTexCoord = vec2(1.0, 0.0);
    gColor = color;
    EmitVertex();

    vec3 vc = worldPos - right + up; // Top-left
    gl_Position = uProjection * uView * vec4(vc, 1.0);
    gTexCoord = vec2(0.0, 1.0);
    gColor = color;
    EmitVertex();

    vec3 vd = worldPos + right + up; // Top-right
    gl_Position = uProjection * uView * vec4(vd, 1.0);
    gTexCoord = vec2(1.0, 1.0);
    gColor = color;
    EmitVertex();

    EndPrimitive();
}
      \end{minted>
    \end{column>
  \end{columns>
\end{frame>

\begin{frame>[fragile]{Geometry Shader Example: Normal Visualization}
  \begin{columns}
    \begin{column>{0.5\textwidth}
      \begin{minted}[fontsize=\tiny, bgcolor=LightGray!20]{glsl}
#version 330 core

// Input: triangles, Output: line strip
layout (triangles) in;
layout (line_strip, max_vertices = 6) out;

// Input from vertex shader
in vec3 vPosition[];
in vec3 vNormal[];

// Uniforms
uniform mat4 uMVP;
uniform float uNormalLength;

void main() {
    // Generate normal lines for each vertex
    for (int i = 0; i < 3; i++) {
        vec3 startPos = vPosition[i];
        vec3 endPos = startPos + vNormal[i] * uNormalLength;

        // Start point
        gl_Position = uMVP * vec4(startPos, 1.0);
        EmitVertex();

        // End point
        gl_Position = uMVP * vec4(endPos, 1.0);
        EmitVertex();

        EndPrimitive();
    }
}
      \end{minted>
    \end{column>
    \begin{column>{0.5\textwidth}
      \begin{tikzpicture}[scale=0.8]
        % Triangle
        \coordinate (A) at (0,0);
        \coordinate (B) at (2.5,0);
        \coordinate (C) at (1.25,2);

        \draw[thick, ObjectColor] (A) -- (B) -- (C) -- cycle;
        \fill[ObjectColor] (A) circle (2pt);
        \fill[ObjectColor] (B) circle (2pt);
        \fill[ObjectColor] (C) circle (2pt);

        % Normal vectors
        \draw[->, thick, PrimaryColor] (A) -- ++(225:1);
        \draw[->, thick, PrimaryColor] (B) -- ++(315:1);
        \draw[->, thick, PrimaryColor] (C) -- ++(90:1);

        \node[below left] at (A) {\scriptsize Vertex A};
        \node[below right] at (B) {\scriptsize Vertex B};
        \node[above] at (C) {\scriptsize Vertex C};

        \node[below] at (1.25,-0.5) {\scriptsize Normal visualization};
      \end{tikzpicture>
    \end{column>
  \end{columns>
\end{frame>

\begin{frame>{Geometry Shader Performance Considerations}
  \begin{columns}
    \begin{column>{0.5\textwidth}
      \begin{raybox>{Performance Impact}
        \textbf{Potential bottlenecks:}
        \begin{itemize>
          \item Sequential execution (not parallel)
          \item Vertex cache invalidation
          \item Dynamic vertex count
          \item Memory bandwidth
        \end{itemize>

        \vspace{0.2cm}
        \textbf{Modern alternatives:}
        \begin{itemize}
          \item Compute shaders
          \item Mesh shaders (newer GPUs)
          \item Instanced rendering
          \item Pre-computed geometry
        \end{itemize>
      \end{raybox>
    \end{column>
    \begin{column>{0.5\textwidth}
      \begin{conceptbox>{Best Practices}
        \begin{itemize}
          \item \textbf{Minimize output:} Generate only necessary vertices
          \item \textbf{Batch operations:} Process multiple primitives efficiently
          \item \textbf{Avoid complex branching:} Keep shader simple
          \item \textbf{Consider alternatives:} Use instancing when possible
          \item \textbf{Profile performance:} Measure actual impact
        \end{itemize>
      \end{conceptbox>
    \end{column>
  \end{columns>

  \vspace{0.3cm}

  \begin{mathbox>{When to Use Geometry Shaders}
    \textbf{Good for:}
    \begin{itemize}
      \item Small-scale geometry generation (particles, debug lines)
      \item Primitive-based effects (wireframe overlay, normal visualization)
      \item Situations where compute shaders aren't available
    \end{itemize>

    \textbf{Avoid for:}
    \begin{itemize}
      \item Large-scale geometry amplification
      \item Performance-critical applications
      \item When modern alternatives (mesh shaders) are available
    \end{itemize>
  \end{mathbox>
\end{frame>
