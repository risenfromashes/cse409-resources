\section{Clipping \& Culling}

\subsection{Overview}
\begin{frame}{Clipping \& Culling Stage}
  \begin{columns}
    \begin{column}{0.6\textwidth}
      \begin{raybox}{Clipping \& Culling}
        \textbf{Input:} Assembled primitives in clip space \\
        \textbf{Output:} Visible primitives ready for rasterization
        
        \vspace{0.3cm}
        \textbf{Purpose:}
        \begin{itemize}
          \item Remove geometry outside the view frustum
          \item Eliminate back-facing triangles
          \item Reduce work for subsequent stages
          \item Prevent rendering artifacts
        \end{itemize>
        
        \vspace{0.2cm}
        \textcolor{DarkGray}{\textbf{Configurable fixed-function stage}}
      \end{raybox}
    \end{column>
    \begin{column}{0.4\textwidth}
      \begin{tikzpicture}[scale=0.7]
        % View frustum
        \draw[thick, LightGray] (0,0) -- (2,1) -- (2,-1) -- cycle;
        \draw[thick, LightGray] (0,0) -- (2,1);
        \draw[thick, LightGray] (0,0) -- (2,-1);
        \draw[thick, LightGray] (2,1) -- (2,-1);
        
        % Triangles - inside, outside, clipped
        \draw[thick, SecondaryColor, fill=SecondaryColor!20] (0.8,0.3) -- (1.2,0.5) -- (1,0) -- cycle;
        \node[below] at (1,-0.2) {\scriptsize Inside};
        
        \draw[thick, DarkGray, dashed] (2.5,0.5) -- (3,0.8) -- (2.8,0.2) -- cycle;
        \node[below] at (2.8,0) {\scriptsize Outside};
        
        \draw[thick, AccentColor] (1.5,0.8) -- (2.5,1.2) -- (1.8,0.4) -- cycle;
        \draw[thick, AccentColor, dashed] (2,1) -- (2.5,1.2) -- (1.8,0.4);
        \node[below] at (1.8,0.2) {\scriptsize Clipped};
        
        \node[below] at (1,-1.3) {\scriptsize View Frustum};
      \end{tikzpicture}
    \end{column>
  \end{columns>
\end{frame>

\subsection{View Frustum Clipping}
\begin{frame}{View Frustum \& Clipping Planes}
  \begin{columns}
    \begin{column}{0.5\textwidth}
      \begin{conceptbox}{View Frustum Planes}
        \textbf{Six clipping planes:}
        \begin{itemize}
          \item \textbf{Near plane:} $z = z_{\text{near}}$
          \item \textbf{Far plane:} $z = z_{\text{far}}$
          \item \textbf{Left plane:} $-w \leq x$
          \item \textbf{Right plane:} $x \leq w$
          \item \textbf{Bottom plane:} $-w \leq y$
          \item \textbf{Top plane:} $y \leq w$
        \end{itemize}
        
        \vspace{0.2cm}
        Where $(x, y, z, w)$ are clip space coordinates after projection matrix.
      \end{conceptbox>
    \end{column>
    \begin{column>{0.5\textwidth}
      \begin{tikzpicture}[scale=0.7]
        % 3D frustum representation
        \coordinate (eye) at (0,0);
        
        % Near plane
        \draw[thick, PrimaryColor] (1,-0.5) -- (1,0.5);
        \node[below] at (1,-0.7) {\scriptsize Near};
        
        % Far plane  
        \draw[thick, SecondaryColor] (3,-1.5) -- (3,1.5);
        \node[below] at (3,-1.7) {\scriptsize Far};
        
        % Side planes
        \draw[thick, ObjectColor] (eye) -- (1,0.5) -- (3,1.5);
        \draw[thick, ObjectColor] (eye) -- (1,-0.5) -- (3,-1.5);
        
        % Top/bottom indicators
        \node[above] at (2,1) {\scriptsize Top};
        \node[below] at (2,-1) {\scriptsize Bottom};
        \node[left] at (2,0.7) {\scriptsize Right};
        \node[left] at (2,-0.7) {\scriptsize Left};
        
        % Camera
        \fill[PrimaryColor] (eye) circle (2pt);
        \node[left] at (eye) {\scriptsize Camera};
      \end{tikzpicture>
    \end{column>
  \end{columns>
  
  \vspace{0.3cm}
  
  \begin{mathbox>{Clipping Test}
    For a vertex in clip space $(x, y, z, w)$, it's inside the frustum if:
    \begin{align}
      -w \leq x \leq w \quad \text{and} \quad -w \leq y \leq w \quad \text{and} \quad 0 \leq z \leq w
    \end{align>
    
    \textbf{Note:} Some APIs use $-w \leq z \leq w$ (OpenGL) vs $0 \leq z \leq w$ (DirectX)
  \end{mathbox>
\end{frame>

\begin{frame>{Clipping Methods}
  \begin{columns}
    \begin{column>{0.5\textwidth}
      \begin{raybox>{Homogeneous Clipping}
        \textbf{Performed in clip space before perspective divide}
        
        \vspace{0.2cm}
        \textbf{Advantages:}
        \begin{itemize}
          \item Linear clipping plane equations
          \item Handles perspective projection correctly
          \item Standard hardware implementation
        \end{itemize>
        
        \vspace{0.2cm}
        \textbf{Process:}
        \begin{enumerate}
          \item Test vertices against clip planes
          \item Classify triangles as inside/outside/intersecting
          \item Split intersecting triangles
        \end{enumerate>
      \end{raybox>
    \end{column>
    \begin{column>{0.5\textwidth}
      \begin{conceptbox}{Screen Space Clipping}
        \textbf{Performed after perspective divide}
        
        \vspace{0.2cm}
        \textbf{Used for:}
        \begin{itemize}
          \item Viewport clipping
          \item Scissor test
          \item Guard band clipping
        \end{itemize>
        
        \vspace{0.2cm}
        \textbf{Simpler math:}
        \begin{itemize}
          \item 2D clipping algorithms
          \item Rectangular clip regions
          \item Post-rasterization culling
        \end{itemize>
      \end{conceptbox>
    \end{column>
  \end{columns>
\end{frame>

\subsection{Triangle Clipping}
\begin{frame>{Triangle Splitting}
  \begin{columns}
    \begin{column>{0.6\textwidth}
      \begin{mathbox>{Clipping Cases}
        \textbf{Three possible outcomes:}
        
        \vspace{0.2cm}
        \textbf{1. Completely inside:} Keep triangle unchanged
        
        \textbf{2. Completely outside:} Discard triangle
        
        \textbf{3. Intersecting:} Split into smaller triangles
        
        \vspace{0.3cm}
        \textbf{Intersection calculation:}
        For edge from $\mathbf{P_1}$ to $\mathbf{P_2}$ intersecting plane:
        \begin{align}
          t &= \frac{d_1}{d_1 - d_2} \\
          \mathbf{P_{new}} &= \mathbf{P_1} + t(\mathbf{P_2} - \mathbf{P_1})
        \end{align>
        Where $d_1, d_2$ are signed distances to the plane.
      \end{mathbox>
    \end{column>
    \begin{column>{0.4\textwidth}
      \begin{tikzpicture}[scale=0.7]
        % Clipping plane
        \draw[thick, LightGray] (0,-0.5) -- (3.5,-0.5);
        \node[below] at (1.75,-0.7) {\scriptsize Clipping Plane};
        
        % Original triangle (partially outside)
        \coordinate (A) at (0.5,1);
        \coordinate (B) at (2.5,1.5);
        \coordinate (C) at (1.5,-1.5);
        
        \draw[thick, ObjectColor, dashed] (A) -- (B) -- (C) -- cycle;
        \fill[ObjectColor] (A) circle (2pt);
        \fill[ObjectColor] (B) circle (2pt);
        \fill[ObjectColor] (C) circle (2pt);
        
        % Intersection points
        \coordinate (I1) at (1,-0.5);
        \coordinate (I2) at (2,-0.5);
        
        \fill[AccentColor] (I1) circle (2pt);
        \fill[AccentColor] (I2) circle (2pt);
        
        % Clipped triangle (kept part)
        \draw[thick, SecondaryColor, fill=SecondaryColor!20] (A) -- (B) -- (I2) -- (I1) -- cycle;
        
        \node[left] at (A) {\scriptsize A};
        \node[right] at (B) {\scriptsize B};
        \node[below] at (C) {\scriptsize C};
        \node[below] at (I1) {\scriptsize $I_1$};
        \node[below] at (I2) {\scriptsize $I_2$};
        
        \node[above] at (1.75,0.5) {\scriptsize Kept region};
      \end{tikzpicture>
    \end{column>
  \end{columns>
\end{frame>

\begin{frame}{Sutherland-Hodgman Clipping Algorithm}
  \begin{columns}
    \begin{column>{0.6\textwidth}
      \begin{raybox>{Algorithm Overview}
        \textbf{Clip polygon against each plane sequentially}
        
        \vspace{0.2cm}
        \textbf{For each clipping plane:}
        \begin{enumerate}
          \item Process each edge of current polygon
          \item Classify vertices as inside/outside
          \item Generate new vertices at intersections
          \item Build new polygon for next iteration
        \end{enumerate}
        
        \vspace{0.2cm}
        \textbf{Edge cases:}
        \begin{itemize}
          \item Inside → Inside: Keep second vertex
          \item Inside → Outside: Add intersection point
          \item Outside → Inside: Add intersection + second vertex
          \item Outside → Outside: Skip edge
        \end{itemize}
      \end{raybox>
    \end{column>
    \begin{column>{0.4\textwidth}
      \begin{tikzpicture}[scale=0.6]
        % Step 1: Original triangle
        \node at (0,3.5) {\textbf{Step 1: Original}};
        \draw[thick, ObjectColor] (0,3) -- (1.5,3.2) -- (0.8,2.2) -- cycle;
        
        % Step 2: Clip against left plane
        \node at (0,2.5) {\textbf{Step 2: Left plane}};
        \draw[thick, LightGray] (0.3,2.8) -- (0.3,1.8);
        \draw[thick, SecondaryColor] (0.3,2.9) -- (1.5,3.1) -- (0.8,2.1) -- (0.3,2.3) -- cycle;
        
        % Step 3: Clip against right plane
        \node at (0,1.5) {\textbf{Step 3: Right plane}};
        \draw[thick, LightGray] (1.2,1.8) -- (1.2,0.8);
        \draw[thick, AccentColor] (0.3,1.6) -- (1.2,1.7) -- (1.2,1.3) -- (0.8,0.8) -- (0.3,1.0) -- cycle;
        
        % Step 4: Final result
        \node at (0,0.5) {\textbf{Step 4: Final}};
        \draw[thick, PrimaryColor, fill=PrimaryColor!20] (0.3,0.3) -- (1.2,0.4) -- (1.2,0) -- (0.8,-0.5) -- (0.3,-0.3) -- cycle;
      \end{tikzpicture>
    \end{column>
  \end{columns>
\end{frame>

\subsection{Back-Face Culling}
\begin{frame}{Back-Face Culling}
  \begin{columns}
    \begin{column>{0.5\textwidth}
      \begin{conceptbox>{Back-Face Culling}
        \textbf{Input:} Triangles with face orientation \\
        \textbf{Output:} Only front-facing triangles
        
        \vspace{0.3cm}
        \textbf{Purpose:}
        \begin{itemize}
          \item Eliminate triangles facing away from camera
          \item Reduce overdraw and fragment processing
          \item Works well for closed meshes
          \item Significant performance improvement
        \end{itemize>
        
        \vspace{0.2cm}
        \textbf{Assumption:} Viewer cannot see inside objects
      \end{conceptbox>
    \end{column>
    \begin{column>{0.5\textwidth}
      \begin{tikzpicture}[scale=0.7]
        % Camera
        \node[circle, fill=PrimaryColor!30, minimum size=0.6cm] (cam) at (0,2) {\tiny \faIcon{eye}};
        
        % 3D object (cube-like)
        \coordinate (A) at (2,2.5);
        \coordinate (B) at (3,2.5);
        \coordinate (C) at (3.5,1.8);
        \coordinate (D) at (2.5,1.8);
        \coordinate (E) at (2.3,1.5);
        \coordinate (F) at (3.3,1.5);
        
        % Front faces (kept)
        \draw[thick, SecondaryColor, fill=SecondaryColor!30] (A) -- (B) -- (C) -- (D) -- cycle;
        \draw[thick, SecondaryColor, fill=SecondaryColor!20] (B) -- (C) -- (F) -- cycle;
        
        % Back faces (culled - shown dashed)
        \draw[thick, DarkGray, dashed] (A) -- (E) -- (F) -- (D) -- cycle;
        \draw[thick, DarkGray, dashed] (E) -- (F) -- (C);
        
        % Labels
        \node[above] at (2.5,2.7) {\scriptsize Front faces};
        \node[above] at (2.5,2.4) {\scriptsize (rendered)};
        
        \node[below] at (2.5,1.3) {\scriptsize Back faces};
        \node[below] at (2.5,1.0) {\scriptsize (culled)};
      \end{tikzpicture>
    \end{column>
  \end{columns>
\end{frame>

\begin{frame>[fragile]{OpenGL Clipping \& Culling Configuration}
  \begin{columns}
    \begin{column>{0.5\textwidth}
      \begin{minted}[fontsize=\tiny, bgcolor=LightGray!20]{c}
// Enable back-face culling
glEnable(GL_CULL_FACE);

// Configure which faces to cull
glCullFace(GL_BACK);    // Cull back faces (default)
glCullFace(GL_FRONT);   // Cull front faces
glCullFace(GL_FRONT_AND_BACK); // Cull both (nothing rendered)

// Set front face orientation
glFrontFace(GL_CCW);    // Counter-clockwise (default)
glFrontFace(GL_CW);     // Clockwise

// Disable culling
glDisable(GL_CULL_FACE);

// Depth testing (related to visibility)
glEnable(GL_DEPTH_TEST);
glDepthFunc(GL_LESS);   // Default depth comparison

// Scissor test (screen-space clipping)
glEnable(GL_SCISSOR_TEST);
glScissor(x, y, width, height);
      \end{minted>
    \end{column>
    \begin{column>{0.5\textwidth}
      \begin{minted}[fontsize=\tiny, bgcolor=LightGray!20]{c}
// Example: Typical 3D rendering setup
void setupRenderingState() {
    // Enable depth testing for 3D
    glEnable(GL_DEPTH_TEST);
    glDepthFunc(GL_LESS);
    
    // Enable back-face culling
    glEnable(GL_CULL_FACE);
    glCullFace(GL_BACK);
    glFrontFace(GL_CCW);
    
    // Clear buffers
    glClear(GL_COLOR_BUFFER_BIT | GL_DEPTH_BUFFER_BIT);
}

// Example: Rendering transparent objects
void renderTransparentObjects() {
    // Disable back-face culling for transparency
    glDisable(GL_CULL_FACE);
    
    // Enable blending
    glEnable(GL_BLEND);
    glBlendFunc(GL_SRC_ALPHA, GL_ONE_MINUS_SRC_ALPHA);
    
    // Render back-to-front
    renderSortedObjects();
    
    // Re-enable culling
    glEnable(GL_CULL_FACE);
}
      \end{minted}
    \end{column>
  \end{columns>
\end{frame>

\begin{frame>{Clipping \& Culling Performance}
  \begin{columns}
    \begin{column>{0.5\textwidth}
      \begin{raybox>{Performance Benefits}
        \textbf{Back-face culling:}
        \begin{itemize}
          \item Eliminates ~50\% of triangles (closed meshes)
          \item Reduces rasterization work
          \item Saves fragment shader invocations
          \item Nearly free performance gain
        \end{itemize>
        
        \vspace{0.2cm}
        \textbf{Frustum clipping:}
        \begin{itemize}
          \item Prevents off-screen geometry processing
          \item Reduces rasterization overhead
          \item Essential for correctness
          \item Hardware optimized
        \end{itemize}
      \end{raybox>
    \end{column>
    \begin{column>{0.5\textwidth}
      \begin{conceptbox>{Considerations}
        \textbf{When to disable culling:}
        \begin{itemize}
          \item Transparent objects
          \item Single-sided geometry viewed from both sides
          \item Objects with inconsistent winding
          \item Debugging geometry issues
        \end{itemize>
        
        \vspace{0.2cm}
        \textbf{Clipping limitations:}
        \begin{itemize}
          \item Can generate many small triangles
          \item May create degenerate cases
          \item Attribute interpolation complexity
        \end{itemize>
      \end{conceptbox>
    \end{column>
  \end{columns>
  
  \vspace{0.3cm}
  
  \begin{mathbox>{Best Practices}
    \begin{itemize}
      \item \textbf{Always enable back-face culling} for opaque 3D objects
      \item \textbf>Consistent winding order} across all geometry
      \item \textbf{Use view frustum culling} at object level before rendering
      \item \textbf{Consider occlusion culling} for complex scenes
    \end{itemize>
  \end{mathbox>
\end{frame>
