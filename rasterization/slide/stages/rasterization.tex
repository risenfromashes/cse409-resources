\section{Rasterization}

\subsection{Overview}
\begin{frame}{Rasterization Stage}
  \begin{columns}
    \begin{column>{0.6\textwidth}
      \begin{raybox}{Rasterization}
        \textbf{Input:} Clipped triangles in screen space \\
        \textbf{Output:} Fragments (potential pixels) with interpolated attributes
        
        \vspace{0.3cm}
        \textbf{Purpose:}
        \begin{itemize}
          \item Convert vector triangles to discrete fragments
          \item Determine which pixels are covered
          \item Interpolate vertex attributes across triangles
          \item Generate depth values for visibility testing
        \end{itemize}
        
        \vspace{0.2cm}
        \textcolor{RayColor}{\textbf{Fixed-function stage - hardware optimized}}
      \end{raybox>
    \end{column>
    \begin{column>{0.4\textwidth}
      \begin{tikzpicture}[scale=0.7]
        % Input triangle
        \node at (0,3.5) {\textbf{Triangle}};
        \draw[thick, ObjectColor] (0,3) -- (1.5,3.2) -- (0.8,2.2) -- cycle;
        \fill[ObjectColor] (0,3) circle (2pt);
        \fill[ObjectColor] (1.5,3.2) circle (2pt);
        \fill[ObjectColor] (0.8,2.2) circle (2pt);
        
        \draw[->, thick, RayColor] (0.8,1.8) -- (0.8,1.2);
        \node[right] at (1,1.5) {\scriptsize Rasterization};
        
        % Output fragments
        \node at (0,0.8) {\textbf{Fragments}};
        \begin{scope}[shift={(0,0)}]
          % Grid showing fragments
          \foreach \x in {0,0.2,...,1.4} {
            \foreach \y in {0,0.2,...,0.8} {
              \fill[LightGray] (\x,\y) rectangle ++(0.15,0.15);
            }
          }
          % Highlight fragments inside triangle
          \clip (0,0.6) -- (1.5,0.8) -- (0.8,-0.2) -- cycle;
          \foreach \x in {0,0.2,...,1.4} {
            \foreach \y in {0,0.2,...,0.8} {
              \fill[RayColor!60] (\x,\y) rectangle ++(0.15,0.15);
            }
          }
        \end{scope>
      \end{tikzpicture>
    \end{column>
  \end{columns>
\end{frame>

\subsection{Triangle Rasterization Process}
\begin{frame}{Triangle Rasterization Algorithm}
  \begin{columns}
    \begin{column>{0.5\textwidth}
      \begin{conceptbox}{Rasterization Steps}
        \textbf{1. Bounding Box Calculation}
        \begin{itemize}
          \item Find min/max x,y coordinates
          \item Limit to screen boundaries
        \end{itemize}
        
        \textbf{2. Fragment Generation}
        \begin{itemize}
          \item Test each pixel in bounding box
          \item Determine if pixel center is inside triangle
        \end{itemize}
        
        \textbf{3. Attribute Interpolation}
        \begin{itemize}
          \item Calculate barycentric coordinates
          \item Interpolate vertex attributes
        \end{itemize>
        
        \textbf{4. Depth Calculation}
        \begin{itemize>
          \item Interpolate z-values
          \item Prepare for depth testing
        \end{itemize>
      \end{conceptbox>
    \end{column>
    \begin{column>{0.5\textwidth}
      \begin{tikzpicture}[scale=0.7]
        % Triangle with vertices
        \coordinate (A) at (0.5,3);
        \coordinate (B) at (3,2.5);
        \coordinate (C) at (1.5,1);
        
        % Bounding box
        \draw[dashed, LightGray] (0.5,1) rectangle (3,3);
        \node[below left] at (0.5,1) {\scriptsize Bounding box};
        
        % Triangle
        \draw[thick, ObjectColor] (A) -- (B) -- (C) -- cycle;
        \fill[ObjectColor] (A) circle (2pt);
        \fill[ObjectColor] (B) circle (2pt);
        \fill[ObjectColor] (C) circle (2pt);
        
        % Grid of potential fragments
        \foreach \x in {0.75,1,...,2.75} {
          \foreach \y in {1.25,1.5,...,2.75} {
            \fill[LightGray] (\x,\y) circle (1pt);
          }
        }
        
        % Highlight inside fragments
        \foreach \x in {0.75,1,...,2.75} {
          \foreach \y in {1.25,1.5,...,2.75} {
            \coordinate (P) at (\x,\y);
            % Simplified inside test visualization
            \pgfmathparse{(\x-0.5)*(-0.5) + (\y-3)*(2.5) > 0 ? 1 : 0}
            \ifnum\pgfmathresult=1
              \pgfmathparse{(\x-3)*(-1.5) + (\y-2.5)*(1) > 0 ? 1 : 0}
              \ifnum\pgfmathresult=1
                \pgfmathparse{(\x-1.5)*(2.5) + (\y-1)*(-1.5) > 0 ? 1 : 0}
                \ifnum\pgfmathresult=1
                  \fill[RayColor] (\x,\y) circle (1.5pt);
                \fi
              \fi
            \fi
          }
        }
        
        % Labels
        \node[left] at (A) {\scriptsize A};
        \node[right] at (B) {\scriptsize B};
        \node[below] at (C) {\scriptsize C};
      \end{tikzpicture>
    \end{column>
  \end{columns>
\end{frame>

\subsection{Inside-Outside Testing}
\begin{frame}{Edge Functions \& Half-Space Test}
  \begin{columns}
    \begin{column>{0.6\textwidth}
      \begin{mathbox}{Edge Function Method}
        \textbf{For triangle with vertices A, B, C:}
        
        Define edge functions:
        \begin{align}
          E_{AB}(x,y) &= (B_x - A_x)(y - A_y) - (B_y - A_y)(x - A_x) \\
          E_{BC}(x,y) &= (C_x - B_x)(y - B_y) - (C_y - B_y)(x - B_x) \\
          E_{CA}(x,y) &= (A_x - C_x)(y - C_y) - (A_y - C_y)(x - C_x)
        \end{align}
        
        \textbf{Inside test:} Point $(x,y)$ is inside if:
        \begin{align}
          E_{AB}(x,y) \geq 0 \text{ and } E_{BC}(x,y) \geq 0 \text{ and } E_{CA}(x,y) \geq 0
        \end{align}
        
        (Assuming counter-clockwise winding)
      \end{mathbox>
    \end{column>
    \begin{column>{0.4\textwidth}
      \begin{tikzpicture}[scale=0.7]
        % Triangle
        \coordinate (A) at (0,0);
        \coordinate (B) at (3,0.5);
        \coordinate (C) at (1.5,2.5);
        
        \draw[thick, ObjectColor] (A) -- (B) -- (C) -- cycle;
        \fill[ObjectColor] (A) circle (2pt);
        \fill[ObjectColor] (B) circle (2pt);
        \fill[ObjectColor] (C) circle (2pt);
        
        % Test points
        \coordinate (P1) at (1.5,1); % Inside
        \coordinate (P2) at (0.5,2); % Outside
        
        \fill[SecondaryColor] (P1) circle (2pt);
        \fill[AccentColor] (P2) circle (2pt);
        
        % Edge normals (pointing inward)
        \coordinate (N1) at (1.5,-0.25);
        \coordinate (N2) at (2.8,1.25);
        \coordinate (N3) at (0.2,1.25);
        
        \draw[->, thick, PrimaryColor] (N1) -- ++(0,0.5);
        \draw[->, thick, PrimaryColor] (N2) -- ++(-0.5,0.3);
        \draw[->, thick, PrimaryColor] (N3) -- ++(0.5,0.3);
        
        % Labels
        \node[below left] at (A) {\scriptsize A};
        \node[below right] at (B) {\scriptsize B};
        \node[above] at (C) {\scriptsize C};
        \node[right] at (P1) {\scriptsize Inside};
        \node[left] at (P2) {\scriptsize Outside};
        
        \node[below] at (1.5,-0.7) {\scriptsize Edge normals};
      \end{tikzpicture>
    \end{column>
  \end{columns>
\end{frame>

\subsection{Barycentric Coordinates}
\begin{frame>{Barycentric Coordinates}
  \begin{columns}
    \begin{column>{0.5\textwidth}
      \begin{raybox>{Barycentric Interpolation}
        \textbf{Express any point P inside triangle as:}
        \begin{align}
          \mathbf{P} = \alpha \mathbf{A} + \beta \mathbf{B} + \gamma \mathbf{C}
        \end{align}
        
        where $\alpha + \beta + \gamma = 1$ and $\alpha, \beta, \gamma \geq 0$
        
        \vspace{0.3cm}
        \textbf{Calculation:}
        \begin{align}
          \alpha &= \frac{E_{BC}(\mathbf{P})}{E_{BC}(\mathbf{A})} \\
          \beta &= \frac{E_{CA}(\mathbf{P})}{E_{CA}(\mathbf{B})} \\
          \gamma &= \frac{E_{AB}(\mathbf{P})}{E_{AB}(\mathbf{C})}
        \end{align>
      \end{raybox>
    \end{column>
    \begin{column>{0.5\textwidth}
      \begin{tikzpicture}[scale=0.7]
        % Triangle
        \coordinate (A) at (0,0);
        \coordinate (B) at (3,0);
        \coordinate (C) at (1.5,2.5);
        
        \draw[thick, ObjectColor] (A) -- (B) -- (C) -- cycle;
        \fill[ObjectColor] (A) circle (3pt);
        \fill[ObjectColor] (B) circle (3pt);
        \fill[ObjectColor] (C) circle (3pt);
        
        % Test point
        \coordinate (P) at (1.2,0.8);
        \fill[PrimaryColor] (P) circle (3pt);
        
        % Barycentric visualization
        \draw[dashed, SecondaryColor] (P) -- (A);
        \draw[dashed, SecondaryColor] (P) -- (B);
        \draw[dashed, SecondaryColor] (P) -- (C);
        
        % Sub-triangles areas
        \draw[thick, AccentColor!50, fill=AccentColor!10] (P) -- (B) -- (C) -- cycle;
        \draw[thick, SecondaryColor!50, fill=SecondaryColor!10] (A) -- (P) -- (C) -- cycle;
        \draw[thick, PrimaryColor!50, fill=PrimaryColor!10] (A) -- (B) -- (P) -- cycle;
        
        % Labels
        \node[below left] at (A) {\scriptsize A};
        \node[below right] at (B) {\scriptsize B};
        \node[above] at (C) {\scriptsize C};
        \node[right] at (P) {\scriptsize P};
        
        % Area labels
        \node at (0.6,1.4) {\scriptsize $\beta$};
        \node at (2.2,1.4) {\scriptsize $\alpha$};
        \node at (1.2,0.3) {\scriptsize $\gamma$};
        
        \node[below] at (1.5,-0.3) {\scriptsize $\alpha + \beta + \gamma = 1$};
      \end{tikzpicture>
    \end{column>
  \end{columns>
\end{frame>

\subsection{Attribute Interpolation}
\begin{frame>{Fragment Attribute Interpolation}
  \begin{columns}
    \begin{column>{0.5\textwidth}
      \begin{conceptbox}{Interpolated Attributes}
        \textbf{Using barycentric coordinates:}
        
        For any vertex attribute $\mathbf{attr}$:
        \begin{align}
          \mathbf{attr}_P = \alpha \mathbf{attr}_A + \beta \mathbf{attr}_B + \gamma \mathbf{attr}_C
        \end{align}
        
        \vspace{0.2cm}
        \textbf{Common interpolated attributes:}
        \begin{itemize>
          \item Texture coordinates (UV)
          \item Vertex colors
          \item Normal vectors
          \item World positions
          \item Custom vertex data
        \end{itemize>
        
        \vspace{0.2cm}
        \textbf{Note:} Hardware performs perspective-correct interpolation
      \end{conceptbox>
    \end{column>
    \begin{column>{0.5\textwidth}
      \begin{tikzpicture}[scale=0.7]
        % Triangle with color gradient
        \coordinate (A) at (0,0);
        \coordinate (B) at (3,0);
        \coordinate (C) at (1.5,2.5);
        
        % Color interpolation visualization
        \shade[left color=red, right color=blue, bottom color=green] (A) -- (B) -- (C) -- cycle;
        
        \fill[red] (A) circle (3pt);
        \fill[blue] (B) circle (3pt);
        \fill[green] (C) circle (3pt);
        
        % Sample points showing interpolation
        \coordinate (P1) at (0.8,0.3);
        \coordinate (P2) at (2.2,0.8);
        \coordinate (P3) at (1.5,1.5);
        
        \fill[white] (P1) circle (2pt);
        \fill[white] (P2) circle (2pt);
        \fill[white] (P3) circle (2pt);
        
        % Labels
        \node[below left, white] at (A) {\scriptsize Red};
        \node[below right, white] at (B) {\scriptsize Blue};
        \node[above, white] at (C) {\scriptsize Green};
        
        \node[below] at (1.5,-0.5) {\scriptsize Color interpolation across triangle};
      \end{tikzpicture>
    \end{column>
  \end{columns>
\end{frame>

\subsection{Depth Interpolation}
\begin{frame>{Z-Buffer Value Interpolation}
  \begin{columns}
    \begin{column>{0.6\textwidth}
      \begin{mathbox}{Depth Value Generation}
        \textbf{Linear interpolation in screen space:}
        \begin{align}
          z_P = \alpha z_A + \beta z_B + \gamma z_C
        \end{align>
        
        where $z_A, z_B, z_C$ are the depth values of triangle vertices.
        
        \vspace{0.3cm}
        \textbf{Depth range mapping:}
        \begin{itemize}
          \item NDC depth: $[-1, 1]$ (OpenGL) or $[0, 1]$ (DirectX)
          \item Z-buffer depth: $[0, 1]$ (typically)
          \item Higher precision near near plane
        \end{itemize>
        
        \vspace{0.2cm}
        \textbf{Usage:} Depth testing, early Z rejection, depth-based effects
      \end{mathbox>
    \end{column>
    \begin{column>{0.4\textwidth}
      \begin{tikzpicture}[scale=0.7]
        % 3D visualization of depth interpolation
        \coordinate (A) at (0,0);
        \coordinate (B) at (3,0);
        \coordinate (C) at (1.5,2);
        
        % Triangle surface
        \fill[ObjectColor!30] (A) -- (B) -- (C) -- cycle;
        \draw[thick, ObjectColor] (A) -- (B) -- (C) -- cycle;
        
        % Depth visualization (heights)
        \coordinate (A3d) at (0,0.5);
        \coordinate (B3d) at (3,1.2);
        \coordinate (C3d) at (1.5,2.8);
        
        \draw[thick, PrimaryColor] (A) -- (A3d);
        \draw[thick, PrimaryColor] (B) -- (B3d);
        \draw[thick, PrimaryColor] (C) -- (C3d);
        
        \fill[PrimaryColor] (A3d) circle (2pt);
        \fill[PrimaryColor] (B3d) circle (2pt);
        \fill[PrimaryColor] (C3d) circle (2pt);
        
        % 3D triangle
        \fill[ObjectColor!50] (A3d) -- (B3d) -- (C3d) -- cycle;
        \draw[thick, ObjectColor] (A3d) -- (B3d) -- (C3d) -- cycle;
        
        % Sample point
        \coordinate (P) at (1.2,0.6);
        \coordinate (P3d) at (1.2,1.4);
        \draw[thick, SecondaryColor] (P) -- (P3d);
        \fill[SecondaryColor] (P3d) circle (2pt);
        
        % Labels
        \node[below] at (A) {\scriptsize $z_A$};
        \node[below] at (B) {\scriptsize $z_B$};
        \node[left] at (C) {\scriptsize $z_C$};
        \node[right] at (P3d) {\scriptsize $z_P$};
        
        \node[below] at (1.5,-0.5) {\scriptsize Depth interpolation};
      \end{tikzpicture>
    \end{column>
  \end{columns>
\end{frame>

\begin{frame>{Fragment Generation}
  \begin{columns}
    \begin{column>{0.5\textwidth}
      \begin{raybox>{Fragment Properties}
        \textbf{Each generated fragment contains:}
        \begin{itemize}
          \item \textbf{Screen coordinates} (x, y)
          \item \textbf{Depth value} (z)
          \item \textbf{Interpolated attributes} (colors, UVs, normals, etc.)
          \item \textbf{Coverage information} (for anti-aliasing)
        \end{itemize>
        
        \vspace{0.3cm}
        \textbf{Fragment ≠ Pixel:}
        \begin{itemize}
          \item Fragments are candidates for pixels
          \item Multiple fragments may affect one pixel
          \item Some fragments may be discarded
        \end{itemize>
      \end{raybox>
    \end{column>
    \begin{column>{0.5\textwidth}
      \begin{conceptbox}{Optimization Techniques}
        \textbf{Early optimizations:}
        \begin{itemize}
          \item \textbf{Hierarchical Z:} Reject blocks of fragments
          \item \textbf{Tile-based processing:} Process fragments in groups
          \item \textbf>Quad-based execution:} Process 2×2 fragment groups
          \item \textbf{Z-culling:} Hardware depth pre-test
        \end{itemize>
        
        \vspace{0.2cm}
        \textbf{Multi-sampling:}
        \begin{itemize>
          \item Multiple samples per pixel
          \item Anti-aliasing support
          \item Sub-pixel accuracy
        \end{itemize>
      \end{conceptbox>
    \end{column>
  \end{columns>
  
  \vspace{0.3cm}
  
  \begin{mathbox>{Rasterization Performance}
    \textbf{Factors affecting performance:}
    \begin{itemize}
      \item \textbf{Triangle size:} Tiny triangles are inefficient
      \item \textbf{Overdraw:} Multiple fragments per pixel
      \item \textbf{Attribute complexity:} More interpolated data = slower
      \item \textbf>Fragment shader cost:} Expensive shaders limit throughput
    \end{itemize>
  \end{mathbox>
\end{frame>
