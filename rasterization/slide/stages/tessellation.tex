\section{Tessellation Shader}

\subsection{Overview}
\begin{frame}{Tessellation Shader Stage}
  \begin{columns}
    \begin{column}{0.7\textwidth}
      \footnotesize
      \begin{raybox}{Tessellation Shader}
        \textbf{Input:} Primitives \\
        \textbf{Output:} Subdivided primitives with more vertices

        \vspace{0.3cm}
        \textbf{Purpose:}
        \begin{itemize}
          \item Subdivide low-poly meshes into high-poly
          \item Level-of-detail (LOD) based on distance
          \item Smooth curved surfaces
          \item Optional - Displacement mapping
        \end{itemize}
      \end{raybox}
    \end{column}
    \begin{column}{0.3\textwidth}
      \begin{tikzpicture}[scale=0.7]
        % Input patch
        \node at (0.8,3.5) {\textbf{Input Patch}};
        \draw[thick, ObjectColor] (0,3) -- (1.5,3) -- (0.75,2.2) -- cycle;
        \foreach \p in {(0,3),(1.5,3),(0.75,2.2)} {
          \fill[ObjectColor] \p circle (2pt);
        }

        \draw[->, thick, PrimaryColor] (0.75,1.8) -- (0.75,0);
        \node[right] at (1,0.9) {\scriptsize Tessellation};

        \node at (0.8,-2) {\textbf{Tessellated}};
        \begin{scope}[shift={(0,-1)}]
          \draw[thick, SecondaryColor] (0,0.6) -- (0.75,0.6) -- (1.5,0.6) -- (1.125,0) -- (0.75,-0.6) -- (0.375,0) -- cycle;
          \draw[thick, SecondaryColor] (0.375,0) -- (0.75,0.6);
          \draw[thick, SecondaryColor] (0.75,0.6) -- (1.125,0);
          \draw[thick, SecondaryColor] (0.375,0) -- (1.125,0);
          \foreach \p in {(0,0.6),(0.75,0.6),(1.5,0.6),(1.125,0),(0.75,-0.6),(0.375,0)} {
            \fill[SecondaryColor] \p circle (1.5pt);
          }
        \end{scope}
      \end{tikzpicture}
    \end{column}
  \end{columns}
\end{frame}

\subsection{Tessellation Pipeline}
\begin{frame}{Tessellation Sub-Stages}
  \begin{center}
    \footnotesize
    \begin{tikzpicture}[scale=0.8]
      % Vertex Shader output
      \node[rectangle, draw, fill=PrimaryColor!20, minimum width=1cm, minimum height=0.8cm] (vs) at (0,3) {
        \begin{minipage}{1.5cm}
          \centering
          \textbf{Vertex} \\
          \textbf{Shader}
        \end{minipage}
      };

      % Tessellation Control Shader
      \node[rectangle, draw, fill=SecondaryColor!30, minimum width=1cm, minimum height=0.8cm] (tcs) at (3,3) {
        \begin{minipage}{1.5cm}
          \centering
          \textbf{Tessellation} \\
          \textbf{Control} \\
          \textbf{Shader}
        \end{minipage}
      };

      % Tessellator (Fixed Function)
      \node[rectangle, draw, fill=LightGray, minimum width=1cm, minimum height=0.8cm] (tess) at (6,3) {
        \begin{minipage}{1.5cm}
          \centering
          \textbf{Tessellator} \\
          \scriptsize (Fixed)
        \end{minipage}
      };

      % Tessellation Evaluation Shader
      \node[rectangle, draw, fill=SecondaryColor!30, minimum width=1cm, minimum height=0.8cm] (tes) at (9,3) {
        \begin{minipage}{1.5cm}
          \centering
          \textbf{Tessellation} \\
          \textbf{Evaluation} \\
          \textbf{Shader}
        \end{minipage}
      };

      % Arrows
      \draw[->, thick] (vs) -- (tcs);
      \draw[->, thick] (tcs) -- (tess);
      \draw[->, thick] (tess) -- (tes);

      % Labels
      \node[below, align=center] at (1.5,4.75) {\scriptsize Control \\ \scriptsize Points};
      \node[below, align=center] at (4.5,4.75) {\scriptsize Tessellation \\ \scriptsize  Levels};
      \node[below, align=center] at (7.5,4.75) {\scriptsize Barycentric \\ \scriptsize Coords};

      % Descriptions
      \node[below, text width=1.8cm, align=center] at (3,2.1) {\scriptsize Determines subdivision levels};
      \node[below, text width=1.8cm, align=center] at (6,2.1) {\scriptsize Generates new vertices};
      \node[below, text width=1.8cm, align=center] at (9,2.1) {\scriptsize Computes final positions};
    \end{tikzpicture}
  \end{center}
\end{frame}

\subsection{Use Cases}
\begin{frame}{Tessellation Use Cases}
  \begin{columns}
    \begin{column}{0.6\textwidth}
      \small
      \begin{conceptbox}{Level of Detail (LOD)}
        \textbf{Adaptive subdivision based on:}
        \begin{itemize}
          \item Distance from camera
          \item Screen space size
          \item Surface curvature
          \item Performance requirements
        \end{itemize}

        \vspace{0.2cm}
        \textbf{Benefits:}
        \begin{itemize}
          \item Optimal vertex count
          \item Smooth transitions
          \item Better performance
        \end{itemize}
      \end{conceptbox}
    \end{column}
    \begin{column}{0.4\textwidth}
      \begin{tikzpicture}[scale=0.6]
        % Camera
        \node[circle, fill=PrimaryColor!30, minimum size=0.6cm] (cam) at (0,2) {\tiny \faIcon{video}};

        % Close object (high tessellation)
        \draw[thick, SecondaryColor] (2,3) -- (3,3) -- (2.5,2.2) -- cycle;
        \draw[thick, SecondaryColor] (2.25,2.6) -- (2.75,2.6);
        \draw[thick, SecondaryColor] (2.25,2.6) -- (2.5,3);
        \draw[thick, SecondaryColor] (2.75,2.6) -- (2.5,3);
        \draw[thick, SecondaryColor] (2.25,2.6) -- (2.5,2.2);
        \draw[thick, SecondaryColor] (2.75,2.6) -- (2.5,2.2);
        \node[below] at (2.5,4) {\scriptsize Close: High Detail};

        % Far object (low tessellation)

        \begin{scope}[shift={(1.5,-1)}]
          \draw[thick, ObjectColor] (2,0.5) -- (2.8,0.5) -- (2.4,0) -- cycle;
          \node[below] at (2.4,-0.2) {\scriptsize Far: Low Detail};
        \end{scope}

        % Distance lines
        \draw[dashed, gray] (cam) -- (2.5,2.6);
        \draw[dashed, gray] (cam) -- (3.43,-0.5);
      \end{tikzpicture}
    \end{column}
  \end{columns}
\end{frame}
