\section{Tessellation Shader}

\subsection{Overview}
\begin{frame}{Tessellation Shader Stage}
  \begin{columns}
    \begin{column}{0.6\textwidth}
      \begin{raybox}{Tessellation Shader}
        \textbf{Input:} Patch primitives (control points) \\
        \textbf{Output:} Subdivided primitives with more vertices

        \vspace{0.3cm}
        \textbf{Purpose:}
        \begin{itemize}
          \item Subdivide low-poly meshes into high-poly
          \item Level-of-detail (LOD) based on distance
          \item Smooth curved surfaces
          \item Displacement mapping
          \end{itemize>

            \vspace{0.2cm}
            \textcolor{SecondaryColor}{\textbf{Optional programmable stage}}
          \end{raybox}
        \end{column}
        \begin{column}{0.4\textwidth}
          \begin{tikzpicture}[scale=0.7]
            % Input patch
            \node at (0,3.5) {\textbf{Input Patch}};
            \draw[thick, ObjectColor] (0,3) -- (1.5,3) -- (0.75,2.2) -- cycle;
            \foreach \p in {(0,3),(1.5,3),(0.75,2.2)} {
              \fill[ObjectColor] \p circle (2pt);
            }

            \draw[->, thick, PrimaryColor] (0.75,1.8) -- (0.75,1.2);
            \node[right] at (1,1.5) {\scriptsize Tessellation};

            % Output subdivided
            \node at (0,0.8) {\textbf{Tessellated}};
            \begin{scope}[shift={(0,-0.3)}]
              \draw[thick, SecondaryColor] (0,0.6) -- (0.75,0.6) -- (1.5,0.6) -- (1.125,0) -- (0.75,-0.6) -- (0.375,0) -- cycle;
              \draw[thick, SecondaryColor] (0.375,0) -- (0.75,0.6);
              \draw[thick, SecondaryColor] (0.75,0.6) -- (1.125,0);
              \draw[thick, SecondaryColor] (1.125,0) -- (0.75,-0.6);
              \draw[thick, SecondaryColor] (0.75,-0.6) -- (0.375,0);
              \foreach \p in {(0,0.6),(0.75,0.6),(1.5,0.6),(1.125,0),(0.75,-0.6),(0.375,0)} {
                \fill[SecondaryColor] \p circle (1.5pt);
              }
            \end{scope}
          \end{tikzpicture}
        \end{column>
        \end{columns>
        \end{frame>

          \subsection{Tessellation Pipeline}
          \begin{frame}{Tessellation Sub-Stages}
            \begin{center}
              \begin{tikzpicture}[scale=0.8]
                % Vertex Shader output
                \node[rectangle, draw, fill=PrimaryColor!20, minimum width=2cm, minimum height=0.8cm] (vs) at (0,3) {
                  \begin{minipage}{1.8cm}
                    \centering
                    \textbf{Vertex} \\
                    \textbf{Shader}
                  \end{minipage>
                  };

                  % Tessellation Control Shader
                  \node[rectangle, draw, fill=SecondaryColor!30, minimum width=2cm, minimum height=0.8cm] (tcs) at (3,3) {
                    \begin{minipage}{1.8cm}
                      \centering
                      \textbf{Tessellation} \\
                      \textbf{Control}
                    \end{minipage>
                    };

                    % Tessellator (Fixed Function)
                    \node[rectangle, draw, fill=LightGray, minimum width=2cm, minimum height=0.8cm] (tess) at (6,3) {
                      \begin{minipage}{1.8cm}
                        \centering
                        \textbf{Tessellator} \\
                        \scriptsize (Fixed)
                      \end{minipage>
                      };

                      % Tessellation Evaluation Shader
                      \node[rectangle, draw, fill=SecondaryColor!30, minimum width=2cm, minimum height=0.8cm] (tes) at (9,3) {
                        \begin{minipage}{1.8cm}
                          \centering
                          \textbf{Tessellation} \\
                          \textbf{Evaluation}
                        \end{minipage>
                        };

                        % Arrows
                        \draw[->, thick] (vs) -- (tcs);
                        \draw[->, thick] (tcs) -- (tess);
                        \draw[->, thick] (tess) -- (tes);

                        % Labels
                        \node[below] at (1.5,2.5) {\scriptsize Control Points};
                        \node[below] at (4.5,2.5) {\scriptsize Tessellation Levels};
                        \node[below] at (7.5,2.5) {\scriptsize Barycentric Coords};

                        % Descriptions
                        \node[below, text width=1.8cm, align=center] at (3,1.8) {\scriptsize Determines subdivision levels};
                        \node[below, text width=1.8cm, align=center] at (6,1.8) {\scriptsize Generates new vertices};
                        \node[below, text width=1.8cm, align=center] at (9,1.8) {\scriptsize Computes final positions};
                      \end{tikzpicture>
                      \end{center>
                      \end{frame>

                        \subsection{Use Cases}
                        \begin{frame}{Tessellation Use Cases}
                          \begin{columns}
                            \begin{column}{0.5\textwidth}
                              \begin{conceptbox}{Level of Detail (LOD)}
                                \textbf{Adaptive subdivision based on:}
                                \begin{itemize}
                                  \item Distance from camera
                                  \item Screen space size
                                  \item Surface curvature
                                  \item Performance requirements
                                  \end{itemize>

                                    \vspace{0.2cm}
                                    \textbf{Benefits:}
                                    \begin{itemize}
                                      \item Optimal vertex count
                                      \item Smooth transitions
                                      \item Better performance
                                      \end{itemize>
                                      \end{conceptbox>
                                      \end{column>
                                        \begin{column>{0.5\textwidth}
                                            \begin{tikzpicture}[scale=0.6]
                                              % Camera
                                              \node[circle, fill=PrimaryColor!30, minimum size=0.6cm] (cam) at (0,2) {\tiny \faIcon{video}};

                                              % Close object (high tessellation)
                                              \draw[thick, SecondaryColor] (2,3) -- (3,3) -- (2.5,2.2) -- cycle;
                                              \draw[thick, SecondaryColor] (2.25,2.6) -- (2.75,2.6);
                                              \draw[thick, SecondaryColor] (2.25,2.6) -- (2.5,3);
                                              \draw[thick, SecondaryColor] (2.75,2.6) -- (2.5,3);
                                              \draw[thick, SecondaryColor] (2.25,2.6) -- (2.5,2.2);
                                              \draw[thick, SecondaryColor] (2.75,2.6) -- (2.5,2.2);
                                              \node[below] at (2.5,2) {\scriptsize Close: High Detail};

                                              % Far object (low tessellation)
                                              \draw[thick, ObjectColor] (2,0.5) -- (2.8,0.5) -- (2.4,0) -- cycle;
                                              \node[below] at (2.4,-0.2) {\scriptsize Far: Low Detail};

                                              % Distance lines
                                              \draw[dashed, gray] (cam) -- (2.5,2.6);
                                              \draw[dashed, gray] (cam) -- (2.4,0.25);
                                            \end{tikzpicture>
                                            \end{column>
                                            \end{columns>

                                              \vspace{0.3cm}

                                              \begin{columns}
                                                \begin{column>{0.5\textwidth}
                                                    \begin{mathbox>{Displacement Mapping}
                                                        \textbf{Process:}
                                                        \begin{enumerate}
                                                          \item Start with low-poly base mesh
                                                          \item Tessellate to add vertices
                                                          \item Displace vertices using height maps
                                                          \item Create detailed surface geometry
                                                          \end{enumerate>
                                                          \end{mathbox>
                                                          \end{column>
                                                            \begin{column>{0.5\textwidth}
                                                                \begin{raybox>{Curved Surfaces}
                                                                    \textbf{Applications:}
                                                                    \begin{itemize}
                                                                      \item Bezier patches
                                                                      \item NURBS surfaces
                                                                      \item Subdivision surfaces
                                                                      \item Smooth character models
                                                                      \end{itemize>
                                                                      \end{raybox>
                                                                      \end{column>
                                                                      \end{columns>
                                                                      \end{frame>

                                                                        \begin{frame>[fragile]{Tessellation Control Shader Example}
                                                                            \begin{columns}
                                                                              \begin{column>{0.5\textwidth}
      \begin{minted}[fontsize=\tiny, bgcolor=LightGray!20]{glsl}
#version 400 core

// Define patch size (number of control points)
layout (vertices = 3) out;

// Input from vertex shader
in vec3 vPosition[];
in vec3 vNormal[];

// Output to tessellation evaluation
out vec3 tcPosition[];
out vec3 tcNormal[];

uniform vec3 uCameraPos;
uniform float uTessellationFactor;

void main() {
    // Pass through control point data
    tcPosition[gl_InvocationID] = vPosition[gl_InvocationID];
    tcNormal[gl_InvocationID] = vNormal[gl_InvocationID];

    // Only first invocation sets tessellation levels
    if (gl_InvocationID == 0) {
        // Calculate distance-based LOD
        vec3 center = (vPosition[0] + vPosition[1] + vPosition[2]) / 3.0;
        float distance = length(uCameraPos - center);

        // Compute tessellation level
        float tessLevel = max(1.0, uTessellationFactor / distance);
        tessLevel = clamp(tessLevel, 1.0, 64.0);
      \end{minted>
    \end{column>
    \begin{column>{0.5\textwidth}
      \begin{minted}[fontsize=\tiny, bgcolor=LightGray!20]{glsl}
        // Set outer tessellation levels (edges)
        gl_TessLevelOuter[0] = tessLevel;
        gl_TessLevelOuter[1] = tessLevel;
        gl_TessLevelOuter[2] = tessLevel;

        // Set inner tessellation level
        gl_TessLevelInner[0] = tessLevel;
    }
}
      \end{minted}

      \vspace{0.3cm}

      \begin{minted}[fontsize=\tiny, bgcolor=LightGray!20]{glsl}
// Tessellation Evaluation Shader
#version 400 core

layout (triangles, equal_spacing, ccw) in;

in vec3 tcPosition[];
in vec3 tcNormal[];

out vec3 tePosition;
out vec3 teNormal;

uniform mat4 uMVP;

void main() {
    // Barycentric interpolation
    vec3 p0 = gl_TessCoord.x * tcPosition[0];
    vec3 p1 = gl_TessCoord.y * tcPosition[1];
    vec3 p2 = gl_TessCoord.z * tcPosition[2];
    tePosition = p0 + p1 + p2;

    vec3 n0 = gl_TessCoord.x * tcNormal[0];
    vec3 n1 = gl_TessCoord.y * tcNormal[1];
    vec3 n2 = gl_TessCoord.z * tcNormal[2];
    teNormal = normalize(n0 + n1 + n2);

    gl_Position = uMVP * vec4(tePosition, 1.0);
}
      \end{minted>
    \end{column>
  \end{columns>
\end{frame>

\begin{frame>{Tessellation Benefits \& Considerations}
  \begin{columns}
    \begin{column>{0.5\textwidth}
      \begin{raybox>{Benefits}
        \begin{itemize}
          \item \textbf{Memory efficient:} Store low-poly base mesh
          \item \textbf{Adaptive detail:} LOD based on viewing conditions
          \item \textbf{Smooth surfaces:} No visible polygon edges
          \item \textbf{Dynamic geometry:} Real-time subdivision
          \item \textbf{Displacement:} High-frequency surface details
        \end{itemize>
      \end{raybox>
    \end{column>
    \begin{column>{0.5\textwidth}
      \begin{conceptbox>{Considerations}
        \begin{itemize}
          \item \textbf{GPU overhead:} Additional processing stages
          \item \textbf{Driver support:} Not available on all hardware
          \item \textbf{Complexity:} More difficult to debug
          \item \textbf{Performance:} Can be expensive for high tessellation
          \item \textbf{Alternatives:} Pre-tessellated meshes, normal mapping
        \end{itemize>
      \end{conceptbox>
    \end{column>
  \end{columns>

  \vspace{0.3cm}

  \begin{mathbox>{When to Use Tessellation}
    \textbf{Good candidates:}
    \begin{itemize}
      \item Terrain rendering with height maps
      \item Character models requiring smooth surfaces
      \item Displacement mapping for surface details
      \item Dynamic LOD systems
    \end{itemize>

    \textbf{Avoid when:}
    \begin{itemize}
      \item Target platforms don't support it
      \item Pre-computed meshes are sufficient
      \item Performance is critical and tessellation overhead is too high
    \end{itemize>
  \end{mathbox>
\end{frame}
